\documentclass[11pt, a4paper]{article}
% =============================================================================
% GitHub Copilot 가이드 공통 프리앰블
% _preamble.tex - 모든 LaTeX 문서에서 % =============================================================================
% GitHub Copilot 가이드 공통 프리앰블
% _preamble.tex - 모든 LaTeX 문서에서 % =============================================================================
% GitHub Copilot 가이드 공통 프리앰블
% _preamble.tex - 모든 LaTeX 문서에서 \input{_preamble} 으로 재사용
% =============================================================================

% --- DOCUMENT VERSION CONTROL ---
% PDF 빌드 시점의 날짜와 시간(KST, UTC+9)이 자동으로 반영됩니다
% 참고: Overleaf 서버는 CET(UTC+1) 시간대를 사용하므로 +8로 보정합니다
\usepackage{datetime2}
\ExplSyntaxOn
\int_new:N \g_kst_hour_int
\int_new:N \g_kst_day_int
\int_new:N \g_kst_month_int
\int_new:N \g_kst_year_int
\int_gset:Nn \g_kst_hour_int { \the\time / 60 + 9 }
\int_gset:Nn \g_kst_day_int { \the\day }
\int_gset:Nn \g_kst_month_int { \the\month }
\int_gset:Nn \g_kst_year_int { \the\year }
\int_compare:nNnT { \g_kst_hour_int } > { 23 }
  {
    \int_gadd:Nn \g_kst_hour_int { -24 }
    \int_gincr:N \g_kst_day_int
  }
\newcommand{\docversion}{최종~수정:~\int_use:N \g_kst_year_int.~\int_compare:nNnTF { \g_kst_month_int } < { 10 } {0}{} \int_use:N \g_kst_month_int.~\int_compare:nNnTF { \g_kst_day_int } < { 10 } {0}{} \int_use:N \g_kst_day_int.
  \c_space_tl \int_compare:nNnTF { \g_kst_hour_int } < { 10 } {0}{} \int_use:N \g_kst_hour_int
  : \int_compare:nNnTF { \int_mod:nn {\the\time} {60} } < { 10 } {0}{} \int_eval:n { \int_mod:nn {\the\time} {60} }}
\newcommand{\docdate}{\int_use:N \g_kst_year_int 년~\int_compare:nNnTF { \g_kst_month_int } < { 10 } {0}{} \int_use:N \g_kst_month_int 월~\int_compare:nNnTF { \g_kst_day_int } < { 10 } {0}{} \int_use:N \g_kst_day_int 일}
\ExplSyntaxOff

% --- 저자 정보 (각 문서에서 재정의 가능) ---
\providecommand{\authors}{동의대학교 서진석(jsseo@deu.ac.kr)}

% --- UNIVERSAL PREAMBLE BLOCK ---
\usepackage[a4paper, top=3cm, bottom=3cm, left=2.5cm, right=2.5cm, headheight=14pt]{geometry}
\usepackage{fontspec}

% Language setup for Korean
\usepackage[korean, provide=*]{babel}

% Fonts: Using Noto Serif for body (formal) and Noto Sans for headers
\babelfont{rm}{Noto Serif CJK KR}
\babelfont{sf}{Noto Sans CJK KR}

% CJK 폰트는 이탤릭이 없으므로 폰트 대체 규칙 설정 (경고 방지)
\DeclareFontShape{TU}{NotoSerifCJKKR(0)}{m}{it}{<->ssub*NotoSerifCJKKR(0)/m/n}{}
\DeclareFontShape{TU}{NotoSerifCJKKR(0)}{m}{sl}{<->ssub*NotoSerifCJKKR(0)/m/n}{}
\DeclareFontShape{TU}{NotoSerifCJKKR(0)}{bx}{it}{<->ssub*NotoSerifCJKKR(0)/bx/n}{}
\DeclareFontShape{TU}{NotoSansCJKKR(0)}{m}{it}{<->ssub*NotoSansCJKKR(0)/m/n}{}
\DeclareFontShape{TU}{NotoSansCJKKR(0)}{m}{sl}{<->ssub*NotoSansCJKKR(0)/m/n}{}
\DeclareFontShape{TU}{NotoSansCJKKR(0)}{bx}{it}{<->ssub*NotoSansCJKKR(0)/bx/n}{}

% Essential Packages
\usepackage{enumitem}
\usepackage{booktabs}   % Professional tables
\usepackage{tabularx}   % Auto-width tables
\usepackage{titlesec}   % Section formatting
\usepackage{xcolor}     % Colors for visual hierarchy
\usepackage{fancyhdr}   % Headers and footers
\usepackage{setspace}   % Line spacing
\usepackage{longtable}  % For tables that might span pages
\usepackage{xltabular}  % For flexible width long tables
\usepackage{float}      % For [H] option to fix table position
\usepackage{kotex}      % Additional Korean support insurance
\usepackage[hidelinks]{hyperref}

% --- STYLING COMMANDS ---

% Define a formal navy blue for headers
\definecolor{eduNavy}{RGB}{0, 43, 91}

% Section styling
\titleformat{\section}
  {\Large\sffamily\bfseries\color{eduNavy}}
  {\thesection}{1em}{}
  [\vspace{-0.5em}\hrule height 1pt]

\titleformat{\subsection}
  {\large\sffamily\bfseries\color{eduNavy}}
  {\thesubsection}{1em}{}

\titleformat{\subsubsection}
  {\normalsize\sffamily\bfseries\color{eduNavy}}
  {\thesubsubsection}{1em}{}

% List settings
\setlist[itemize]{label=\textbullet, leftmargin=1.5em, itemsep=0.2em}
\setlist[enumerate]{label=\arabic*., leftmargin=1.5em, itemsep=0.2em}

% Line spacing for readability
\setstretch{1.2}

% Header and Footer
\pagestyle{fancy}
\fancyhf{}
\fancyhead[L]{\small\sffamily \authors}
\fancyhead[R]{\small\sffamily \docversion}
\fancyfoot[C]{\thepage}
\renewcommand{\headrulewidth}{0.5pt}

% =============================================================================
% 사용법:
% 각 문서에서 다음과 같이 사용합니다:
%
% \documentclass[11pt, a4paper]{article}
% \input{_preamble}
%
% % 필요시 저자 재정의
% \renewcommand{\authors}{작성자명 (email@example.com)}
%
% % 제목 설정
% \title{...}
% \author{\authors}
% \date{\docversion}
%
% \begin{document}
% \maketitle
% ...
% \end{document}
% =============================================================================
 으로 재사용
% =============================================================================

% --- DOCUMENT VERSION CONTROL ---
% PDF 빌드 시점의 날짜와 시간(KST, UTC+9)이 자동으로 반영됩니다
% 참고: Overleaf 서버는 CET(UTC+1) 시간대를 사용하므로 +8로 보정합니다
\usepackage{datetime2}
\ExplSyntaxOn
\int_new:N \g_kst_hour_int
\int_new:N \g_kst_day_int
\int_new:N \g_kst_month_int
\int_new:N \g_kst_year_int
\int_gset:Nn \g_kst_hour_int { \the\time / 60 + 9 }
\int_gset:Nn \g_kst_day_int { \the\day }
\int_gset:Nn \g_kst_month_int { \the\month }
\int_gset:Nn \g_kst_year_int { \the\year }
\int_compare:nNnT { \g_kst_hour_int } > { 23 }
  {
    \int_gadd:Nn \g_kst_hour_int { -24 }
    \int_gincr:N \g_kst_day_int
  }
\newcommand{\docversion}{최종~수정:~\int_use:N \g_kst_year_int.~\int_compare:nNnTF { \g_kst_month_int } < { 10 } {0}{} \int_use:N \g_kst_month_int.~\int_compare:nNnTF { \g_kst_day_int } < { 10 } {0}{} \int_use:N \g_kst_day_int.
  \c_space_tl \int_compare:nNnTF { \g_kst_hour_int } < { 10 } {0}{} \int_use:N \g_kst_hour_int
  : \int_compare:nNnTF { \int_mod:nn {\the\time} {60} } < { 10 } {0}{} \int_eval:n { \int_mod:nn {\the\time} {60} }}
\newcommand{\docdate}{\int_use:N \g_kst_year_int 년~\int_compare:nNnTF { \g_kst_month_int } < { 10 } {0}{} \int_use:N \g_kst_month_int 월~\int_compare:nNnTF { \g_kst_day_int } < { 10 } {0}{} \int_use:N \g_kst_day_int 일}
\ExplSyntaxOff

% --- 저자 정보 (각 문서에서 재정의 가능) ---
\providecommand{\authors}{동의대학교 서진석(jsseo@deu.ac.kr)}

% --- UNIVERSAL PREAMBLE BLOCK ---
\usepackage[a4paper, top=3cm, bottom=3cm, left=2.5cm, right=2.5cm, headheight=14pt]{geometry}
\usepackage{fontspec}

% Language setup for Korean
\usepackage[korean, provide=*]{babel}

% Fonts: Using Noto Serif for body (formal) and Noto Sans for headers
\babelfont{rm}{Noto Serif CJK KR}
\babelfont{sf}{Noto Sans CJK KR}

% CJK 폰트는 이탤릭이 없으므로 폰트 대체 규칙 설정 (경고 방지)
\DeclareFontShape{TU}{NotoSerifCJKKR(0)}{m}{it}{<->ssub*NotoSerifCJKKR(0)/m/n}{}
\DeclareFontShape{TU}{NotoSerifCJKKR(0)}{m}{sl}{<->ssub*NotoSerifCJKKR(0)/m/n}{}
\DeclareFontShape{TU}{NotoSerifCJKKR(0)}{bx}{it}{<->ssub*NotoSerifCJKKR(0)/bx/n}{}
\DeclareFontShape{TU}{NotoSansCJKKR(0)}{m}{it}{<->ssub*NotoSansCJKKR(0)/m/n}{}
\DeclareFontShape{TU}{NotoSansCJKKR(0)}{m}{sl}{<->ssub*NotoSansCJKKR(0)/m/n}{}
\DeclareFontShape{TU}{NotoSansCJKKR(0)}{bx}{it}{<->ssub*NotoSansCJKKR(0)/bx/n}{}

% Essential Packages
\usepackage{enumitem}
\usepackage{booktabs}   % Professional tables
\usepackage{tabularx}   % Auto-width tables
\usepackage{titlesec}   % Section formatting
\usepackage{xcolor}     % Colors for visual hierarchy
\usepackage{fancyhdr}   % Headers and footers
\usepackage{setspace}   % Line spacing
\usepackage{longtable}  % For tables that might span pages
\usepackage{xltabular}  % For flexible width long tables
\usepackage{float}      % For [H] option to fix table position
\usepackage{kotex}      % Additional Korean support insurance
\usepackage[hidelinks]{hyperref}

% --- STYLING COMMANDS ---

% Define a formal navy blue for headers
\definecolor{eduNavy}{RGB}{0, 43, 91}

% Section styling
\titleformat{\section}
  {\Large\sffamily\bfseries\color{eduNavy}}
  {\thesection}{1em}{}
  [\vspace{-0.5em}\hrule height 1pt]

\titleformat{\subsection}
  {\large\sffamily\bfseries\color{eduNavy}}
  {\thesubsection}{1em}{}

\titleformat{\subsubsection}
  {\normalsize\sffamily\bfseries\color{eduNavy}}
  {\thesubsubsection}{1em}{}

% List settings
\setlist[itemize]{label=\textbullet, leftmargin=1.5em, itemsep=0.2em}
\setlist[enumerate]{label=\arabic*., leftmargin=1.5em, itemsep=0.2em}

% Line spacing for readability
\setstretch{1.2}

% Header and Footer
\pagestyle{fancy}
\fancyhf{}
\fancyhead[L]{\small\sffamily \authors}
\fancyhead[R]{\small\sffamily \docversion}
\fancyfoot[C]{\thepage}
\renewcommand{\headrulewidth}{0.5pt}

% =============================================================================
% 사용법:
% 각 문서에서 다음과 같이 사용합니다:
%
% \documentclass[11pt, a4paper]{article}
% % =============================================================================
% GitHub Copilot 가이드 공통 프리앰블
% _preamble.tex - 모든 LaTeX 문서에서 \input{_preamble} 으로 재사용
% =============================================================================

% --- DOCUMENT VERSION CONTROL ---
% PDF 빌드 시점의 날짜와 시간(KST, UTC+9)이 자동으로 반영됩니다
% 참고: Overleaf 서버는 CET(UTC+1) 시간대를 사용하므로 +8로 보정합니다
\usepackage{datetime2}
\ExplSyntaxOn
\int_new:N \g_kst_hour_int
\int_new:N \g_kst_day_int
\int_new:N \g_kst_month_int
\int_new:N \g_kst_year_int
\int_gset:Nn \g_kst_hour_int { \the\time / 60 + 9 }
\int_gset:Nn \g_kst_day_int { \the\day }
\int_gset:Nn \g_kst_month_int { \the\month }
\int_gset:Nn \g_kst_year_int { \the\year }
\int_compare:nNnT { \g_kst_hour_int } > { 23 }
  {
    \int_gadd:Nn \g_kst_hour_int { -24 }
    \int_gincr:N \g_kst_day_int
  }
\newcommand{\docversion}{최종~수정:~\int_use:N \g_kst_year_int.~\int_compare:nNnTF { \g_kst_month_int } < { 10 } {0}{} \int_use:N \g_kst_month_int.~\int_compare:nNnTF { \g_kst_day_int } < { 10 } {0}{} \int_use:N \g_kst_day_int.
  \c_space_tl \int_compare:nNnTF { \g_kst_hour_int } < { 10 } {0}{} \int_use:N \g_kst_hour_int
  : \int_compare:nNnTF { \int_mod:nn {\the\time} {60} } < { 10 } {0}{} \int_eval:n { \int_mod:nn {\the\time} {60} }}
\newcommand{\docdate}{\int_use:N \g_kst_year_int 년~\int_compare:nNnTF { \g_kst_month_int } < { 10 } {0}{} \int_use:N \g_kst_month_int 월~\int_compare:nNnTF { \g_kst_day_int } < { 10 } {0}{} \int_use:N \g_kst_day_int 일}
\ExplSyntaxOff

% --- 저자 정보 (각 문서에서 재정의 가능) ---
\providecommand{\authors}{동의대학교 서진석(jsseo@deu.ac.kr)}

% --- UNIVERSAL PREAMBLE BLOCK ---
\usepackage[a4paper, top=3cm, bottom=3cm, left=2.5cm, right=2.5cm, headheight=14pt]{geometry}
\usepackage{fontspec}

% Language setup for Korean
\usepackage[korean, provide=*]{babel}

% Fonts: Using Noto Serif for body (formal) and Noto Sans for headers
\babelfont{rm}{Noto Serif CJK KR}
\babelfont{sf}{Noto Sans CJK KR}

% CJK 폰트는 이탤릭이 없으므로 폰트 대체 규칙 설정 (경고 방지)
\DeclareFontShape{TU}{NotoSerifCJKKR(0)}{m}{it}{<->ssub*NotoSerifCJKKR(0)/m/n}{}
\DeclareFontShape{TU}{NotoSerifCJKKR(0)}{m}{sl}{<->ssub*NotoSerifCJKKR(0)/m/n}{}
\DeclareFontShape{TU}{NotoSerifCJKKR(0)}{bx}{it}{<->ssub*NotoSerifCJKKR(0)/bx/n}{}
\DeclareFontShape{TU}{NotoSansCJKKR(0)}{m}{it}{<->ssub*NotoSansCJKKR(0)/m/n}{}
\DeclareFontShape{TU}{NotoSansCJKKR(0)}{m}{sl}{<->ssub*NotoSansCJKKR(0)/m/n}{}
\DeclareFontShape{TU}{NotoSansCJKKR(0)}{bx}{it}{<->ssub*NotoSansCJKKR(0)/bx/n}{}

% Essential Packages
\usepackage{enumitem}
\usepackage{booktabs}   % Professional tables
\usepackage{tabularx}   % Auto-width tables
\usepackage{titlesec}   % Section formatting
\usepackage{xcolor}     % Colors for visual hierarchy
\usepackage{fancyhdr}   % Headers and footers
\usepackage{setspace}   % Line spacing
\usepackage{longtable}  % For tables that might span pages
\usepackage{xltabular}  % For flexible width long tables
\usepackage{float}      % For [H] option to fix table position
\usepackage{kotex}      % Additional Korean support insurance
\usepackage[hidelinks]{hyperref}

% --- STYLING COMMANDS ---

% Define a formal navy blue for headers
\definecolor{eduNavy}{RGB}{0, 43, 91}

% Section styling
\titleformat{\section}
  {\Large\sffamily\bfseries\color{eduNavy}}
  {\thesection}{1em}{}
  [\vspace{-0.5em}\hrule height 1pt]

\titleformat{\subsection}
  {\large\sffamily\bfseries\color{eduNavy}}
  {\thesubsection}{1em}{}

\titleformat{\subsubsection}
  {\normalsize\sffamily\bfseries\color{eduNavy}}
  {\thesubsubsection}{1em}{}

% List settings
\setlist[itemize]{label=\textbullet, leftmargin=1.5em, itemsep=0.2em}
\setlist[enumerate]{label=\arabic*., leftmargin=1.5em, itemsep=0.2em}

% Line spacing for readability
\setstretch{1.2}

% Header and Footer
\pagestyle{fancy}
\fancyhf{}
\fancyhead[L]{\small\sffamily \authors}
\fancyhead[R]{\small\sffamily \docversion}
\fancyfoot[C]{\thepage}
\renewcommand{\headrulewidth}{0.5pt}

% =============================================================================
% 사용법:
% 각 문서에서 다음과 같이 사용합니다:
%
% \documentclass[11pt, a4paper]{article}
% \input{_preamble}
%
% % 필요시 저자 재정의
% \renewcommand{\authors}{작성자명 (email@example.com)}
%
% % 제목 설정
% \title{...}
% \author{\authors}
% \date{\docversion}
%
% \begin{document}
% \maketitle
% ...
% \end{document}
% =============================================================================

%
% % 필요시 저자 재정의
% \renewcommand{\authors}{작성자명 (email@example.com)}
%
% % 제목 설정
% \title{...}
% \author{\authors}
% \date{\docversion}
%
% \begin{document}
% \maketitle
% ...
% \end{document}
% =============================================================================
 으로 재사용
% =============================================================================

% --- DOCUMENT VERSION CONTROL ---
% PDF 빌드 시점의 날짜와 시간(KST, UTC+9)이 자동으로 반영됩니다
% 참고: Overleaf 서버는 CET(UTC+1) 시간대를 사용하므로 +8로 보정합니다
\usepackage{datetime2}
\ExplSyntaxOn
\int_new:N \g_kst_hour_int
\int_new:N \g_kst_day_int
\int_new:N \g_kst_month_int
\int_new:N \g_kst_year_int
\int_gset:Nn \g_kst_hour_int { \the\time / 60 + 9 }
\int_gset:Nn \g_kst_day_int { \the\day }
\int_gset:Nn \g_kst_month_int { \the\month }
\int_gset:Nn \g_kst_year_int { \the\year }
\int_compare:nNnT { \g_kst_hour_int } > { 23 }
  {
    \int_gadd:Nn \g_kst_hour_int { -24 }
    \int_gincr:N \g_kst_day_int
  }
\newcommand{\docversion}{최종~수정:~\int_use:N \g_kst_year_int.~\int_compare:nNnTF { \g_kst_month_int } < { 10 } {0}{} \int_use:N \g_kst_month_int.~\int_compare:nNnTF { \g_kst_day_int } < { 10 } {0}{} \int_use:N \g_kst_day_int.
  \c_space_tl \int_compare:nNnTF { \g_kst_hour_int } < { 10 } {0}{} \int_use:N \g_kst_hour_int
  : \int_compare:nNnTF { \int_mod:nn {\the\time} {60} } < { 10 } {0}{} \int_eval:n { \int_mod:nn {\the\time} {60} }}
\newcommand{\docdate}{\int_use:N \g_kst_year_int 년~\int_compare:nNnTF { \g_kst_month_int } < { 10 } {0}{} \int_use:N \g_kst_month_int 월~\int_compare:nNnTF { \g_kst_day_int } < { 10 } {0}{} \int_use:N \g_kst_day_int 일}
\ExplSyntaxOff

% --- 저자 정보 (각 문서에서 재정의 가능) ---
\providecommand{\authors}{동의대학교 서진석(jsseo@deu.ac.kr)}

% --- UNIVERSAL PREAMBLE BLOCK ---
\usepackage[a4paper, top=3cm, bottom=3cm, left=2.5cm, right=2.5cm, headheight=14pt]{geometry}
\usepackage{fontspec}

% Language setup for Korean
\usepackage[korean, provide=*]{babel}

% Fonts: Using Noto Serif for body (formal) and Noto Sans for headers
\babelfont{rm}{Noto Serif CJK KR}
\babelfont{sf}{Noto Sans CJK KR}

% CJK 폰트는 이탤릭이 없으므로 폰트 대체 규칙 설정 (경고 방지)
\DeclareFontShape{TU}{NotoSerifCJKKR(0)}{m}{it}{<->ssub*NotoSerifCJKKR(0)/m/n}{}
\DeclareFontShape{TU}{NotoSerifCJKKR(0)}{m}{sl}{<->ssub*NotoSerifCJKKR(0)/m/n}{}
\DeclareFontShape{TU}{NotoSerifCJKKR(0)}{bx}{it}{<->ssub*NotoSerifCJKKR(0)/bx/n}{}
\DeclareFontShape{TU}{NotoSansCJKKR(0)}{m}{it}{<->ssub*NotoSansCJKKR(0)/m/n}{}
\DeclareFontShape{TU}{NotoSansCJKKR(0)}{m}{sl}{<->ssub*NotoSansCJKKR(0)/m/n}{}
\DeclareFontShape{TU}{NotoSansCJKKR(0)}{bx}{it}{<->ssub*NotoSansCJKKR(0)/bx/n}{}

% Essential Packages
\usepackage{enumitem}
\usepackage{booktabs}   % Professional tables
\usepackage{tabularx}   % Auto-width tables
\usepackage{titlesec}   % Section formatting
\usepackage{xcolor}     % Colors for visual hierarchy
\usepackage{fancyhdr}   % Headers and footers
\usepackage{setspace}   % Line spacing
\usepackage{longtable}  % For tables that might span pages
\usepackage{xltabular}  % For flexible width long tables
\usepackage{float}      % For [H] option to fix table position
\usepackage{kotex}      % Additional Korean support insurance
\usepackage[hidelinks]{hyperref}

% --- STYLING COMMANDS ---

% Define a formal navy blue for headers
\definecolor{eduNavy}{RGB}{0, 43, 91}

% Section styling
\titleformat{\section}
  {\Large\sffamily\bfseries\color{eduNavy}}
  {\thesection}{1em}{}
  [\vspace{-0.5em}\hrule height 1pt]

\titleformat{\subsection}
  {\large\sffamily\bfseries\color{eduNavy}}
  {\thesubsection}{1em}{}

\titleformat{\subsubsection}
  {\normalsize\sffamily\bfseries\color{eduNavy}}
  {\thesubsubsection}{1em}{}

% List settings
\setlist[itemize]{label=\textbullet, leftmargin=1.5em, itemsep=0.2em}
\setlist[enumerate]{label=\arabic*., leftmargin=1.5em, itemsep=0.2em}

% Line spacing for readability
\setstretch{1.2}

% Header and Footer
\pagestyle{fancy}
\fancyhf{}
\fancyhead[L]{\small\sffamily \authors}
\fancyhead[R]{\small\sffamily \docversion}
\fancyfoot[C]{\thepage}
\renewcommand{\headrulewidth}{0.5pt}

% =============================================================================
% 사용법:
% 각 문서에서 다음과 같이 사용합니다:
%
% \documentclass[11pt, a4paper]{article}
% % =============================================================================
% GitHub Copilot 가이드 공통 프리앰블
% _preamble.tex - 모든 LaTeX 문서에서 % =============================================================================
% GitHub Copilot 가이드 공통 프리앰블
% _preamble.tex - 모든 LaTeX 문서에서 \input{_preamble} 으로 재사용
% =============================================================================

% --- DOCUMENT VERSION CONTROL ---
% PDF 빌드 시점의 날짜와 시간(KST, UTC+9)이 자동으로 반영됩니다
% 참고: Overleaf 서버는 CET(UTC+1) 시간대를 사용하므로 +8로 보정합니다
\usepackage{datetime2}
\ExplSyntaxOn
\int_new:N \g_kst_hour_int
\int_new:N \g_kst_day_int
\int_new:N \g_kst_month_int
\int_new:N \g_kst_year_int
\int_gset:Nn \g_kst_hour_int { \the\time / 60 + 9 }
\int_gset:Nn \g_kst_day_int { \the\day }
\int_gset:Nn \g_kst_month_int { \the\month }
\int_gset:Nn \g_kst_year_int { \the\year }
\int_compare:nNnT { \g_kst_hour_int } > { 23 }
  {
    \int_gadd:Nn \g_kst_hour_int { -24 }
    \int_gincr:N \g_kst_day_int
  }
\newcommand{\docversion}{최종~수정:~\int_use:N \g_kst_year_int.~\int_compare:nNnTF { \g_kst_month_int } < { 10 } {0}{} \int_use:N \g_kst_month_int.~\int_compare:nNnTF { \g_kst_day_int } < { 10 } {0}{} \int_use:N \g_kst_day_int.
  \c_space_tl \int_compare:nNnTF { \g_kst_hour_int } < { 10 } {0}{} \int_use:N \g_kst_hour_int
  : \int_compare:nNnTF { \int_mod:nn {\the\time} {60} } < { 10 } {0}{} \int_eval:n { \int_mod:nn {\the\time} {60} }}
\newcommand{\docdate}{\int_use:N \g_kst_year_int 년~\int_compare:nNnTF { \g_kst_month_int } < { 10 } {0}{} \int_use:N \g_kst_month_int 월~\int_compare:nNnTF { \g_kst_day_int } < { 10 } {0}{} \int_use:N \g_kst_day_int 일}
\ExplSyntaxOff

% --- 저자 정보 (각 문서에서 재정의 가능) ---
\providecommand{\authors}{동의대학교 서진석(jsseo@deu.ac.kr)}

% --- UNIVERSAL PREAMBLE BLOCK ---
\usepackage[a4paper, top=3cm, bottom=3cm, left=2.5cm, right=2.5cm, headheight=14pt]{geometry}
\usepackage{fontspec}

% Language setup for Korean
\usepackage[korean, provide=*]{babel}

% Fonts: Using Noto Serif for body (formal) and Noto Sans for headers
\babelfont{rm}{Noto Serif CJK KR}
\babelfont{sf}{Noto Sans CJK KR}

% CJK 폰트는 이탤릭이 없으므로 폰트 대체 규칙 설정 (경고 방지)
\DeclareFontShape{TU}{NotoSerifCJKKR(0)}{m}{it}{<->ssub*NotoSerifCJKKR(0)/m/n}{}
\DeclareFontShape{TU}{NotoSerifCJKKR(0)}{m}{sl}{<->ssub*NotoSerifCJKKR(0)/m/n}{}
\DeclareFontShape{TU}{NotoSerifCJKKR(0)}{bx}{it}{<->ssub*NotoSerifCJKKR(0)/bx/n}{}
\DeclareFontShape{TU}{NotoSansCJKKR(0)}{m}{it}{<->ssub*NotoSansCJKKR(0)/m/n}{}
\DeclareFontShape{TU}{NotoSansCJKKR(0)}{m}{sl}{<->ssub*NotoSansCJKKR(0)/m/n}{}
\DeclareFontShape{TU}{NotoSansCJKKR(0)}{bx}{it}{<->ssub*NotoSansCJKKR(0)/bx/n}{}

% Essential Packages
\usepackage{enumitem}
\usepackage{booktabs}   % Professional tables
\usepackage{tabularx}   % Auto-width tables
\usepackage{titlesec}   % Section formatting
\usepackage{xcolor}     % Colors for visual hierarchy
\usepackage{fancyhdr}   % Headers and footers
\usepackage{setspace}   % Line spacing
\usepackage{longtable}  % For tables that might span pages
\usepackage{xltabular}  % For flexible width long tables
\usepackage{float}      % For [H] option to fix table position
\usepackage{kotex}      % Additional Korean support insurance
\usepackage[hidelinks]{hyperref}

% --- STYLING COMMANDS ---

% Define a formal navy blue for headers
\definecolor{eduNavy}{RGB}{0, 43, 91}

% Section styling
\titleformat{\section}
  {\Large\sffamily\bfseries\color{eduNavy}}
  {\thesection}{1em}{}
  [\vspace{-0.5em}\hrule height 1pt]

\titleformat{\subsection}
  {\large\sffamily\bfseries\color{eduNavy}}
  {\thesubsection}{1em}{}

\titleformat{\subsubsection}
  {\normalsize\sffamily\bfseries\color{eduNavy}}
  {\thesubsubsection}{1em}{}

% List settings
\setlist[itemize]{label=\textbullet, leftmargin=1.5em, itemsep=0.2em}
\setlist[enumerate]{label=\arabic*., leftmargin=1.5em, itemsep=0.2em}

% Line spacing for readability
\setstretch{1.2}

% Header and Footer
\pagestyle{fancy}
\fancyhf{}
\fancyhead[L]{\small\sffamily \authors}
\fancyhead[R]{\small\sffamily \docversion}
\fancyfoot[C]{\thepage}
\renewcommand{\headrulewidth}{0.5pt}

% =============================================================================
% 사용법:
% 각 문서에서 다음과 같이 사용합니다:
%
% \documentclass[11pt, a4paper]{article}
% \input{_preamble}
%
% % 필요시 저자 재정의
% \renewcommand{\authors}{작성자명 (email@example.com)}
%
% % 제목 설정
% \title{...}
% \author{\authors}
% \date{\docversion}
%
% \begin{document}
% \maketitle
% ...
% \end{document}
% =============================================================================
 으로 재사용
% =============================================================================

% --- DOCUMENT VERSION CONTROL ---
% PDF 빌드 시점의 날짜와 시간(KST, UTC+9)이 자동으로 반영됩니다
% 참고: Overleaf 서버는 CET(UTC+1) 시간대를 사용하므로 +8로 보정합니다
\usepackage{datetime2}
\ExplSyntaxOn
\int_new:N \g_kst_hour_int
\int_new:N \g_kst_day_int
\int_new:N \g_kst_month_int
\int_new:N \g_kst_year_int
\int_gset:Nn \g_kst_hour_int { \the\time / 60 + 9 }
\int_gset:Nn \g_kst_day_int { \the\day }
\int_gset:Nn \g_kst_month_int { \the\month }
\int_gset:Nn \g_kst_year_int { \the\year }
\int_compare:nNnT { \g_kst_hour_int } > { 23 }
  {
    \int_gadd:Nn \g_kst_hour_int { -24 }
    \int_gincr:N \g_kst_day_int
  }
\newcommand{\docversion}{최종~수정:~\int_use:N \g_kst_year_int.~\int_compare:nNnTF { \g_kst_month_int } < { 10 } {0}{} \int_use:N \g_kst_month_int.~\int_compare:nNnTF { \g_kst_day_int } < { 10 } {0}{} \int_use:N \g_kst_day_int.
  \c_space_tl \int_compare:nNnTF { \g_kst_hour_int } < { 10 } {0}{} \int_use:N \g_kst_hour_int
  : \int_compare:nNnTF { \int_mod:nn {\the\time} {60} } < { 10 } {0}{} \int_eval:n { \int_mod:nn {\the\time} {60} }}
\newcommand{\docdate}{\int_use:N \g_kst_year_int 년~\int_compare:nNnTF { \g_kst_month_int } < { 10 } {0}{} \int_use:N \g_kst_month_int 월~\int_compare:nNnTF { \g_kst_day_int } < { 10 } {0}{} \int_use:N \g_kst_day_int 일}
\ExplSyntaxOff

% --- 저자 정보 (각 문서에서 재정의 가능) ---
\providecommand{\authors}{동의대학교 서진석(jsseo@deu.ac.kr)}

% --- UNIVERSAL PREAMBLE BLOCK ---
\usepackage[a4paper, top=3cm, bottom=3cm, left=2.5cm, right=2.5cm, headheight=14pt]{geometry}
\usepackage{fontspec}

% Language setup for Korean
\usepackage[korean, provide=*]{babel}

% Fonts: Using Noto Serif for body (formal) and Noto Sans for headers
\babelfont{rm}{Noto Serif CJK KR}
\babelfont{sf}{Noto Sans CJK KR}

% CJK 폰트는 이탤릭이 없으므로 폰트 대체 규칙 설정 (경고 방지)
\DeclareFontShape{TU}{NotoSerifCJKKR(0)}{m}{it}{<->ssub*NotoSerifCJKKR(0)/m/n}{}
\DeclareFontShape{TU}{NotoSerifCJKKR(0)}{m}{sl}{<->ssub*NotoSerifCJKKR(0)/m/n}{}
\DeclareFontShape{TU}{NotoSerifCJKKR(0)}{bx}{it}{<->ssub*NotoSerifCJKKR(0)/bx/n}{}
\DeclareFontShape{TU}{NotoSansCJKKR(0)}{m}{it}{<->ssub*NotoSansCJKKR(0)/m/n}{}
\DeclareFontShape{TU}{NotoSansCJKKR(0)}{m}{sl}{<->ssub*NotoSansCJKKR(0)/m/n}{}
\DeclareFontShape{TU}{NotoSansCJKKR(0)}{bx}{it}{<->ssub*NotoSansCJKKR(0)/bx/n}{}

% Essential Packages
\usepackage{enumitem}
\usepackage{booktabs}   % Professional tables
\usepackage{tabularx}   % Auto-width tables
\usepackage{titlesec}   % Section formatting
\usepackage{xcolor}     % Colors for visual hierarchy
\usepackage{fancyhdr}   % Headers and footers
\usepackage{setspace}   % Line spacing
\usepackage{longtable}  % For tables that might span pages
\usepackage{xltabular}  % For flexible width long tables
\usepackage{float}      % For [H] option to fix table position
\usepackage{kotex}      % Additional Korean support insurance
\usepackage[hidelinks]{hyperref}

% --- STYLING COMMANDS ---

% Define a formal navy blue for headers
\definecolor{eduNavy}{RGB}{0, 43, 91}

% Section styling
\titleformat{\section}
  {\Large\sffamily\bfseries\color{eduNavy}}
  {\thesection}{1em}{}
  [\vspace{-0.5em}\hrule height 1pt]

\titleformat{\subsection}
  {\large\sffamily\bfseries\color{eduNavy}}
  {\thesubsection}{1em}{}

\titleformat{\subsubsection}
  {\normalsize\sffamily\bfseries\color{eduNavy}}
  {\thesubsubsection}{1em}{}

% List settings
\setlist[itemize]{label=\textbullet, leftmargin=1.5em, itemsep=0.2em}
\setlist[enumerate]{label=\arabic*., leftmargin=1.5em, itemsep=0.2em}

% Line spacing for readability
\setstretch{1.2}

% Header and Footer
\pagestyle{fancy}
\fancyhf{}
\fancyhead[L]{\small\sffamily \authors}
\fancyhead[R]{\small\sffamily \docversion}
\fancyfoot[C]{\thepage}
\renewcommand{\headrulewidth}{0.5pt}

% =============================================================================
% 사용법:
% 각 문서에서 다음과 같이 사용합니다:
%
% \documentclass[11pt, a4paper]{article}
% % =============================================================================
% GitHub Copilot 가이드 공통 프리앰블
% _preamble.tex - 모든 LaTeX 문서에서 \input{_preamble} 으로 재사용
% =============================================================================

% --- DOCUMENT VERSION CONTROL ---
% PDF 빌드 시점의 날짜와 시간(KST, UTC+9)이 자동으로 반영됩니다
% 참고: Overleaf 서버는 CET(UTC+1) 시간대를 사용하므로 +8로 보정합니다
\usepackage{datetime2}
\ExplSyntaxOn
\int_new:N \g_kst_hour_int
\int_new:N \g_kst_day_int
\int_new:N \g_kst_month_int
\int_new:N \g_kst_year_int
\int_gset:Nn \g_kst_hour_int { \the\time / 60 + 9 }
\int_gset:Nn \g_kst_day_int { \the\day }
\int_gset:Nn \g_kst_month_int { \the\month }
\int_gset:Nn \g_kst_year_int { \the\year }
\int_compare:nNnT { \g_kst_hour_int } > { 23 }
  {
    \int_gadd:Nn \g_kst_hour_int { -24 }
    \int_gincr:N \g_kst_day_int
  }
\newcommand{\docversion}{최종~수정:~\int_use:N \g_kst_year_int.~\int_compare:nNnTF { \g_kst_month_int } < { 10 } {0}{} \int_use:N \g_kst_month_int.~\int_compare:nNnTF { \g_kst_day_int } < { 10 } {0}{} \int_use:N \g_kst_day_int.
  \c_space_tl \int_compare:nNnTF { \g_kst_hour_int } < { 10 } {0}{} \int_use:N \g_kst_hour_int
  : \int_compare:nNnTF { \int_mod:nn {\the\time} {60} } < { 10 } {0}{} \int_eval:n { \int_mod:nn {\the\time} {60} }}
\newcommand{\docdate}{\int_use:N \g_kst_year_int 년~\int_compare:nNnTF { \g_kst_month_int } < { 10 } {0}{} \int_use:N \g_kst_month_int 월~\int_compare:nNnTF { \g_kst_day_int } < { 10 } {0}{} \int_use:N \g_kst_day_int 일}
\ExplSyntaxOff

% --- 저자 정보 (각 문서에서 재정의 가능) ---
\providecommand{\authors}{동의대학교 서진석(jsseo@deu.ac.kr)}

% --- UNIVERSAL PREAMBLE BLOCK ---
\usepackage[a4paper, top=3cm, bottom=3cm, left=2.5cm, right=2.5cm, headheight=14pt]{geometry}
\usepackage{fontspec}

% Language setup for Korean
\usepackage[korean, provide=*]{babel}

% Fonts: Using Noto Serif for body (formal) and Noto Sans for headers
\babelfont{rm}{Noto Serif CJK KR}
\babelfont{sf}{Noto Sans CJK KR}

% CJK 폰트는 이탤릭이 없으므로 폰트 대체 규칙 설정 (경고 방지)
\DeclareFontShape{TU}{NotoSerifCJKKR(0)}{m}{it}{<->ssub*NotoSerifCJKKR(0)/m/n}{}
\DeclareFontShape{TU}{NotoSerifCJKKR(0)}{m}{sl}{<->ssub*NotoSerifCJKKR(0)/m/n}{}
\DeclareFontShape{TU}{NotoSerifCJKKR(0)}{bx}{it}{<->ssub*NotoSerifCJKKR(0)/bx/n}{}
\DeclareFontShape{TU}{NotoSansCJKKR(0)}{m}{it}{<->ssub*NotoSansCJKKR(0)/m/n}{}
\DeclareFontShape{TU}{NotoSansCJKKR(0)}{m}{sl}{<->ssub*NotoSansCJKKR(0)/m/n}{}
\DeclareFontShape{TU}{NotoSansCJKKR(0)}{bx}{it}{<->ssub*NotoSansCJKKR(0)/bx/n}{}

% Essential Packages
\usepackage{enumitem}
\usepackage{booktabs}   % Professional tables
\usepackage{tabularx}   % Auto-width tables
\usepackage{titlesec}   % Section formatting
\usepackage{xcolor}     % Colors for visual hierarchy
\usepackage{fancyhdr}   % Headers and footers
\usepackage{setspace}   % Line spacing
\usepackage{longtable}  % For tables that might span pages
\usepackage{xltabular}  % For flexible width long tables
\usepackage{float}      % For [H] option to fix table position
\usepackage{kotex}      % Additional Korean support insurance
\usepackage[hidelinks]{hyperref}

% --- STYLING COMMANDS ---

% Define a formal navy blue for headers
\definecolor{eduNavy}{RGB}{0, 43, 91}

% Section styling
\titleformat{\section}
  {\Large\sffamily\bfseries\color{eduNavy}}
  {\thesection}{1em}{}
  [\vspace{-0.5em}\hrule height 1pt]

\titleformat{\subsection}
  {\large\sffamily\bfseries\color{eduNavy}}
  {\thesubsection}{1em}{}

\titleformat{\subsubsection}
  {\normalsize\sffamily\bfseries\color{eduNavy}}
  {\thesubsubsection}{1em}{}

% List settings
\setlist[itemize]{label=\textbullet, leftmargin=1.5em, itemsep=0.2em}
\setlist[enumerate]{label=\arabic*., leftmargin=1.5em, itemsep=0.2em}

% Line spacing for readability
\setstretch{1.2}

% Header and Footer
\pagestyle{fancy}
\fancyhf{}
\fancyhead[L]{\small\sffamily \authors}
\fancyhead[R]{\small\sffamily \docversion}
\fancyfoot[C]{\thepage}
\renewcommand{\headrulewidth}{0.5pt}

% =============================================================================
% 사용법:
% 각 문서에서 다음과 같이 사용합니다:
%
% \documentclass[11pt, a4paper]{article}
% \input{_preamble}
%
% % 필요시 저자 재정의
% \renewcommand{\authors}{작성자명 (email@example.com)}
%
% % 제목 설정
% \title{...}
% \author{\authors}
% \date{\docversion}
%
% \begin{document}
% \maketitle
% ...
% \end{document}
% =============================================================================

%
% % 필요시 저자 재정의
% \renewcommand{\authors}{작성자명 (email@example.com)}
%
% % 제목 설정
% \title{...}
% \author{\authors}
% \date{\docversion}
%
% \begin{document}
% \maketitle
% ...
% \end{document}
% =============================================================================

%
% % 필요시 저자 재정의
% \renewcommand{\authors}{작성자명 (email@example.com)}
%
% % 제목 설정
% \title{...}
% \author{\authors}
% \date{\docversion}
%
% \begin{document}
% \maketitle
% ...
% \end{document}
% =============================================================================


% --- 문서별 메타데이터 ---
\renewcommand{\authors}{동의대학교 서진석(jsseo@deu.ac.kr)}

\title{\vspace{-2.5cm} \Huge\sffamily\bfseries GitHub Copilot 웹 활용 가이드 \\ \large\vspace{0.2em} \textmd{브라우저에서 바로 사용하는 AI 채팅 서비스}\vspace{-0.5em}}
\author{\vspace{-0.5em}\authors}
\date{\vspace{-0.5em}\docversion}

\begin{document}

\maketitle
\thispagestyle{empty}
\vspace{-1.5cm}

\section{개요}
\subsection{GitHub Copilot이란?}
GitHub Copilot(깃허브 코파일럿)은 웹 브라우저에서 OpenAI, Google, Anthropic 등 다양한 최신 AI 모델을 선택하여 사용할 수 있는 범용 AI 채팅 서비스입니다.

\subsection{주요 특징}
\begin{itemize}
    \item \textbf{무료 라이선스}: 대학 구성원(교수, 학생)에게 \textbf{GitHub Copilot Pro} 무료 제공
    \item \textbf{다양한 AI 모델}: GPT-5, Claude, Gemini 등 최신 모델을 한 곳에서 선택하여 사용 가능
\end{itemize}

\subsection{지원 자격}
\begin{table}[H]
    \centering
    \renewcommand{\arraystretch}{1.1}
    \begin{tabularx}{\textwidth}{@{}l l X@{}}
        \toprule
        \textbf{구분} & \textbf{신청 가능 여부} & \textbf{필요 서류 및 비고} \\
        \midrule
        학부생 & \textbf{가능} & 영문 재학증명서 \\
        대학원생 (석사/박사) & \textbf{가능} & 영문 재학증명서 \\
        전임교원 & \textbf{가능} & 영문 재직증명서 \\
        비전임교원 (강사, 겸임) & 직접 확인 필요 & 승인 사례가 일관되지 않음 \\
        연구원 & 직접 확인 필요 & 고용 형태에 따라 다름 \\
        행정 직원 & \textbf{불가} & 유료 라이선스 구매 필요 \\
        휴학생 & \textbf{불가} & 복학 후 재학증명서로 신청 \\
        졸업생 & \textbf{불가} & 재학 중에만 사용 가능 \\
        \bottomrule
    \end{tabularx}
\end{table}

\subsection{GitHub Copilot Pro 라이선스 혜택}
\begin{table}[H]
    \centering
    \renewcommand{\arraystretch}{1.2}
    \small
    \begin{tabularx}{\textwidth}{@{}l X X@{}}
        \toprule
        \textbf{구분} & \textbf{무료 사용자 (Basic)} & \textbf{Pro 사용자 (대학 구성원)} \\
        \midrule
        \textbf{대화 한도} & 기본 모델 월 50회 제한 & \textbf{기본 모델 무제한, 프리미엄 월 300회} \\
        \textbf{사용 모델} & 기본 모델만 (GPT-5 mini 등) & \textbf{전체 모델} (Claude Opus, Gemini Pro 등) \\
        \textbf{데이터 보안} & 학습에 사용될 수 있음 & \textbf{학습 데이터로 사용 안 됨} \\
        \bottomrule
    \end{tabularx}
\end{table}

\section{무료 사용 신청 가이드}
신청 후 승인까지는 통상 영업일 기준 \textbf{1\textasciitilde3일}이 소요됩니다.

\subsection{사전 준비 사항}
\begin{itemize}
    \item \textbf{학교 공식 이메일}: \texttt{@deu.ac.kr}, \texttt{@g.deu.ac.kr} 등 공식 도메인 사용
    \item \textbf{영문 증명 서류}: \textbf{영문(English)} 재학증명서(학생) 또는 재직증명서(교수)
    \begin{itemize}
        \item \textbf{필수 준비물}: \textbf{스캔 파일(JPG)}과 \textbf{종이 인쇄본}을 모두 준비하십시오.
        \item \textbf{기기 준비}: 서류 제출 단계에서 파일 업로드 대신 \textbf{카메라 촬영}만 허용되는 경우가 있습니다.
        \begin{itemize}
            \item 카메라가 없는 PC 사용 시 진행이 막힐 수 있으므로, \textbf{스마트폰}이나 \textbf{노트북} 사용을 권장합니다.
            \item PC에서 진행 중 문제가 생기면 스마트폰으로 다시 로그인하여 이어서 진행할 수 있습니다.
        \end{itemize}
        \item \textbf{학생 추가 서류}: 학생의 경우 \textbf{학생증}과 \textbf{영문 재학증명서}를 함께 준비하면 도움이 됩니다.
        \item \textbf{스캔 시 중요}: 반드시 \textbf{JPG 형식}으로 \textbf{컬러 모드} 스캔하고 \textbf{자동 문서 보정 기능} 비활성화. PDF는 업로드 불가. 학교 로고(워터마크)가 사라지면 위조 문서로 의심받아 승인이 거절됨.
    \end{itemize}
    \item \textbf{인증 앱}: 스마트폰에 \textit{Microsoft Authenticator} 설치 (회원가입 불필요, 설치만)
\end{itemize}

\subsection{GitHub 영문 인터페이스 안내}
GitHub은 영문으로 제공됩니다. 주요 메뉴 번역을 참고하십시오:

\begin{table}[H]
    \centering
    \begin{tabularx}{\textwidth}{@{}l l X@{}}
        \toprule
        \textbf{영문} & \textbf{한국어} & \textbf{위치/용도} \\
        \midrule
        Sign up & 회원가입 & 첫 화면 우측 상단 \\
        Sign in & 로그인 & 첫 화면 우측 상단 \\
        Settings & 설정 & (GitHub 로그인 후 화면 최 우측 상단)프로필 아이콘 클릭 후 메뉴 \\
        Profile & 프로필 & 개인정보 설정 \\
        Billing & 결제 & 결제 정보 설정 \\
        Password and authentication & 비밀번호 및 인증 & 2단계 인증 설정 \\
        Submit application & 신청서 제출 & 최종 제출 버튼 \\
        Download & 다운로드 & 복구 코드 저장 버튼 \\
        Enable two-factor authentication & 2단계 인증 활성화 & 보안 설정 \\
        \bottomrule
    \end{tabularx}
\end{table}

\subsection{등록 절차}
\begin{enumerate}
    \item \textbf{GitHub 회원가입}: \href{https://github.com}{github.com}에서 학교 이메일로 가입 후 이메일 인증
    \inlineImage{01-github-signup.png}
    \item \textbf{프로필 설정}: GitHub 로그인 $\rightarrow$ 우측 상단 프로필 아이콘 클릭 $\rightarrow$ Settings $\rightarrow$ Public profile에서 \textbf{Name}을 \textbf{증명서와 동일한 영문 이름}으로 입력
    \inlineImage{02-profile-settings.png}
    \item \textbf{결제 정보 입력}: Settings $\rightarrow$ Billing and licensing $\rightarrow$ Payment information에서 \textbf{증명서와 동일한 영문 이름} 입력 (결제 수단 등록 불필요)
    \inlineImage{03-billing-info.png}
    \item \textbf{2단계 인증(2FA) 설정}:
    \begin{itemize}
        \item 프로필 $\rightarrow$ \textbf{Settings} $\rightarrow$ \textbf{Password and authentication} $\rightarrow$ \textbf{Two-factor authentication} 활성화
        \item Authenticator 앱에서 QR 코드 스캔 후 6자리 숫자 입력
        \item \textbf{복구 코드 저장 (필수)}:
        \begin{itemize}
            \item 6자리 인증 코드 입력 후 \textbf{복구 코드 다운로드} 화면이 표시됨
            \item 반드시 \textbf{Download} 버튼을 클릭하여 파일로 저장
            \item 저장 위치: USB, 클라우드(OneDrive, Google Drive), 또는 인쇄물
            \item \textbf{복구 코드가 필요한 상황}:
            \begin{itemize}
                \item 휴대폰 분실 또는 교체 시
                \item 인증 앱(예: Microsoft Authenticator) 삭제 시
                \item 인증 앱 데이터 초기화 시
            \end{itemize}
            \item \textit{주의: 복구 코드 없이 인증 수단을 분실하면 계정 접근이 영구적으로 불가능해질 수 있음}
            \item \textit{권장: 복구 코드 화면을 캡처하여 별도 보관}
        \end{itemize}
    \end{itemize}
    \inlineImage{04-2fa-setup.png}
    \inlineImage{05-recovery-codes.png}
    \item \textbf{GitHub Education 신청}: GitHub 웹페이지에 로그인 된 상태에서 $\rightarrow$ 웹 브라우저 주소창에 다음 주소 입력(\href{https://github.com/education}{github.com/education} $\rightarrow$ \textbf{Join GitHub Education} 버튼 클릭 $\rightarrow$ (다음 웹 페이지에서) \textbf{Start an application} 버튼 클릭
    \inlineImage{06-education-apply.png}
    \item \textbf{학교 선택}:
    \begin{itemize}
        \item 신분(Student/Teacher) 선택 후 학교 검색 화면이 표시됨
        \item \textbf{브라우저 권장}: 오류 최소화를 위해 \textbf{Chrome 브라우저} 사용을 권장합니다.
        \item \textbf{위치 공유 오류 해결}:
        \begin{itemize}
            \item 브라우저의 위치 공유 요청을 반드시 \textbf{허용}해야 합니다.
            \item \textbf{[Error selecting location!]} 오류 발생 시: 브라우저 설정(특히 Edge의 '추적 방지' 기능)이 원인일 수 있습니다. 설정을 변경하거나 \textbf{스마트폰}으로 다시 시도하십시오.
        \end{itemize}
        \item \textbf{자동 검색되는 경우}: 목록에서 학교를 찾아 \textbf{Select this school} 클릭
        \item \textbf{검색되지 않는 경우}: 
        \begin{enumerate}
            \item 검색창에 \textbf{Dong-eui University} 직접 입력
            \item 여전히 없으면 \textbf{I don't see my school} 또는 수동 입력 옵션 선택
            \item ``How would you describe your school?''에서 \textbf{Higher-education: university college} 선택
            \item City란에 \textbf{Busan} 입력
            \item Country란에 \textbf{Korea, South} 선택
            \item \textbf{Submit application} 버튼 클릭
        \end{enumerate}
    \end{itemize}
    \inlineImage{07-school-select.png}
    \item \textbf{서류 제출}:
    \begin{itemize}
        \item 준비한 \textbf{영문 증명서}를 업로드하거나 카메라로 촬영합니다.
        \item 파일 업로드 버튼이 보이지 않는 경우, \textbf{Take a picture}를 선택하여 종이 증명서를 촬영하십시오.
        \item 학생의 경우 학생증을 함께 촬영하여 제출할 수도 있습니다.
    \end{itemize}
    \textbf{Submit application} 버튼을 클릭하여 완료합니다.
    \inlineImage{08-document-upload.png}
\end{enumerate}

\noindent\fbox{\parbox{\dimexpr\textwidth-2\fboxsep-2\fboxrule}{%
\textbf{참고: 2FA와 비밀번호의 차이}\\[0.5em]
\begin{itemize}[leftmargin=1.5em, topsep=0pt, itemsep=0pt]
    \item \textbf{비밀번호 분실}: 이메일로 비밀번호 재설정 가능
    \item \textbf{2FA 인증 수단 분실}: 복구 코드가 유일한 대안
\end{itemize}
\vspace{0.3em}
GitHub는 보안상 2FA를 우회하는 다른 방법을 제공하지 않으므로, 복구 코드 보관이 필수입니다.
}}

\subsection{증명서 스캔 시 주의사항}
\begin{itemize}
    \item \textbf{파일 형식}: \textbf{JPG 형식 필수} (PDF 업로드 불가)
    \item \textbf{스캔 모드}: \textbf{컬러 모드}로 스캔 (흑백/그레이스케일 금지)
    \item \textbf{자동 보정 기능 비활성화}: 
    \begin{itemize}
        \item 스캐너 소프트웨어에서 ``자동 문서 보정'', ``배경 제거'', ``화이트 밸런스'' 등의 기능을 \textbf{끄기}
    \end{itemize}
    \item \textbf{해상도}: 150-300 DPI 권장
    \item \textbf{파일 크기}: 10MB 이하
\end{itemize}

\textbf{스캔이 어려운 경우}: 스마트폰 카메라로 촬영 가능 (평평한 곳에 증명서를 놓고 그림자 없이 촬영)

\section{설정 및 활용 방법}
승인 완료 후 \href{https://github.com/copilot}{copilot.github.com}에서 AI 기능을 사용할 수 있습니다.

\subsection{초기 설정}
\begin{enumerate}
    \item GitHub 로그인 후 프로필 메뉴의 \textbf{Copilot Settings}로 이동합니다.
    \item \textbf{Features} 항목에서 비활성화(Disabled)된 모델들을 \textbf{Enabled}로 변경합니다.
    \item \textbf{Copilot can search the web}을 활성화하면 실시간 웹 검색 기반 답변이 가능해집니다.
\end{enumerate}

\subsection{대화 시작 및 파일 첨부}
\begin{itemize}
    \item \href{https://github.com/copilot}{copilot.github.com} 접속 후 입력창에 질문을 입력하고 \textbf{Enter}를 누르면 AI가 응답합니다.
    \item \textbf{파일 첨부}: 입력창 좌측 \textbf{+ 아이콘} $\rightarrow$ \textbf{Upload from computer}
    \item \textbf{지원 형식}: 이미지(JPG, PNG, GIF, WEBP), 텍스트(TXT, MD, CSV)
    \item \textbf{미지원}: PDF, Word, Excel, PPT, 동영상, 음성 \textrightarrow{} 텍스트 복사 또는 화면 캡처로 대체
    \item \textbf{Excel 팁}: CSV 형식으로 변환 후 첨부 (파일 $\rightarrow$ 다른 이름으로 저장 $\rightarrow$ CSV UTF-8)
    \item \textbf{모델 변경}: 입력창 우측 하단의 모델명 클릭
\end{itemize}

\section{AI 모델 선택 가이드}
\textit{※ 일반적 권장사항이며, 직접 사용 후 본인에게 적합한 모델을 찾으시길 권장합니다.}

\subsection{모델별 상세 사양}
\begin{xltabular}{\textwidth}{@{}l l X@{}}
\toprule
\textbf{등급} & \textbf{모델명} & \textbf{특징 및 주요 강점} \\
\midrule
\endhead

\textbf{일반} & GPT-5 mini & \textbf{빠른 속도 + 이미지 분석}. 일상적 문의, 문서 작성에 최적. \\
\textbf{일반} & GPT-4.1 & \textbf{안정적 범용 + 이미지 분석}. 다양한 업무에 균형 잡힌 품질. \\
\textbf{일반} & Claude Haiku 4.5 & \textbf{대량 작업 처리}. 많은 문서를 빠르게 처리해야 할 때 적합. \\
\midrule
\textbf{프리미엄} & Grok Code Fast 1 & \textbf{프로그래밍 전문}. 코드 작성, 오류 수정에 빠른 응답. \\
\textbf{프리미엄} & GPT-5 & \textbf{고급 문제 해결}. 복잡한 문제를 단계별로 분석. \\
\textbf{프리미엄} & GPT-5.1 & \textbf{장기 프로젝트}. 여러 단계의 복잡한 작업 수행에 특화. \\
\textbf{프리미엄} & Claude Sonnet 4 & \textbf{이미지 분석 + 심층 추론}. 코딩 작업 흐름에 최적화. \\
\textbf{프리미엄} & Claude Sonnet 4.5 & \textbf{한국어 문서 작성 최적}. 보고서, 기획서, 공문서에 탁월. \\
\textbf{프리미엄} & Claude Opus 4.1 & \textbf{학술 연구 및 심층 분석}. 논문 분석, 정책 검토에 적합. \\
\textbf{프리미엄} & Claude Opus 4.5 & \textbf{최신 최고 성능}. 극도로 복잡한 분석이나 창의적 업무에 적합. \\
\textbf{프리미엄} & Gemini 2.5 Pro & \textbf{대용량 문서 분석}. 매우 긴 문서 처리 + 이미지 분석 가능. \\
\textbf{프리미엄} & Gemini 3 Pro & \textbf{최신 멀티모달}. 뛰어난 추론 + 이미지 분석 (영상/음성 미지원). \\
\bottomrule
\end{xltabular}

\subsection{인증 거절 시 해결 방법}
\label{sec:인증거절해결}

신청이 거절된 경우 다음 사항을 확인하고 재신청하십시오.

\subsubsection{주요 거절 사유 및 해결 방법}

\begin{table}[H]
    \centering
    \caption{인증 거절 사유 및 해결 방법}
    \label{tab:거절사유해결}
    \renewcommand{\arraystretch}{1.3}
    \begin{tabularx}{\textwidth}{@{}l X@{}}
        \toprule
        \textbf{거절 사유} & \textbf{해결 방법} \\
        \midrule
        학교 이름 미표시 & 제출 문서에 학교의 \textbf{정식 명칭}이 표시되어야 합니다 (로고만으로는 불충분). 한국어 문서의 경우 원본과 영문 번역본을 함께 촬영하여 제출하십시오. \\
        \addlinespace
        이름 불일치 & GitHub 결제 정보의 이름과 학교 문서의 이름이 \textbf{정확히 일치}해야 합니다. 영문 표기 순서를 확인하십시오 (예: Jinseok Seo 또는 Seo Jinseok). \\
        \addlinespace
        프로필 정보 미완성 & GitHub 프로필의 Name을 학교 문서와 동일하게 입력한 후 \textbf{로그아웃 $\rightarrow$ 재로그인}하여 재신청하십시오. \\
        \bottomrule
    \end{tabularx}
\end{table}

\subsubsection{재신청 전 확인 사항}

\begin{itemize}
    \item GitHub 프로필 이름 = 결제 정보 이름 = 학교 문서 이름이 \textbf{정확히 일치}하는지 확인
    \item 제출 문서에 학교 \textbf{정식 명칭}이 표시되어 있는지 확인 (로고만 불가)
    \item 한국어 문서의 경우 \textbf{영문 번역본을 함께 촬영}하여 제출
    \item 프로필 수정 후 \textbf{로그아웃 $\rightarrow$ 재로그인} 완료 여부 확인
    \item 외자 이름인 경우 First Name과 Last Name 모두에 동일하게 입력
\end{itemize}

\begin{tcolorbox}[
    colback=blue!5!white,
    colframe=blue!75!black,
    title=권장 제출 서류,
    fonttitle=\bfseries
]
\begin{itemize}[leftmargin=1.5em, topsep=0pt, itemsep=3pt]
    \item \textbf{영문 재학증명서}: 학교 정식 명칭, 본인 영문 이름, 재학 기간이 명확히 표시된 문서
    \item \textbf{학생증 + 학교 공식 문서}: 학교명과 이름이 명확히 보이도록 함께 촬영
    \item \textbf{학교 이메일 인증}: @university.ac.kr 형태의 학교 이메일 사용 시 승인 확률이 높아집니다
\end{itemize}
\end{tcolorbox}

\subsection{업무별 추천}
\begin{itemize}
    \item \textbf{일상 간단한 질문}: GPT-5 mini, GPT-4.1
    \item \textbf{한국어 문서}: Claude Sonnet 4.5 \quad \textbf{연구/심층 분석}: Claude Opus 4.1/4.5
    \item \textbf{대량 처리}: Claude Haiku 4.5 \quad \textbf{초대용량}: Gemini 2.5/3 Pro
\end{itemize}

\vspace{1em}
\hrule
\vspace{0.5em}

\section*{유의 사항}
\small
본 문서는 작성 시점(\docdate) 기준이며, GitHub Copilot의 정책 및 기능은 변경될 수 있습니다. 최신 정보는 공식 페이지(\href{https://github.com/features/copilot}{github.com/features/copilot}) 및 공식 문서(\href{https://docs.github.com/en/copilot}{docs.github.com/copilot})를 참조하십시오.
\end{document}