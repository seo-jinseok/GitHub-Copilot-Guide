\documentclass[11pt, a4paper]{article}
% =============================================================================
% GitHub Copilot 가이드 공통 프리앰블
% _preamble.tex - 모든 LaTeX 문서에서 % =============================================================================
% GitHub Copilot 가이드 공통 프리앰블
% _preamble.tex - 모든 LaTeX 문서에서 % =============================================================================
% GitHub Copilot 가이드 공통 프리앰블
% _preamble.tex - 모든 LaTeX 문서에서 \input{_preamble} 으로 재사용
% =============================================================================

% --- DOCUMENT VERSION CONTROL ---
% PDF 빌드 시점의 날짜와 시간(KST, UTC+9)이 자동으로 반영됩니다
% 참고: Overleaf 서버는 CET(UTC+1) 시간대를 사용하므로 +8로 보정합니다
\usepackage{datetime2}
\ExplSyntaxOn
\int_new:N \g_kst_hour_int
\int_new:N \g_kst_day_int
\int_new:N \g_kst_month_int
\int_new:N \g_kst_year_int
\int_gset:Nn \g_kst_hour_int { \the\time / 60 + 9 }
\int_gset:Nn \g_kst_day_int { \the\day }
\int_gset:Nn \g_kst_month_int { \the\month }
\int_gset:Nn \g_kst_year_int { \the\year }
\int_compare:nNnT { \g_kst_hour_int } > { 23 }
  {
    \int_gadd:Nn \g_kst_hour_int { -24 }
    \int_gincr:N \g_kst_day_int
  }
\newcommand{\docversion}{최종~수정:~\int_use:N \g_kst_year_int.~\int_compare:nNnTF { \g_kst_month_int } < { 10 } {0}{} \int_use:N \g_kst_month_int.~\int_compare:nNnTF { \g_kst_day_int } < { 10 } {0}{} \int_use:N \g_kst_day_int.
  \c_space_tl \int_compare:nNnTF { \g_kst_hour_int } < { 10 } {0}{} \int_use:N \g_kst_hour_int
  : \int_compare:nNnTF { \int_mod:nn {\the\time} {60} } < { 10 } {0}{} \int_eval:n { \int_mod:nn {\the\time} {60} }}
\newcommand{\docdate}{\int_use:N \g_kst_year_int 년~\int_compare:nNnTF { \g_kst_month_int } < { 10 } {0}{} \int_use:N \g_kst_month_int 월~\int_compare:nNnTF { \g_kst_day_int } < { 10 } {0}{} \int_use:N \g_kst_day_int 일}
\ExplSyntaxOff

% --- 저자 정보 (각 문서에서 재정의 가능) ---
\providecommand{\authors}{동의대학교 서진석(jsseo@deu.ac.kr)}

% --- UNIVERSAL PREAMBLE BLOCK ---
\usepackage[a4paper, top=3cm, bottom=3cm, left=2.5cm, right=2.5cm, headheight=14pt]{geometry}
\usepackage{fontspec}

% Language setup for Korean
\usepackage[korean, provide=*]{babel}

% Fonts: Using Noto Serif for body (formal) and Noto Sans for headers
\babelfont{rm}{Noto Serif CJK KR}
\babelfont{sf}{Noto Sans CJK KR}

% CJK 폰트는 이탤릭이 없으므로 폰트 대체 규칙 설정 (경고 방지)
\DeclareFontShape{TU}{NotoSerifCJKKR(0)}{m}{it}{<->ssub*NotoSerifCJKKR(0)/m/n}{}
\DeclareFontShape{TU}{NotoSerifCJKKR(0)}{m}{sl}{<->ssub*NotoSerifCJKKR(0)/m/n}{}
\DeclareFontShape{TU}{NotoSerifCJKKR(0)}{bx}{it}{<->ssub*NotoSerifCJKKR(0)/bx/n}{}
\DeclareFontShape{TU}{NotoSansCJKKR(0)}{m}{it}{<->ssub*NotoSansCJKKR(0)/m/n}{}
\DeclareFontShape{TU}{NotoSansCJKKR(0)}{m}{sl}{<->ssub*NotoSansCJKKR(0)/m/n}{}
\DeclareFontShape{TU}{NotoSansCJKKR(0)}{bx}{it}{<->ssub*NotoSansCJKKR(0)/bx/n}{}

% Essential Packages
\usepackage{enumitem}
\usepackage{booktabs}   % Professional tables
\usepackage{tabularx}   % Auto-width tables
\usepackage{titlesec}   % Section formatting
\usepackage{xcolor}     % Colors for visual hierarchy
\usepackage{fancyhdr}   % Headers and footers
\usepackage{setspace}   % Line spacing
\usepackage{longtable}  % For tables that might span pages
\usepackage{xltabular}  % For flexible width long tables
\usepackage{float}      % For [H] option to fix table position
\usepackage{kotex}      % Additional Korean support insurance
\usepackage[hidelinks]{hyperref}

% --- STYLING COMMANDS ---

% Define a formal navy blue for headers
\definecolor{eduNavy}{RGB}{0, 43, 91}

% Section styling
\titleformat{\section}
  {\Large\sffamily\bfseries\color{eduNavy}}
  {\thesection}{1em}{}
  [\vspace{-0.5em}\hrule height 1pt]

\titleformat{\subsection}
  {\large\sffamily\bfseries\color{eduNavy}}
  {\thesubsection}{1em}{}

\titleformat{\subsubsection}
  {\normalsize\sffamily\bfseries\color{eduNavy}}
  {\thesubsubsection}{1em}{}

% List settings
\setlist[itemize]{label=\textbullet, leftmargin=1.5em, itemsep=0.2em}
\setlist[enumerate]{label=\arabic*., leftmargin=1.5em, itemsep=0.2em}

% Line spacing for readability
\setstretch{1.2}

% Header and Footer
\pagestyle{fancy}
\fancyhf{}
\fancyhead[L]{\small\sffamily \authors}
\fancyhead[R]{\small\sffamily \docversion}
\fancyfoot[C]{\thepage}
\renewcommand{\headrulewidth}{0.5pt}

% =============================================================================
% 사용법:
% 각 문서에서 다음과 같이 사용합니다:
%
% \documentclass[11pt, a4paper]{article}
% \input{_preamble}
%
% % 필요시 저자 재정의
% \renewcommand{\authors}{작성자명 (email@example.com)}
%
% % 제목 설정
% \title{...}
% \author{\authors}
% \date{\docversion}
%
% \begin{document}
% \maketitle
% ...
% \end{document}
% =============================================================================
 으로 재사용
% =============================================================================

% --- DOCUMENT VERSION CONTROL ---
% PDF 빌드 시점의 날짜와 시간(KST, UTC+9)이 자동으로 반영됩니다
% 참고: Overleaf 서버는 CET(UTC+1) 시간대를 사용하므로 +8로 보정합니다
\usepackage{datetime2}
\ExplSyntaxOn
\int_new:N \g_kst_hour_int
\int_new:N \g_kst_day_int
\int_new:N \g_kst_month_int
\int_new:N \g_kst_year_int
\int_gset:Nn \g_kst_hour_int { \the\time / 60 + 9 }
\int_gset:Nn \g_kst_day_int { \the\day }
\int_gset:Nn \g_kst_month_int { \the\month }
\int_gset:Nn \g_kst_year_int { \the\year }
\int_compare:nNnT { \g_kst_hour_int } > { 23 }
  {
    \int_gadd:Nn \g_kst_hour_int { -24 }
    \int_gincr:N \g_kst_day_int
  }
\newcommand{\docversion}{최종~수정:~\int_use:N \g_kst_year_int.~\int_compare:nNnTF { \g_kst_month_int } < { 10 } {0}{} \int_use:N \g_kst_month_int.~\int_compare:nNnTF { \g_kst_day_int } < { 10 } {0}{} \int_use:N \g_kst_day_int.
  \c_space_tl \int_compare:nNnTF { \g_kst_hour_int } < { 10 } {0}{} \int_use:N \g_kst_hour_int
  : \int_compare:nNnTF { \int_mod:nn {\the\time} {60} } < { 10 } {0}{} \int_eval:n { \int_mod:nn {\the\time} {60} }}
\newcommand{\docdate}{\int_use:N \g_kst_year_int 년~\int_compare:nNnTF { \g_kst_month_int } < { 10 } {0}{} \int_use:N \g_kst_month_int 월~\int_compare:nNnTF { \g_kst_day_int } < { 10 } {0}{} \int_use:N \g_kst_day_int 일}
\ExplSyntaxOff

% --- 저자 정보 (각 문서에서 재정의 가능) ---
\providecommand{\authors}{동의대학교 서진석(jsseo@deu.ac.kr)}

% --- UNIVERSAL PREAMBLE BLOCK ---
\usepackage[a4paper, top=3cm, bottom=3cm, left=2.5cm, right=2.5cm, headheight=14pt]{geometry}
\usepackage{fontspec}

% Language setup for Korean
\usepackage[korean, provide=*]{babel}

% Fonts: Using Noto Serif for body (formal) and Noto Sans for headers
\babelfont{rm}{Noto Serif CJK KR}
\babelfont{sf}{Noto Sans CJK KR}

% CJK 폰트는 이탤릭이 없으므로 폰트 대체 규칙 설정 (경고 방지)
\DeclareFontShape{TU}{NotoSerifCJKKR(0)}{m}{it}{<->ssub*NotoSerifCJKKR(0)/m/n}{}
\DeclareFontShape{TU}{NotoSerifCJKKR(0)}{m}{sl}{<->ssub*NotoSerifCJKKR(0)/m/n}{}
\DeclareFontShape{TU}{NotoSerifCJKKR(0)}{bx}{it}{<->ssub*NotoSerifCJKKR(0)/bx/n}{}
\DeclareFontShape{TU}{NotoSansCJKKR(0)}{m}{it}{<->ssub*NotoSansCJKKR(0)/m/n}{}
\DeclareFontShape{TU}{NotoSansCJKKR(0)}{m}{sl}{<->ssub*NotoSansCJKKR(0)/m/n}{}
\DeclareFontShape{TU}{NotoSansCJKKR(0)}{bx}{it}{<->ssub*NotoSansCJKKR(0)/bx/n}{}

% Essential Packages
\usepackage{enumitem}
\usepackage{booktabs}   % Professional tables
\usepackage{tabularx}   % Auto-width tables
\usepackage{titlesec}   % Section formatting
\usepackage{xcolor}     % Colors for visual hierarchy
\usepackage{fancyhdr}   % Headers and footers
\usepackage{setspace}   % Line spacing
\usepackage{longtable}  % For tables that might span pages
\usepackage{xltabular}  % For flexible width long tables
\usepackage{float}      % For [H] option to fix table position
\usepackage{kotex}      % Additional Korean support insurance
\usepackage[hidelinks]{hyperref}

% --- STYLING COMMANDS ---

% Define a formal navy blue for headers
\definecolor{eduNavy}{RGB}{0, 43, 91}

% Section styling
\titleformat{\section}
  {\Large\sffamily\bfseries\color{eduNavy}}
  {\thesection}{1em}{}
  [\vspace{-0.5em}\hrule height 1pt]

\titleformat{\subsection}
  {\large\sffamily\bfseries\color{eduNavy}}
  {\thesubsection}{1em}{}

\titleformat{\subsubsection}
  {\normalsize\sffamily\bfseries\color{eduNavy}}
  {\thesubsubsection}{1em}{}

% List settings
\setlist[itemize]{label=\textbullet, leftmargin=1.5em, itemsep=0.2em}
\setlist[enumerate]{label=\arabic*., leftmargin=1.5em, itemsep=0.2em}

% Line spacing for readability
\setstretch{1.2}

% Header and Footer
\pagestyle{fancy}
\fancyhf{}
\fancyhead[L]{\small\sffamily \authors}
\fancyhead[R]{\small\sffamily \docversion}
\fancyfoot[C]{\thepage}
\renewcommand{\headrulewidth}{0.5pt}

% =============================================================================
% 사용법:
% 각 문서에서 다음과 같이 사용합니다:
%
% \documentclass[11pt, a4paper]{article}
% % =============================================================================
% GitHub Copilot 가이드 공통 프리앰블
% _preamble.tex - 모든 LaTeX 문서에서 \input{_preamble} 으로 재사용
% =============================================================================

% --- DOCUMENT VERSION CONTROL ---
% PDF 빌드 시점의 날짜와 시간(KST, UTC+9)이 자동으로 반영됩니다
% 참고: Overleaf 서버는 CET(UTC+1) 시간대를 사용하므로 +8로 보정합니다
\usepackage{datetime2}
\ExplSyntaxOn
\int_new:N \g_kst_hour_int
\int_new:N \g_kst_day_int
\int_new:N \g_kst_month_int
\int_new:N \g_kst_year_int
\int_gset:Nn \g_kst_hour_int { \the\time / 60 + 9 }
\int_gset:Nn \g_kst_day_int { \the\day }
\int_gset:Nn \g_kst_month_int { \the\month }
\int_gset:Nn \g_kst_year_int { \the\year }
\int_compare:nNnT { \g_kst_hour_int } > { 23 }
  {
    \int_gadd:Nn \g_kst_hour_int { -24 }
    \int_gincr:N \g_kst_day_int
  }
\newcommand{\docversion}{최종~수정:~\int_use:N \g_kst_year_int.~\int_compare:nNnTF { \g_kst_month_int } < { 10 } {0}{} \int_use:N \g_kst_month_int.~\int_compare:nNnTF { \g_kst_day_int } < { 10 } {0}{} \int_use:N \g_kst_day_int.
  \c_space_tl \int_compare:nNnTF { \g_kst_hour_int } < { 10 } {0}{} \int_use:N \g_kst_hour_int
  : \int_compare:nNnTF { \int_mod:nn {\the\time} {60} } < { 10 } {0}{} \int_eval:n { \int_mod:nn {\the\time} {60} }}
\newcommand{\docdate}{\int_use:N \g_kst_year_int 년~\int_compare:nNnTF { \g_kst_month_int } < { 10 } {0}{} \int_use:N \g_kst_month_int 월~\int_compare:nNnTF { \g_kst_day_int } < { 10 } {0}{} \int_use:N \g_kst_day_int 일}
\ExplSyntaxOff

% --- 저자 정보 (각 문서에서 재정의 가능) ---
\providecommand{\authors}{동의대학교 서진석(jsseo@deu.ac.kr)}

% --- UNIVERSAL PREAMBLE BLOCK ---
\usepackage[a4paper, top=3cm, bottom=3cm, left=2.5cm, right=2.5cm, headheight=14pt]{geometry}
\usepackage{fontspec}

% Language setup for Korean
\usepackage[korean, provide=*]{babel}

% Fonts: Using Noto Serif for body (formal) and Noto Sans for headers
\babelfont{rm}{Noto Serif CJK KR}
\babelfont{sf}{Noto Sans CJK KR}

% CJK 폰트는 이탤릭이 없으므로 폰트 대체 규칙 설정 (경고 방지)
\DeclareFontShape{TU}{NotoSerifCJKKR(0)}{m}{it}{<->ssub*NotoSerifCJKKR(0)/m/n}{}
\DeclareFontShape{TU}{NotoSerifCJKKR(0)}{m}{sl}{<->ssub*NotoSerifCJKKR(0)/m/n}{}
\DeclareFontShape{TU}{NotoSerifCJKKR(0)}{bx}{it}{<->ssub*NotoSerifCJKKR(0)/bx/n}{}
\DeclareFontShape{TU}{NotoSansCJKKR(0)}{m}{it}{<->ssub*NotoSansCJKKR(0)/m/n}{}
\DeclareFontShape{TU}{NotoSansCJKKR(0)}{m}{sl}{<->ssub*NotoSansCJKKR(0)/m/n}{}
\DeclareFontShape{TU}{NotoSansCJKKR(0)}{bx}{it}{<->ssub*NotoSansCJKKR(0)/bx/n}{}

% Essential Packages
\usepackage{enumitem}
\usepackage{booktabs}   % Professional tables
\usepackage{tabularx}   % Auto-width tables
\usepackage{titlesec}   % Section formatting
\usepackage{xcolor}     % Colors for visual hierarchy
\usepackage{fancyhdr}   % Headers and footers
\usepackage{setspace}   % Line spacing
\usepackage{longtable}  % For tables that might span pages
\usepackage{xltabular}  % For flexible width long tables
\usepackage{float}      % For [H] option to fix table position
\usepackage{kotex}      % Additional Korean support insurance
\usepackage[hidelinks]{hyperref}

% --- STYLING COMMANDS ---

% Define a formal navy blue for headers
\definecolor{eduNavy}{RGB}{0, 43, 91}

% Section styling
\titleformat{\section}
  {\Large\sffamily\bfseries\color{eduNavy}}
  {\thesection}{1em}{}
  [\vspace{-0.5em}\hrule height 1pt]

\titleformat{\subsection}
  {\large\sffamily\bfseries\color{eduNavy}}
  {\thesubsection}{1em}{}

\titleformat{\subsubsection}
  {\normalsize\sffamily\bfseries\color{eduNavy}}
  {\thesubsubsection}{1em}{}

% List settings
\setlist[itemize]{label=\textbullet, leftmargin=1.5em, itemsep=0.2em}
\setlist[enumerate]{label=\arabic*., leftmargin=1.5em, itemsep=0.2em}

% Line spacing for readability
\setstretch{1.2}

% Header and Footer
\pagestyle{fancy}
\fancyhf{}
\fancyhead[L]{\small\sffamily \authors}
\fancyhead[R]{\small\sffamily \docversion}
\fancyfoot[C]{\thepage}
\renewcommand{\headrulewidth}{0.5pt}

% =============================================================================
% 사용법:
% 각 문서에서 다음과 같이 사용합니다:
%
% \documentclass[11pt, a4paper]{article}
% \input{_preamble}
%
% % 필요시 저자 재정의
% \renewcommand{\authors}{작성자명 (email@example.com)}
%
% % 제목 설정
% \title{...}
% \author{\authors}
% \date{\docversion}
%
% \begin{document}
% \maketitle
% ...
% \end{document}
% =============================================================================

%
% % 필요시 저자 재정의
% \renewcommand{\authors}{작성자명 (email@example.com)}
%
% % 제목 설정
% \title{...}
% \author{\authors}
% \date{\docversion}
%
% \begin{document}
% \maketitle
% ...
% \end{document}
% =============================================================================
 으로 재사용
% =============================================================================

% --- DOCUMENT VERSION CONTROL ---
% PDF 빌드 시점의 날짜와 시간(KST, UTC+9)이 자동으로 반영됩니다
% 참고: Overleaf 서버는 CET(UTC+1) 시간대를 사용하므로 +8로 보정합니다
\usepackage{datetime2}
\ExplSyntaxOn
\int_new:N \g_kst_hour_int
\int_new:N \g_kst_day_int
\int_new:N \g_kst_month_int
\int_new:N \g_kst_year_int
\int_gset:Nn \g_kst_hour_int { \the\time / 60 + 9 }
\int_gset:Nn \g_kst_day_int { \the\day }
\int_gset:Nn \g_kst_month_int { \the\month }
\int_gset:Nn \g_kst_year_int { \the\year }
\int_compare:nNnT { \g_kst_hour_int } > { 23 }
  {
    \int_gadd:Nn \g_kst_hour_int { -24 }
    \int_gincr:N \g_kst_day_int
  }
\newcommand{\docversion}{최종~수정:~\int_use:N \g_kst_year_int.~\int_compare:nNnTF { \g_kst_month_int } < { 10 } {0}{} \int_use:N \g_kst_month_int.~\int_compare:nNnTF { \g_kst_day_int } < { 10 } {0}{} \int_use:N \g_kst_day_int.
  \c_space_tl \int_compare:nNnTF { \g_kst_hour_int } < { 10 } {0}{} \int_use:N \g_kst_hour_int
  : \int_compare:nNnTF { \int_mod:nn {\the\time} {60} } < { 10 } {0}{} \int_eval:n { \int_mod:nn {\the\time} {60} }}
\newcommand{\docdate}{\int_use:N \g_kst_year_int 년~\int_compare:nNnTF { \g_kst_month_int } < { 10 } {0}{} \int_use:N \g_kst_month_int 월~\int_compare:nNnTF { \g_kst_day_int } < { 10 } {0}{} \int_use:N \g_kst_day_int 일}
\ExplSyntaxOff

% --- 저자 정보 (각 문서에서 재정의 가능) ---
\providecommand{\authors}{동의대학교 서진석(jsseo@deu.ac.kr)}

% --- UNIVERSAL PREAMBLE BLOCK ---
\usepackage[a4paper, top=3cm, bottom=3cm, left=2.5cm, right=2.5cm, headheight=14pt]{geometry}
\usepackage{fontspec}

% Language setup for Korean
\usepackage[korean, provide=*]{babel}

% Fonts: Using Noto Serif for body (formal) and Noto Sans for headers
\babelfont{rm}{Noto Serif CJK KR}
\babelfont{sf}{Noto Sans CJK KR}

% CJK 폰트는 이탤릭이 없으므로 폰트 대체 규칙 설정 (경고 방지)
\DeclareFontShape{TU}{NotoSerifCJKKR(0)}{m}{it}{<->ssub*NotoSerifCJKKR(0)/m/n}{}
\DeclareFontShape{TU}{NotoSerifCJKKR(0)}{m}{sl}{<->ssub*NotoSerifCJKKR(0)/m/n}{}
\DeclareFontShape{TU}{NotoSerifCJKKR(0)}{bx}{it}{<->ssub*NotoSerifCJKKR(0)/bx/n}{}
\DeclareFontShape{TU}{NotoSansCJKKR(0)}{m}{it}{<->ssub*NotoSansCJKKR(0)/m/n}{}
\DeclareFontShape{TU}{NotoSansCJKKR(0)}{m}{sl}{<->ssub*NotoSansCJKKR(0)/m/n}{}
\DeclareFontShape{TU}{NotoSansCJKKR(0)}{bx}{it}{<->ssub*NotoSansCJKKR(0)/bx/n}{}

% Essential Packages
\usepackage{enumitem}
\usepackage{booktabs}   % Professional tables
\usepackage{tabularx}   % Auto-width tables
\usepackage{titlesec}   % Section formatting
\usepackage{xcolor}     % Colors for visual hierarchy
\usepackage{fancyhdr}   % Headers and footers
\usepackage{setspace}   % Line spacing
\usepackage{longtable}  % For tables that might span pages
\usepackage{xltabular}  % For flexible width long tables
\usepackage{float}      % For [H] option to fix table position
\usepackage{kotex}      % Additional Korean support insurance
\usepackage[hidelinks]{hyperref}

% --- STYLING COMMANDS ---

% Define a formal navy blue for headers
\definecolor{eduNavy}{RGB}{0, 43, 91}

% Section styling
\titleformat{\section}
  {\Large\sffamily\bfseries\color{eduNavy}}
  {\thesection}{1em}{}
  [\vspace{-0.5em}\hrule height 1pt]

\titleformat{\subsection}
  {\large\sffamily\bfseries\color{eduNavy}}
  {\thesubsection}{1em}{}

\titleformat{\subsubsection}
  {\normalsize\sffamily\bfseries\color{eduNavy}}
  {\thesubsubsection}{1em}{}

% List settings
\setlist[itemize]{label=\textbullet, leftmargin=1.5em, itemsep=0.2em}
\setlist[enumerate]{label=\arabic*., leftmargin=1.5em, itemsep=0.2em}

% Line spacing for readability
\setstretch{1.2}

% Header and Footer
\pagestyle{fancy}
\fancyhf{}
\fancyhead[L]{\small\sffamily \authors}
\fancyhead[R]{\small\sffamily \docversion}
\fancyfoot[C]{\thepage}
\renewcommand{\headrulewidth}{0.5pt}

% =============================================================================
% 사용법:
% 각 문서에서 다음과 같이 사용합니다:
%
% \documentclass[11pt, a4paper]{article}
% % =============================================================================
% GitHub Copilot 가이드 공통 프리앰블
% _preamble.tex - 모든 LaTeX 문서에서 % =============================================================================
% GitHub Copilot 가이드 공통 프리앰블
% _preamble.tex - 모든 LaTeX 문서에서 \input{_preamble} 으로 재사용
% =============================================================================

% --- DOCUMENT VERSION CONTROL ---
% PDF 빌드 시점의 날짜와 시간(KST, UTC+9)이 자동으로 반영됩니다
% 참고: Overleaf 서버는 CET(UTC+1) 시간대를 사용하므로 +8로 보정합니다
\usepackage{datetime2}
\ExplSyntaxOn
\int_new:N \g_kst_hour_int
\int_new:N \g_kst_day_int
\int_new:N \g_kst_month_int
\int_new:N \g_kst_year_int
\int_gset:Nn \g_kst_hour_int { \the\time / 60 + 9 }
\int_gset:Nn \g_kst_day_int { \the\day }
\int_gset:Nn \g_kst_month_int { \the\month }
\int_gset:Nn \g_kst_year_int { \the\year }
\int_compare:nNnT { \g_kst_hour_int } > { 23 }
  {
    \int_gadd:Nn \g_kst_hour_int { -24 }
    \int_gincr:N \g_kst_day_int
  }
\newcommand{\docversion}{최종~수정:~\int_use:N \g_kst_year_int.~\int_compare:nNnTF { \g_kst_month_int } < { 10 } {0}{} \int_use:N \g_kst_month_int.~\int_compare:nNnTF { \g_kst_day_int } < { 10 } {0}{} \int_use:N \g_kst_day_int.
  \c_space_tl \int_compare:nNnTF { \g_kst_hour_int } < { 10 } {0}{} \int_use:N \g_kst_hour_int
  : \int_compare:nNnTF { \int_mod:nn {\the\time} {60} } < { 10 } {0}{} \int_eval:n { \int_mod:nn {\the\time} {60} }}
\newcommand{\docdate}{\int_use:N \g_kst_year_int 년~\int_compare:nNnTF { \g_kst_month_int } < { 10 } {0}{} \int_use:N \g_kst_month_int 월~\int_compare:nNnTF { \g_kst_day_int } < { 10 } {0}{} \int_use:N \g_kst_day_int 일}
\ExplSyntaxOff

% --- 저자 정보 (각 문서에서 재정의 가능) ---
\providecommand{\authors}{동의대학교 서진석(jsseo@deu.ac.kr)}

% --- UNIVERSAL PREAMBLE BLOCK ---
\usepackage[a4paper, top=3cm, bottom=3cm, left=2.5cm, right=2.5cm, headheight=14pt]{geometry}
\usepackage{fontspec}

% Language setup for Korean
\usepackage[korean, provide=*]{babel}

% Fonts: Using Noto Serif for body (formal) and Noto Sans for headers
\babelfont{rm}{Noto Serif CJK KR}
\babelfont{sf}{Noto Sans CJK KR}

% CJK 폰트는 이탤릭이 없으므로 폰트 대체 규칙 설정 (경고 방지)
\DeclareFontShape{TU}{NotoSerifCJKKR(0)}{m}{it}{<->ssub*NotoSerifCJKKR(0)/m/n}{}
\DeclareFontShape{TU}{NotoSerifCJKKR(0)}{m}{sl}{<->ssub*NotoSerifCJKKR(0)/m/n}{}
\DeclareFontShape{TU}{NotoSerifCJKKR(0)}{bx}{it}{<->ssub*NotoSerifCJKKR(0)/bx/n}{}
\DeclareFontShape{TU}{NotoSansCJKKR(0)}{m}{it}{<->ssub*NotoSansCJKKR(0)/m/n}{}
\DeclareFontShape{TU}{NotoSansCJKKR(0)}{m}{sl}{<->ssub*NotoSansCJKKR(0)/m/n}{}
\DeclareFontShape{TU}{NotoSansCJKKR(0)}{bx}{it}{<->ssub*NotoSansCJKKR(0)/bx/n}{}

% Essential Packages
\usepackage{enumitem}
\usepackage{booktabs}   % Professional tables
\usepackage{tabularx}   % Auto-width tables
\usepackage{titlesec}   % Section formatting
\usepackage{xcolor}     % Colors for visual hierarchy
\usepackage{fancyhdr}   % Headers and footers
\usepackage{setspace}   % Line spacing
\usepackage{longtable}  % For tables that might span pages
\usepackage{xltabular}  % For flexible width long tables
\usepackage{float}      % For [H] option to fix table position
\usepackage{kotex}      % Additional Korean support insurance
\usepackage[hidelinks]{hyperref}

% --- STYLING COMMANDS ---

% Define a formal navy blue for headers
\definecolor{eduNavy}{RGB}{0, 43, 91}

% Section styling
\titleformat{\section}
  {\Large\sffamily\bfseries\color{eduNavy}}
  {\thesection}{1em}{}
  [\vspace{-0.5em}\hrule height 1pt]

\titleformat{\subsection}
  {\large\sffamily\bfseries\color{eduNavy}}
  {\thesubsection}{1em}{}

\titleformat{\subsubsection}
  {\normalsize\sffamily\bfseries\color{eduNavy}}
  {\thesubsubsection}{1em}{}

% List settings
\setlist[itemize]{label=\textbullet, leftmargin=1.5em, itemsep=0.2em}
\setlist[enumerate]{label=\arabic*., leftmargin=1.5em, itemsep=0.2em}

% Line spacing for readability
\setstretch{1.2}

% Header and Footer
\pagestyle{fancy}
\fancyhf{}
\fancyhead[L]{\small\sffamily \authors}
\fancyhead[R]{\small\sffamily \docversion}
\fancyfoot[C]{\thepage}
\renewcommand{\headrulewidth}{0.5pt}

% =============================================================================
% 사용법:
% 각 문서에서 다음과 같이 사용합니다:
%
% \documentclass[11pt, a4paper]{article}
% \input{_preamble}
%
% % 필요시 저자 재정의
% \renewcommand{\authors}{작성자명 (email@example.com)}
%
% % 제목 설정
% \title{...}
% \author{\authors}
% \date{\docversion}
%
% \begin{document}
% \maketitle
% ...
% \end{document}
% =============================================================================
 으로 재사용
% =============================================================================

% --- DOCUMENT VERSION CONTROL ---
% PDF 빌드 시점의 날짜와 시간(KST, UTC+9)이 자동으로 반영됩니다
% 참고: Overleaf 서버는 CET(UTC+1) 시간대를 사용하므로 +8로 보정합니다
\usepackage{datetime2}
\ExplSyntaxOn
\int_new:N \g_kst_hour_int
\int_new:N \g_kst_day_int
\int_new:N \g_kst_month_int
\int_new:N \g_kst_year_int
\int_gset:Nn \g_kst_hour_int { \the\time / 60 + 9 }
\int_gset:Nn \g_kst_day_int { \the\day }
\int_gset:Nn \g_kst_month_int { \the\month }
\int_gset:Nn \g_kst_year_int { \the\year }
\int_compare:nNnT { \g_kst_hour_int } > { 23 }
  {
    \int_gadd:Nn \g_kst_hour_int { -24 }
    \int_gincr:N \g_kst_day_int
  }
\newcommand{\docversion}{최종~수정:~\int_use:N \g_kst_year_int.~\int_compare:nNnTF { \g_kst_month_int } < { 10 } {0}{} \int_use:N \g_kst_month_int.~\int_compare:nNnTF { \g_kst_day_int } < { 10 } {0}{} \int_use:N \g_kst_day_int.
  \c_space_tl \int_compare:nNnTF { \g_kst_hour_int } < { 10 } {0}{} \int_use:N \g_kst_hour_int
  : \int_compare:nNnTF { \int_mod:nn {\the\time} {60} } < { 10 } {0}{} \int_eval:n { \int_mod:nn {\the\time} {60} }}
\newcommand{\docdate}{\int_use:N \g_kst_year_int 년~\int_compare:nNnTF { \g_kst_month_int } < { 10 } {0}{} \int_use:N \g_kst_month_int 월~\int_compare:nNnTF { \g_kst_day_int } < { 10 } {0}{} \int_use:N \g_kst_day_int 일}
\ExplSyntaxOff

% --- 저자 정보 (각 문서에서 재정의 가능) ---
\providecommand{\authors}{동의대학교 서진석(jsseo@deu.ac.kr)}

% --- UNIVERSAL PREAMBLE BLOCK ---
\usepackage[a4paper, top=3cm, bottom=3cm, left=2.5cm, right=2.5cm, headheight=14pt]{geometry}
\usepackage{fontspec}

% Language setup for Korean
\usepackage[korean, provide=*]{babel}

% Fonts: Using Noto Serif for body (formal) and Noto Sans for headers
\babelfont{rm}{Noto Serif CJK KR}
\babelfont{sf}{Noto Sans CJK KR}

% CJK 폰트는 이탤릭이 없으므로 폰트 대체 규칙 설정 (경고 방지)
\DeclareFontShape{TU}{NotoSerifCJKKR(0)}{m}{it}{<->ssub*NotoSerifCJKKR(0)/m/n}{}
\DeclareFontShape{TU}{NotoSerifCJKKR(0)}{m}{sl}{<->ssub*NotoSerifCJKKR(0)/m/n}{}
\DeclareFontShape{TU}{NotoSerifCJKKR(0)}{bx}{it}{<->ssub*NotoSerifCJKKR(0)/bx/n}{}
\DeclareFontShape{TU}{NotoSansCJKKR(0)}{m}{it}{<->ssub*NotoSansCJKKR(0)/m/n}{}
\DeclareFontShape{TU}{NotoSansCJKKR(0)}{m}{sl}{<->ssub*NotoSansCJKKR(0)/m/n}{}
\DeclareFontShape{TU}{NotoSansCJKKR(0)}{bx}{it}{<->ssub*NotoSansCJKKR(0)/bx/n}{}

% Essential Packages
\usepackage{enumitem}
\usepackage{booktabs}   % Professional tables
\usepackage{tabularx}   % Auto-width tables
\usepackage{titlesec}   % Section formatting
\usepackage{xcolor}     % Colors for visual hierarchy
\usepackage{fancyhdr}   % Headers and footers
\usepackage{setspace}   % Line spacing
\usepackage{longtable}  % For tables that might span pages
\usepackage{xltabular}  % For flexible width long tables
\usepackage{float}      % For [H] option to fix table position
\usepackage{kotex}      % Additional Korean support insurance
\usepackage[hidelinks]{hyperref}

% --- STYLING COMMANDS ---

% Define a formal navy blue for headers
\definecolor{eduNavy}{RGB}{0, 43, 91}

% Section styling
\titleformat{\section}
  {\Large\sffamily\bfseries\color{eduNavy}}
  {\thesection}{1em}{}
  [\vspace{-0.5em}\hrule height 1pt]

\titleformat{\subsection}
  {\large\sffamily\bfseries\color{eduNavy}}
  {\thesubsection}{1em}{}

\titleformat{\subsubsection}
  {\normalsize\sffamily\bfseries\color{eduNavy}}
  {\thesubsubsection}{1em}{}

% List settings
\setlist[itemize]{label=\textbullet, leftmargin=1.5em, itemsep=0.2em}
\setlist[enumerate]{label=\arabic*., leftmargin=1.5em, itemsep=0.2em}

% Line spacing for readability
\setstretch{1.2}

% Header and Footer
\pagestyle{fancy}
\fancyhf{}
\fancyhead[L]{\small\sffamily \authors}
\fancyhead[R]{\small\sffamily \docversion}
\fancyfoot[C]{\thepage}
\renewcommand{\headrulewidth}{0.5pt}

% =============================================================================
% 사용법:
% 각 문서에서 다음과 같이 사용합니다:
%
% \documentclass[11pt, a4paper]{article}
% % =============================================================================
% GitHub Copilot 가이드 공통 프리앰블
% _preamble.tex - 모든 LaTeX 문서에서 \input{_preamble} 으로 재사용
% =============================================================================

% --- DOCUMENT VERSION CONTROL ---
% PDF 빌드 시점의 날짜와 시간(KST, UTC+9)이 자동으로 반영됩니다
% 참고: Overleaf 서버는 CET(UTC+1) 시간대를 사용하므로 +8로 보정합니다
\usepackage{datetime2}
\ExplSyntaxOn
\int_new:N \g_kst_hour_int
\int_new:N \g_kst_day_int
\int_new:N \g_kst_month_int
\int_new:N \g_kst_year_int
\int_gset:Nn \g_kst_hour_int { \the\time / 60 + 9 }
\int_gset:Nn \g_kst_day_int { \the\day }
\int_gset:Nn \g_kst_month_int { \the\month }
\int_gset:Nn \g_kst_year_int { \the\year }
\int_compare:nNnT { \g_kst_hour_int } > { 23 }
  {
    \int_gadd:Nn \g_kst_hour_int { -24 }
    \int_gincr:N \g_kst_day_int
  }
\newcommand{\docversion}{최종~수정:~\int_use:N \g_kst_year_int.~\int_compare:nNnTF { \g_kst_month_int } < { 10 } {0}{} \int_use:N \g_kst_month_int.~\int_compare:nNnTF { \g_kst_day_int } < { 10 } {0}{} \int_use:N \g_kst_day_int.
  \c_space_tl \int_compare:nNnTF { \g_kst_hour_int } < { 10 } {0}{} \int_use:N \g_kst_hour_int
  : \int_compare:nNnTF { \int_mod:nn {\the\time} {60} } < { 10 } {0}{} \int_eval:n { \int_mod:nn {\the\time} {60} }}
\newcommand{\docdate}{\int_use:N \g_kst_year_int 년~\int_compare:nNnTF { \g_kst_month_int } < { 10 } {0}{} \int_use:N \g_kst_month_int 월~\int_compare:nNnTF { \g_kst_day_int } < { 10 } {0}{} \int_use:N \g_kst_day_int 일}
\ExplSyntaxOff

% --- 저자 정보 (각 문서에서 재정의 가능) ---
\providecommand{\authors}{동의대학교 서진석(jsseo@deu.ac.kr)}

% --- UNIVERSAL PREAMBLE BLOCK ---
\usepackage[a4paper, top=3cm, bottom=3cm, left=2.5cm, right=2.5cm, headheight=14pt]{geometry}
\usepackage{fontspec}

% Language setup for Korean
\usepackage[korean, provide=*]{babel}

% Fonts: Using Noto Serif for body (formal) and Noto Sans for headers
\babelfont{rm}{Noto Serif CJK KR}
\babelfont{sf}{Noto Sans CJK KR}

% CJK 폰트는 이탤릭이 없으므로 폰트 대체 규칙 설정 (경고 방지)
\DeclareFontShape{TU}{NotoSerifCJKKR(0)}{m}{it}{<->ssub*NotoSerifCJKKR(0)/m/n}{}
\DeclareFontShape{TU}{NotoSerifCJKKR(0)}{m}{sl}{<->ssub*NotoSerifCJKKR(0)/m/n}{}
\DeclareFontShape{TU}{NotoSerifCJKKR(0)}{bx}{it}{<->ssub*NotoSerifCJKKR(0)/bx/n}{}
\DeclareFontShape{TU}{NotoSansCJKKR(0)}{m}{it}{<->ssub*NotoSansCJKKR(0)/m/n}{}
\DeclareFontShape{TU}{NotoSansCJKKR(0)}{m}{sl}{<->ssub*NotoSansCJKKR(0)/m/n}{}
\DeclareFontShape{TU}{NotoSansCJKKR(0)}{bx}{it}{<->ssub*NotoSansCJKKR(0)/bx/n}{}

% Essential Packages
\usepackage{enumitem}
\usepackage{booktabs}   % Professional tables
\usepackage{tabularx}   % Auto-width tables
\usepackage{titlesec}   % Section formatting
\usepackage{xcolor}     % Colors for visual hierarchy
\usepackage{fancyhdr}   % Headers and footers
\usepackage{setspace}   % Line spacing
\usepackage{longtable}  % For tables that might span pages
\usepackage{xltabular}  % For flexible width long tables
\usepackage{float}      % For [H] option to fix table position
\usepackage{kotex}      % Additional Korean support insurance
\usepackage[hidelinks]{hyperref}

% --- STYLING COMMANDS ---

% Define a formal navy blue for headers
\definecolor{eduNavy}{RGB}{0, 43, 91}

% Section styling
\titleformat{\section}
  {\Large\sffamily\bfseries\color{eduNavy}}
  {\thesection}{1em}{}
  [\vspace{-0.5em}\hrule height 1pt]

\titleformat{\subsection}
  {\large\sffamily\bfseries\color{eduNavy}}
  {\thesubsection}{1em}{}

\titleformat{\subsubsection}
  {\normalsize\sffamily\bfseries\color{eduNavy}}
  {\thesubsubsection}{1em}{}

% List settings
\setlist[itemize]{label=\textbullet, leftmargin=1.5em, itemsep=0.2em}
\setlist[enumerate]{label=\arabic*., leftmargin=1.5em, itemsep=0.2em}

% Line spacing for readability
\setstretch{1.2}

% Header and Footer
\pagestyle{fancy}
\fancyhf{}
\fancyhead[L]{\small\sffamily \authors}
\fancyhead[R]{\small\sffamily \docversion}
\fancyfoot[C]{\thepage}
\renewcommand{\headrulewidth}{0.5pt}

% =============================================================================
% 사용법:
% 각 문서에서 다음과 같이 사용합니다:
%
% \documentclass[11pt, a4paper]{article}
% \input{_preamble}
%
% % 필요시 저자 재정의
% \renewcommand{\authors}{작성자명 (email@example.com)}
%
% % 제목 설정
% \title{...}
% \author{\authors}
% \date{\docversion}
%
% \begin{document}
% \maketitle
% ...
% \end{document}
% =============================================================================

%
% % 필요시 저자 재정의
% \renewcommand{\authors}{작성자명 (email@example.com)}
%
% % 제목 설정
% \title{...}
% \author{\authors}
% \date{\docversion}
%
% \begin{document}
% \maketitle
% ...
% \end{document}
% =============================================================================

%
% % 필요시 저자 재정의
% \renewcommand{\authors}{작성자명 (email@example.com)}
%
% % 제목 설정
% \title{...}
% \author{\authors}
% \date{\docversion}
%
% \begin{document}
% \maketitle
% ...
% \end{document}
% =============================================================================


% --- 문서별 메타데이터 ---
\renewcommand{\authors}{동의대학교 서진석(jsseo@deu.ac.kr)}

\title{\vspace{-2.5cm} \Huge\sffamily\bfseries GitHub Copilot 맞춤 설정 가이드 \\ \large\vspace{0.2em} \textmd{프로젝트별 AI 지침 파일 작성법}\vspace{-0.5em}}
\author{\vspace{-0.5em}\authors}
\date{\vspace{-0.5em}\docversion}

\begin{document}

\maketitle
\thispagestyle{empty}
\vspace{-1.5cm}

\tableofcontents
\newpage

% ==============================================================================
\section{개요}
% ==============================================================================

\subsection{Copilot 지침 파일이란?}
GitHub Copilot 지침 파일은 AI 어시스턴트인 Copilot에게 \textbf{``이 프로젝트에서는 이렇게 답변해줘''}라고 미리 알려주는 설정 파일입니다.

\textbf{비유로 이해하기}:
\begin{itemize}
    \item 새로 채용한 조교에게 ``우리 연구실에서는 보고서를 이런 형식으로 작성해''라고 알려주는 매뉴얼
    \item 학생에게 ``이 과목 리포트는 APA 형식으로 작성하세요''라고 알려주는 가이드라인
    \item 공문 작성 시 참고하는 기관 내부 양식집
\end{itemize}

\subsection{왜 필요한가? (대학 업무 관점)}

\begin{table}[H]
    \centering
    \renewcommand{\arraystretch}{1.2}
    \begin{tabularx}{\textwidth}{@{}l X X@{}}
        \toprule
        \textbf{상황} & \textbf{지침 파일 없이} & \textbf{지침 파일 있으면} \\
        \midrule
        연구 제안서 작성 & 매번 형식 설명 필요 & 자동으로 형식 준수 \\
        논문 작성 & 학술 용어 번역 불일관 & 일관된 전문용어 사용 \\
        강의자료 제작 & 학과별 스타일 무시 & 학과 맞춤형 콘텐츠 \\
        행정 문서 & 공문서 형식 오류 & 기관 양식 자동 적용 \\
        \bottomrule
    \end{tabularx}
\end{table}

\subsection{주요 이점}
\begin{itemize}
    \item \textbf{일관성}: 동일한 프로젝트에서 항상 같은 스타일로 응답
    \item \textbf{효율성}: 매번 같은 설명을 반복할 필요 없음
    \item \textbf{협업}: 연구실/학과 단위로 설정 공유 가능
    \item \textbf{맥락 이해}: 프로젝트별 특수 용어나 요구사항 자동 적용
\end{itemize}

% ==============================================================================
\section{주요 파일 유형}
% ==============================================================================

GitHub Copilot은 네 가지 유형의 지침 파일을 지원합니다.

\subsection{copilot-instructions.md (프로젝트 기본 지침)}

\begin{table}[H]
    \centering
    \renewcommand{\arraystretch}{1.2}
    \begin{tabularx}{\textwidth}{@{}l X@{}}
        \toprule
        \textbf{항목} & \textbf{내용} \\
        \midrule
        용도 & 프로젝트 전체에 적용되는 기본 규칙 \\
        위치 & \texttt{.github/copilot-instructions.md} \\
        특징 & 프로젝트 폴더를 열면 자동으로 적용, 팀원과 Git으로 공유 가능 \\
        \bottomrule
    \end{tabularx}
\end{table}

\textbf{예시 (연구 프로젝트용)}:
\begin{quote}
\texttt{\# 연구 프로젝트 기본 지침} \\
\texttt{\#\# 프로젝트 개요} \\
이 프로젝트는 2026년도 NRF 기초연구사업 신청을 위한 연구 제안서 작성 프로젝트입니다. \\
\texttt{\#\# 문서 작성 원칙} \\
- 모든 문서는 한국어로 작성 \\
- 학술 용어는 영문 병기 (예: 인공지능(Artificial Intelligence)) \\
- 참고문헌은 APA 7th Edition 형식 사용
\end{quote}

\subsection{Instruction Files (.instructions.md)}

\begin{table}[H]
    \centering
    \renewcommand{\arraystretch}{1.2}
    \begin{tabularx}{\textwidth}{@{}l X@{}}
        \toprule
        \textbf{항목} & \textbf{내용} \\
        \midrule
        용도 & 특정 파일 유형이나 폴더에만 적용되는 세부 규칙 \\
        위치 & \texttt{.github/instructions/*.instructions.md} \\
        특징 & \texttt{applyTo} 패턴으로 적용 범위 지정 \\
        \bottomrule
    \end{tabularx}
\end{table}

\textbf{예시 (논문 작성용)}:
\begin{quote}
\texttt{---} \\
\texttt{applyTo: "**/paper/**/*.md"} \\
\texttt{---} \\
\texttt{\# 논문 작성 지침} \\
\texttt{\#\# 구조}: 초록(200단어 이내) $\rightarrow$ 서론 $\rightarrow$ 방법 $\rightarrow$ 결과 $\rightarrow$ 논의 $\rightarrow$ 결론 \\
\texttt{\#\# 인용 형식}: 본문 내 인용 (저자, 연도), 참고문헌 APA 7th Edition
\end{quote}

\subsection{Prompt Files (.prompt.md)}

\begin{table}[H]
    \centering
    \renewcommand{\arraystretch}{1.2}
    \begin{tabularx}{\textwidth}{@{}l X@{}}
        \toprule
        \textbf{항목} & \textbf{내용} \\
        \midrule
        용도 & 재사용 가능한 프롬프트 템플릿 \\
        위치 & \texttt{.github/prompts/*.prompt.md} \\
        특징 & 채팅창에서 \texttt{/} 명령으로 호출, 변수 사용 가능 (\texttt{\$\{selection\}}, \texttt{\$\{input:제목\}}) \\
        \bottomrule
    \end{tabularx}
\end{table}

\textbf{예시 (논문 요약용)}:
\begin{quote}
\texttt{---} \\
\texttt{description: "학술 논문을 구조화된 형식으로 요약"} \\
\texttt{mode: "ask"} \\
\texttt{---} \\
다음 논문을 요약해주세요: \texttt{\$\{selection\}} \\
형식: 연구 목적(1-2문장), 연구 방법(3-4문장), 주요 결과(목록), 인용 정보(APA)
\end{quote}

\subsection{AGENTS.md (에이전트 모드용)}

\begin{table}[H]
    \centering
    \renewcommand{\arraystretch}{1.2}
    \begin{tabularx}{\textwidth}{@{}l X@{}}
        \toprule
        \textbf{항목} & \textbf{내용} \\
        \midrule
        용도 & Copilot 에이전트 모드에서 자율적 작업 수행 시 참조 \\
        위치 & 프로젝트 루트 또는 하위 폴더의 \texttt{AGENTS.md} \\
        특징 & 에이전트가 코드 수정, 명령 실행 시 참조 \\
        \bottomrule
    \end{tabularx}
\end{table}

% ==============================================================================
\section{지침 설정 방법}
% ==============================================================================

\subsection{프로젝트 레벨 설정 (팀 공유용)}
\textbf{장점}: Git으로 공유, 팀원 모두 동일 설정 사용

\begin{enumerate}
    \item 프로젝트 폴더에 \texttt{.github} 폴더 생성
    \item \texttt{copilot-instructions.md} 파일 생성
    \item 지침 내용 작성
    \item Git commit하여 팀과 공유
\end{enumerate}

\textbf{권장 프로젝트 폴더 구조}:
\begin{quote}
\texttt{프로젝트폴더/} \\
\texttt{├── .github/} \\
\texttt{│   ├── copilot-instructions.md} \hfill \textit{\# 기본 지침} \\
\texttt{│   ├── instructions/} \hfill \textit{\# 세부 지침} \\
\texttt{│   │   ├── paper.instructions.md} \\
\texttt{│   │   └── proposal.instructions.md} \\
\texttt{│   └── prompts/} \hfill \textit{\# 프롬프트 템플릿} \\
\texttt{│       ├── summarize.prompt.md} \\
\texttt{│       └── translate.prompt.md} \\
\texttt{└── (프로젝트 파일들...)}
\end{quote}

\subsection{개인 설정 (모든 프로젝트 적용)}
\textbf{장점}: 한 번 설정으로 모든 프로젝트에서 사용

\begin{enumerate}
    \item VS Code 설정 열기 (\texttt{Cmd+,} 또는 \texttt{Ctrl+,})
    \item ``copilot chat prompts'' 검색
    \item ``Chat: Prompts Folders'' 설정에 폴더 경로 추가
    \item 해당 폴더에 \texttt{.instructions.md} 또는 \texttt{.prompt.md} 파일 생성
\end{enumerate}

\subsection{VS Code 설정 예시}
\begin{quote}
\texttt{\{} \\
\texttt{\ \ "github.copilot.chat.codeGeneration.useInstructionFiles": true,} \\
\texttt{\ \ "chat.promptFiles.locations": [} \\
\texttt{\ \ \ \ \{ "path": "/Users/username/my-copilot-prompts" \}} \\
\texttt{\ \ ]} \\
\texttt{\}}
\end{quote}

% ==============================================================================
\section{구성원별 활용 예시}
% ==============================================================================

\subsection{교수용}

\subsubsection{연구 제안서 작성 지침}
\textbf{파일}: \texttt{.github/instructions/proposal.instructions.md}

\begin{quote}
\texttt{---} \\
\texttt{applyTo: "**/proposal/**"} \\
\texttt{---} \\
\texttt{\# 연구 제안서 작성 지침} \\
\textbf{작성 원칙}: \\
1. 간결하고 명확한 문장: 한 문장에 하나의 아이디어만 \\
2. 객관적 근거 제시: 주장마다 참고문헌 인용 \\
3. 정량적 표현 선호: ``많은'' 대신 ``약 70\%의'' \\
\textbf{필수 포함 요소}: 연구 목적, 국내외 연구 동향, 연구 내용 및 방법, 기대 효과
\end{quote}

\subsubsection{논문 작성 지침}
\textbf{파일}: \texttt{.github/instructions/paper.instructions.md}

\begin{quote}
\texttt{---} \\
\texttt{applyTo: "**/papers/**/*.md"} \\
\texttt{---} \\
\texttt{\# 학술 논문 작성 지침} \\
\textbf{문체}: 수동태보다 능동태 선호, 1인칭 복수 사용 (We propose, Our approach) \\
\textbf{인용 형식}: IEEE 형식 [1], [2], [3-5] / 저자 언급 시: Kim et al. [1] proposed...
\end{quote}

\subsection{학생용}

\subsubsection{과제 및 리포트 작성 지침}
\textbf{파일}: \texttt{.github/instructions/assignment.instructions.md}

\begin{quote}
\texttt{---} \\
\texttt{applyTo: "**/assignments/**"} \\
\texttt{---} \\
\texttt{\# 과제 작성 지침} \\
\textbf{기본 형식}: 맑은 고딕 11pt, 줄간격 1.5줄, 여백 상하좌우 2.5cm \\
\textbf{AI 활용 시 주의사항}: \\
- AI는 아이디어 브레인스토밍에 활용 \\
- 생성된 내용 반드시 검증 \\
- AI 활용 사실 명시 (교수님 지침 따름)
\end{quote}

\subsection{연구원용}

\subsubsection{실험 보고서 작성 지침}
\textbf{파일}: \texttt{.github/instructions/experiment.instructions.md}

\begin{quote}
\texttt{---} \\
\texttt{applyTo: "**/experiments/**"} \\
\texttt{---} \\
\texttt{\# 실험 보고서 작성 지침} \\
\textbf{보고서 구조}: 실험 개요 $\rightarrow$ 실험 환경 $\rightarrow$ 실험 방법 $\rightarrow$ 실험 결과 $\rightarrow$ 결론 \\
\textbf{재현성 보장 체크리스트}: \\
$\square$ 랜덤 시드 기록 / $\square$ 코드 버전 (commit hash) / $\square$ 데이터셋 버전 / $\square$ 환경 설정 파일 첨부
\end{quote}

\subsection{행정직원용}

\subsubsection{공문서 작성 지침}
\textbf{파일}: \texttt{.github/instructions/official-document.instructions.md}

\begin{quote}
\texttt{---} \\
\texttt{applyTo: "**/documents/**"} \\
\texttt{---} \\
\texttt{\# 공문서 작성 지침} \\
\textbf{문서 형식}: A4 세로, 휴먼명조 또는 바탕체, 본문 13pt, 줄간격 160\% \\
\textbf{문체 규칙}: 경어체 사용 (\textasciitilde습니다, \textasciitilde합니다), 간결하고 명확한 표현 \\
\textbf{날짜 표기}: 2025. 11. 26. / \textbf{금액 표기}: 금 1,000,000원(일백만 원)
\end{quote}

% ==============================================================================
\section{applyTo 패턴 활용법}
% ==============================================================================

\begin{table}[H]
    \centering
    \renewcommand{\arraystretch}{1.2}
    \begin{tabularx}{\textwidth}{@{}l l X@{}}
        \toprule
        \textbf{패턴} & \textbf{설명} & \textbf{적용 대상} \\
        \midrule
        \texttt{**/*.md} & 모든 Markdown 파일 & 프로젝트 내 모든 .md 파일 \\
        \texttt{**/paper/**} & paper 폴더 내 모든 파일 & paper 폴더 하위 전체 \\
        \texttt{src/**/*.py} & src 폴더의 Python 파일 & src 하위 모든 .py 파일 \\
        \texttt{*.{md,txt}} & 루트의 md 또는 txt & 최상위 폴더만 \\
        \bottomrule
    \end{tabularx}
\end{table}

\textbf{주의사항}:
\begin{itemize}
    \item 패턴은 프로젝트 루트 기준
    \item 대소문자 구분 (OS에 따라 다름)
    \item 여러 instructions 파일이 같은 파일에 적용될 수 있음
    \item 충돌 시 더 구체적인 패턴 우선
\end{itemize}

% ==============================================================================
\section{시작하기 (단계별 가이드)}
% ==============================================================================

\subsection{Step 1: 첫 지침 파일 만들기}
\begin{enumerate}
    \item \textbf{프로젝트 폴더 열기}: File $\rightarrow$ Open Folder
    \item \textbf{\texttt{.github} 폴더 생성}: 탐색기에서 우클릭 $\rightarrow$ New Folder
    \item \textbf{\texttt{copilot-instructions.md} 파일 생성}
    \item \textbf{기본 템플릿 작성}:
\end{enumerate}

\begin{quote}
\texttt{\# 프로젝트 지침} \\
\texttt{\#\# 기본 정보} \\
이 프로젝트는 [프로젝트 설명]을 위한 것입니다. \\
\texttt{\#\# 작성 규칙} \\
- 한국어로 작성 \\
- 존댓말 사용 \\
- 간결한 문장 선호 \\
\texttt{\#\# 용어집} \\
- [용어1]: [설명]
\end{quote}

\subsection{Step 2: 동작 확인}
\begin{enumerate}
    \item Copilot Chat 열기 (\texttt{Cmd+Shift+I} 또는 \texttt{Ctrl+Shift+I})
    \item 테스트 질문: ``이 프로젝트에 대해 알려줘''
    \item 지침 파일 내용이 반영되었는지 확인
\end{enumerate}

\subsection{Step 3: 활용 확장}
\begin{enumerate}
    \item \texttt{.github/instructions/} 폴더 생성하여 상황별 지침 추가
    \item \texttt{.github/prompts/} 폴더 생성하여 프롬프트 템플릿 추가
    \item VS Code 설정에서 개인 프롬프트 폴더 경로 추가 (선택)
\end{enumerate}

% ==============================================================================
\section{자주 묻는 질문 (FAQ)}
% ==============================================================================

\subsection{Q1: 지침 파일이 적용되었는지 어떻게 확인하나요?}
Copilot Chat에서 ``현재 적용된 지침을 알려줘''라고 물어보면 됩니다. 또는 채팅창 상단의 컨텍스트 영역에서 첨부된 파일을 확인할 수 있습니다.

\subsection{Q2: 여러 지침 파일이 충돌하면 어떻게 되나요?}
모든 해당 지침이 합쳐져서 적용됩니다. 내용이 충돌하는 경우, Copilot이 가장 관련성 높은 규칙을 선택하거나 두 규칙을 조합합니다.

\subsection{Q3: 지침 파일을 팀원과 공유하려면?}
\texttt{.github} 폴더를 Git에 커밋하면 됩니다. 팀원이 저장소를 clone하면 자동으로 같은 지침이 적용됩니다.

\subsection{Q4: 지침이 너무 길면 문제가 있나요?}
Copilot은 컨텍스트 길이에 제한이 있으므로, 핵심 내용 위주로 간결하게 작성하는 것이 좋습니다. 긴 지침보다는 여러 개의 짧은 지침 파일로 분리하세요.

\subsection{Q5: 영문 응답을 한글로 바꾸고 싶어요.}
\texttt{copilot-instructions.md}에 다음을 추가하세요:
\begin{quote}
\texttt{\#\# 언어 설정} \\
- 모든 응답은 한국어로 작성 \\
- 전문용어는 영문 병기 가능
\end{quote}

% ==============================================================================
\section{빠른 참조}
% ==============================================================================

\begin{table}[H]
    \centering
    \renewcommand{\arraystretch}{1.2}
    \begin{tabularx}{\textwidth}{@{}l l l X@{}}
        \toprule
        \textbf{파일 유형} & \textbf{확장자} & \textbf{위치} & \textbf{용도} \\
        \midrule
        기본 지침 & \texttt{copilot-instructions.md} & \texttt{.github/} & 프로젝트 전체 규칙 \\
        세부 지침 & \texttt{*.instructions.md} & \texttt{.github/instructions/} & 특정 파일/폴더 규칙 \\
        프롬프트 & \texttt{*.prompt.md} & \texttt{.github/prompts/} & 재사용 템플릿 \\
        에이전트 & \texttt{AGENTS.md} & 루트 또는 하위 폴더 & 에이전트 모드 지침 \\
        \bottomrule
    \end{tabularx}
\end{table}

\begin{table}[H]
    \centering
    \renewcommand{\arraystretch}{1.2}
    \begin{tabularx}{\textwidth}{@{}l X X@{}}
        \toprule
        \textbf{동작} & \textbf{Mac} & \textbf{Windows} \\
        \midrule
        Copilot Chat 열기 & \texttt{Cmd+Shift+I} & \texttt{Ctrl+Shift+I} \\
        인라인 제안 수락 & \texttt{Tab} & \texttt{Tab} \\
        다음 제안 & \texttt{Alt+]} & \texttt{Alt+]} \\
        이전 제안 & \texttt{Alt+[} & \texttt{Alt+[} \\
        제안 거부 & \texttt{Esc} & \texttt{Esc} \\
        \bottomrule
    \end{tabularx}
\end{table}

\end{document}
