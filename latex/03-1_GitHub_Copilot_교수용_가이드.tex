\documentclass[11pt, a4paper]{article}
% =============================================================================
% GitHub Copilot 가이드 공통 프리앰블
% _preamble.tex - 모든 LaTeX 문서에서 % =============================================================================
% GitHub Copilot 가이드 공통 프리앰블
% _preamble.tex - 모든 LaTeX 문서에서 % =============================================================================
% GitHub Copilot 가이드 공통 프리앰블
% _preamble.tex - 모든 LaTeX 문서에서 \input{_preamble} 으로 재사용
% =============================================================================

% --- DOCUMENT VERSION CONTROL ---
% PDF 빌드 시점의 날짜와 시간(KST, UTC+9)이 자동으로 반영됩니다
% 참고: Overleaf 서버는 CET(UTC+1) 시간대를 사용하므로 +8로 보정합니다
\usepackage{datetime2}
\ExplSyntaxOn
\int_new:N \g_kst_hour_int
\int_new:N \g_kst_day_int
\int_new:N \g_kst_month_int
\int_new:N \g_kst_year_int
\int_gset:Nn \g_kst_hour_int { \the\time / 60 + 9 }
\int_gset:Nn \g_kst_day_int { \the\day }
\int_gset:Nn \g_kst_month_int { \the\month }
\int_gset:Nn \g_kst_year_int { \the\year }
\int_compare:nNnT { \g_kst_hour_int } > { 23 }
  {
    \int_gadd:Nn \g_kst_hour_int { -24 }
    \int_gincr:N \g_kst_day_int
  }
\newcommand{\docversion}{최종~수정:~\int_use:N \g_kst_year_int.~\int_compare:nNnTF { \g_kst_month_int } < { 10 } {0}{} \int_use:N \g_kst_month_int.~\int_compare:nNnTF { \g_kst_day_int } < { 10 } {0}{} \int_use:N \g_kst_day_int.
  \c_space_tl \int_compare:nNnTF { \g_kst_hour_int } < { 10 } {0}{} \int_use:N \g_kst_hour_int
  : \int_compare:nNnTF { \int_mod:nn {\the\time} {60} } < { 10 } {0}{} \int_eval:n { \int_mod:nn {\the\time} {60} }}
\newcommand{\docdate}{\int_use:N \g_kst_year_int 년~\int_compare:nNnTF { \g_kst_month_int } < { 10 } {0}{} \int_use:N \g_kst_month_int 월~\int_compare:nNnTF { \g_kst_day_int } < { 10 } {0}{} \int_use:N \g_kst_day_int 일}
\ExplSyntaxOff

% --- 저자 정보 (각 문서에서 재정의 가능) ---
\providecommand{\authors}{동의대학교 서진석(jsseo@deu.ac.kr)}

% --- UNIVERSAL PREAMBLE BLOCK ---
\usepackage[a4paper, top=3cm, bottom=3cm, left=2.5cm, right=2.5cm, headheight=14pt]{geometry}
\usepackage{fontspec}

% Language setup for Korean
\usepackage[korean, provide=*]{babel}

% Fonts: Using Noto Serif for body (formal) and Noto Sans for headers
\babelfont{rm}{Noto Serif CJK KR}
\babelfont{sf}{Noto Sans CJK KR}

% CJK 폰트는 이탤릭이 없으므로 폰트 대체 규칙 설정 (경고 방지)
\DeclareFontShape{TU}{NotoSerifCJKKR(0)}{m}{it}{<->ssub*NotoSerifCJKKR(0)/m/n}{}
\DeclareFontShape{TU}{NotoSerifCJKKR(0)}{m}{sl}{<->ssub*NotoSerifCJKKR(0)/m/n}{}
\DeclareFontShape{TU}{NotoSerifCJKKR(0)}{bx}{it}{<->ssub*NotoSerifCJKKR(0)/bx/n}{}
\DeclareFontShape{TU}{NotoSansCJKKR(0)}{m}{it}{<->ssub*NotoSansCJKKR(0)/m/n}{}
\DeclareFontShape{TU}{NotoSansCJKKR(0)}{m}{sl}{<->ssub*NotoSansCJKKR(0)/m/n}{}
\DeclareFontShape{TU}{NotoSansCJKKR(0)}{bx}{it}{<->ssub*NotoSansCJKKR(0)/bx/n}{}

% Essential Packages
\usepackage{enumitem}
\usepackage{booktabs}   % Professional tables
\usepackage{tabularx}   % Auto-width tables
\usepackage{titlesec}   % Section formatting
\usepackage{xcolor}     % Colors for visual hierarchy
\usepackage{fancyhdr}   % Headers and footers
\usepackage{setspace}   % Line spacing
\usepackage{longtable}  % For tables that might span pages
\usepackage{xltabular}  % For flexible width long tables
\usepackage{float}      % For [H] option to fix table position
\usepackage{kotex}      % Additional Korean support insurance
\usepackage[hidelinks]{hyperref}

% --- STYLING COMMANDS ---

% Define a formal navy blue for headers
\definecolor{eduNavy}{RGB}{0, 43, 91}

% Section styling
\titleformat{\section}
  {\Large\sffamily\bfseries\color{eduNavy}}
  {\thesection}{1em}{}
  [\vspace{-0.5em}\hrule height 1pt]

\titleformat{\subsection}
  {\large\sffamily\bfseries\color{eduNavy}}
  {\thesubsection}{1em}{}

\titleformat{\subsubsection}
  {\normalsize\sffamily\bfseries\color{eduNavy}}
  {\thesubsubsection}{1em}{}

% List settings
\setlist[itemize]{label=\textbullet, leftmargin=1.5em, itemsep=0.2em}
\setlist[enumerate]{label=\arabic*., leftmargin=1.5em, itemsep=0.2em}

% Line spacing for readability
\setstretch{1.2}

% Header and Footer
\pagestyle{fancy}
\fancyhf{}
\fancyhead[L]{\small\sffamily \authors}
\fancyhead[R]{\small\sffamily \docversion}
\fancyfoot[C]{\thepage}
\renewcommand{\headrulewidth}{0.5pt}

% =============================================================================
% 사용법:
% 각 문서에서 다음과 같이 사용합니다:
%
% \documentclass[11pt, a4paper]{article}
% \input{_preamble}
%
% % 필요시 저자 재정의
% \renewcommand{\authors}{작성자명 (email@example.com)}
%
% % 제목 설정
% \title{...}
% \author{\authors}
% \date{\docversion}
%
% \begin{document}
% \maketitle
% ...
% \end{document}
% =============================================================================
 으로 재사용
% =============================================================================

% --- DOCUMENT VERSION CONTROL ---
% PDF 빌드 시점의 날짜와 시간(KST, UTC+9)이 자동으로 반영됩니다
% 참고: Overleaf 서버는 CET(UTC+1) 시간대를 사용하므로 +8로 보정합니다
\usepackage{datetime2}
\ExplSyntaxOn
\int_new:N \g_kst_hour_int
\int_new:N \g_kst_day_int
\int_new:N \g_kst_month_int
\int_new:N \g_kst_year_int
\int_gset:Nn \g_kst_hour_int { \the\time / 60 + 9 }
\int_gset:Nn \g_kst_day_int { \the\day }
\int_gset:Nn \g_kst_month_int { \the\month }
\int_gset:Nn \g_kst_year_int { \the\year }
\int_compare:nNnT { \g_kst_hour_int } > { 23 }
  {
    \int_gadd:Nn \g_kst_hour_int { -24 }
    \int_gincr:N \g_kst_day_int
  }
\newcommand{\docversion}{최종~수정:~\int_use:N \g_kst_year_int.~\int_compare:nNnTF { \g_kst_month_int } < { 10 } {0}{} \int_use:N \g_kst_month_int.~\int_compare:nNnTF { \g_kst_day_int } < { 10 } {0}{} \int_use:N \g_kst_day_int.
  \c_space_tl \int_compare:nNnTF { \g_kst_hour_int } < { 10 } {0}{} \int_use:N \g_kst_hour_int
  : \int_compare:nNnTF { \int_mod:nn {\the\time} {60} } < { 10 } {0}{} \int_eval:n { \int_mod:nn {\the\time} {60} }}
\newcommand{\docdate}{\int_use:N \g_kst_year_int 년~\int_compare:nNnTF { \g_kst_month_int } < { 10 } {0}{} \int_use:N \g_kst_month_int 월~\int_compare:nNnTF { \g_kst_day_int } < { 10 } {0}{} \int_use:N \g_kst_day_int 일}
\ExplSyntaxOff

% --- 저자 정보 (각 문서에서 재정의 가능) ---
\providecommand{\authors}{동의대학교 서진석(jsseo@deu.ac.kr)}

% --- UNIVERSAL PREAMBLE BLOCK ---
\usepackage[a4paper, top=3cm, bottom=3cm, left=2.5cm, right=2.5cm, headheight=14pt]{geometry}
\usepackage{fontspec}

% Language setup for Korean
\usepackage[korean, provide=*]{babel}

% Fonts: Using Noto Serif for body (formal) and Noto Sans for headers
\babelfont{rm}{Noto Serif CJK KR}
\babelfont{sf}{Noto Sans CJK KR}

% CJK 폰트는 이탤릭이 없으므로 폰트 대체 규칙 설정 (경고 방지)
\DeclareFontShape{TU}{NotoSerifCJKKR(0)}{m}{it}{<->ssub*NotoSerifCJKKR(0)/m/n}{}
\DeclareFontShape{TU}{NotoSerifCJKKR(0)}{m}{sl}{<->ssub*NotoSerifCJKKR(0)/m/n}{}
\DeclareFontShape{TU}{NotoSerifCJKKR(0)}{bx}{it}{<->ssub*NotoSerifCJKKR(0)/bx/n}{}
\DeclareFontShape{TU}{NotoSansCJKKR(0)}{m}{it}{<->ssub*NotoSansCJKKR(0)/m/n}{}
\DeclareFontShape{TU}{NotoSansCJKKR(0)}{m}{sl}{<->ssub*NotoSansCJKKR(0)/m/n}{}
\DeclareFontShape{TU}{NotoSansCJKKR(0)}{bx}{it}{<->ssub*NotoSansCJKKR(0)/bx/n}{}

% Essential Packages
\usepackage{enumitem}
\usepackage{booktabs}   % Professional tables
\usepackage{tabularx}   % Auto-width tables
\usepackage{titlesec}   % Section formatting
\usepackage{xcolor}     % Colors for visual hierarchy
\usepackage{fancyhdr}   % Headers and footers
\usepackage{setspace}   % Line spacing
\usepackage{longtable}  % For tables that might span pages
\usepackage{xltabular}  % For flexible width long tables
\usepackage{float}      % For [H] option to fix table position
\usepackage{kotex}      % Additional Korean support insurance
\usepackage[hidelinks]{hyperref}

% --- STYLING COMMANDS ---

% Define a formal navy blue for headers
\definecolor{eduNavy}{RGB}{0, 43, 91}

% Section styling
\titleformat{\section}
  {\Large\sffamily\bfseries\color{eduNavy}}
  {\thesection}{1em}{}
  [\vspace{-0.5em}\hrule height 1pt]

\titleformat{\subsection}
  {\large\sffamily\bfseries\color{eduNavy}}
  {\thesubsection}{1em}{}

\titleformat{\subsubsection}
  {\normalsize\sffamily\bfseries\color{eduNavy}}
  {\thesubsubsection}{1em}{}

% List settings
\setlist[itemize]{label=\textbullet, leftmargin=1.5em, itemsep=0.2em}
\setlist[enumerate]{label=\arabic*., leftmargin=1.5em, itemsep=0.2em}

% Line spacing for readability
\setstretch{1.2}

% Header and Footer
\pagestyle{fancy}
\fancyhf{}
\fancyhead[L]{\small\sffamily \authors}
\fancyhead[R]{\small\sffamily \docversion}
\fancyfoot[C]{\thepage}
\renewcommand{\headrulewidth}{0.5pt}

% =============================================================================
% 사용법:
% 각 문서에서 다음과 같이 사용합니다:
%
% \documentclass[11pt, a4paper]{article}
% % =============================================================================
% GitHub Copilot 가이드 공통 프리앰블
% _preamble.tex - 모든 LaTeX 문서에서 \input{_preamble} 으로 재사용
% =============================================================================

% --- DOCUMENT VERSION CONTROL ---
% PDF 빌드 시점의 날짜와 시간(KST, UTC+9)이 자동으로 반영됩니다
% 참고: Overleaf 서버는 CET(UTC+1) 시간대를 사용하므로 +8로 보정합니다
\usepackage{datetime2}
\ExplSyntaxOn
\int_new:N \g_kst_hour_int
\int_new:N \g_kst_day_int
\int_new:N \g_kst_month_int
\int_new:N \g_kst_year_int
\int_gset:Nn \g_kst_hour_int { \the\time / 60 + 9 }
\int_gset:Nn \g_kst_day_int { \the\day }
\int_gset:Nn \g_kst_month_int { \the\month }
\int_gset:Nn \g_kst_year_int { \the\year }
\int_compare:nNnT { \g_kst_hour_int } > { 23 }
  {
    \int_gadd:Nn \g_kst_hour_int { -24 }
    \int_gincr:N \g_kst_day_int
  }
\newcommand{\docversion}{최종~수정:~\int_use:N \g_kst_year_int.~\int_compare:nNnTF { \g_kst_month_int } < { 10 } {0}{} \int_use:N \g_kst_month_int.~\int_compare:nNnTF { \g_kst_day_int } < { 10 } {0}{} \int_use:N \g_kst_day_int.
  \c_space_tl \int_compare:nNnTF { \g_kst_hour_int } < { 10 } {0}{} \int_use:N \g_kst_hour_int
  : \int_compare:nNnTF { \int_mod:nn {\the\time} {60} } < { 10 } {0}{} \int_eval:n { \int_mod:nn {\the\time} {60} }}
\newcommand{\docdate}{\int_use:N \g_kst_year_int 년~\int_compare:nNnTF { \g_kst_month_int } < { 10 } {0}{} \int_use:N \g_kst_month_int 월~\int_compare:nNnTF { \g_kst_day_int } < { 10 } {0}{} \int_use:N \g_kst_day_int 일}
\ExplSyntaxOff

% --- 저자 정보 (각 문서에서 재정의 가능) ---
\providecommand{\authors}{동의대학교 서진석(jsseo@deu.ac.kr)}

% --- UNIVERSAL PREAMBLE BLOCK ---
\usepackage[a4paper, top=3cm, bottom=3cm, left=2.5cm, right=2.5cm, headheight=14pt]{geometry}
\usepackage{fontspec}

% Language setup for Korean
\usepackage[korean, provide=*]{babel}

% Fonts: Using Noto Serif for body (formal) and Noto Sans for headers
\babelfont{rm}{Noto Serif CJK KR}
\babelfont{sf}{Noto Sans CJK KR}

% CJK 폰트는 이탤릭이 없으므로 폰트 대체 규칙 설정 (경고 방지)
\DeclareFontShape{TU}{NotoSerifCJKKR(0)}{m}{it}{<->ssub*NotoSerifCJKKR(0)/m/n}{}
\DeclareFontShape{TU}{NotoSerifCJKKR(0)}{m}{sl}{<->ssub*NotoSerifCJKKR(0)/m/n}{}
\DeclareFontShape{TU}{NotoSerifCJKKR(0)}{bx}{it}{<->ssub*NotoSerifCJKKR(0)/bx/n}{}
\DeclareFontShape{TU}{NotoSansCJKKR(0)}{m}{it}{<->ssub*NotoSansCJKKR(0)/m/n}{}
\DeclareFontShape{TU}{NotoSansCJKKR(0)}{m}{sl}{<->ssub*NotoSansCJKKR(0)/m/n}{}
\DeclareFontShape{TU}{NotoSansCJKKR(0)}{bx}{it}{<->ssub*NotoSansCJKKR(0)/bx/n}{}

% Essential Packages
\usepackage{enumitem}
\usepackage{booktabs}   % Professional tables
\usepackage{tabularx}   % Auto-width tables
\usepackage{titlesec}   % Section formatting
\usepackage{xcolor}     % Colors for visual hierarchy
\usepackage{fancyhdr}   % Headers and footers
\usepackage{setspace}   % Line spacing
\usepackage{longtable}  % For tables that might span pages
\usepackage{xltabular}  % For flexible width long tables
\usepackage{float}      % For [H] option to fix table position
\usepackage{kotex}      % Additional Korean support insurance
\usepackage[hidelinks]{hyperref}

% --- STYLING COMMANDS ---

% Define a formal navy blue for headers
\definecolor{eduNavy}{RGB}{0, 43, 91}

% Section styling
\titleformat{\section}
  {\Large\sffamily\bfseries\color{eduNavy}}
  {\thesection}{1em}{}
  [\vspace{-0.5em}\hrule height 1pt]

\titleformat{\subsection}
  {\large\sffamily\bfseries\color{eduNavy}}
  {\thesubsection}{1em}{}

\titleformat{\subsubsection}
  {\normalsize\sffamily\bfseries\color{eduNavy}}
  {\thesubsubsection}{1em}{}

% List settings
\setlist[itemize]{label=\textbullet, leftmargin=1.5em, itemsep=0.2em}
\setlist[enumerate]{label=\arabic*., leftmargin=1.5em, itemsep=0.2em}

% Line spacing for readability
\setstretch{1.2}

% Header and Footer
\pagestyle{fancy}
\fancyhf{}
\fancyhead[L]{\small\sffamily \authors}
\fancyhead[R]{\small\sffamily \docversion}
\fancyfoot[C]{\thepage}
\renewcommand{\headrulewidth}{0.5pt}

% =============================================================================
% 사용법:
% 각 문서에서 다음과 같이 사용합니다:
%
% \documentclass[11pt, a4paper]{article}
% \input{_preamble}
%
% % 필요시 저자 재정의
% \renewcommand{\authors}{작성자명 (email@example.com)}
%
% % 제목 설정
% \title{...}
% \author{\authors}
% \date{\docversion}
%
% \begin{document}
% \maketitle
% ...
% \end{document}
% =============================================================================

%
% % 필요시 저자 재정의
% \renewcommand{\authors}{작성자명 (email@example.com)}
%
% % 제목 설정
% \title{...}
% \author{\authors}
% \date{\docversion}
%
% \begin{document}
% \maketitle
% ...
% \end{document}
% =============================================================================
 으로 재사용
% =============================================================================

% --- DOCUMENT VERSION CONTROL ---
% PDF 빌드 시점의 날짜와 시간(KST, UTC+9)이 자동으로 반영됩니다
% 참고: Overleaf 서버는 CET(UTC+1) 시간대를 사용하므로 +8로 보정합니다
\usepackage{datetime2}
\ExplSyntaxOn
\int_new:N \g_kst_hour_int
\int_new:N \g_kst_day_int
\int_new:N \g_kst_month_int
\int_new:N \g_kst_year_int
\int_gset:Nn \g_kst_hour_int { \the\time / 60 + 9 }
\int_gset:Nn \g_kst_day_int { \the\day }
\int_gset:Nn \g_kst_month_int { \the\month }
\int_gset:Nn \g_kst_year_int { \the\year }
\int_compare:nNnT { \g_kst_hour_int } > { 23 }
  {
    \int_gadd:Nn \g_kst_hour_int { -24 }
    \int_gincr:N \g_kst_day_int
  }
\newcommand{\docversion}{최종~수정:~\int_use:N \g_kst_year_int.~\int_compare:nNnTF { \g_kst_month_int } < { 10 } {0}{} \int_use:N \g_kst_month_int.~\int_compare:nNnTF { \g_kst_day_int } < { 10 } {0}{} \int_use:N \g_kst_day_int.
  \c_space_tl \int_compare:nNnTF { \g_kst_hour_int } < { 10 } {0}{} \int_use:N \g_kst_hour_int
  : \int_compare:nNnTF { \int_mod:nn {\the\time} {60} } < { 10 } {0}{} \int_eval:n { \int_mod:nn {\the\time} {60} }}
\newcommand{\docdate}{\int_use:N \g_kst_year_int 년~\int_compare:nNnTF { \g_kst_month_int } < { 10 } {0}{} \int_use:N \g_kst_month_int 월~\int_compare:nNnTF { \g_kst_day_int } < { 10 } {0}{} \int_use:N \g_kst_day_int 일}
\ExplSyntaxOff

% --- 저자 정보 (각 문서에서 재정의 가능) ---
\providecommand{\authors}{동의대학교 서진석(jsseo@deu.ac.kr)}

% --- UNIVERSAL PREAMBLE BLOCK ---
\usepackage[a4paper, top=3cm, bottom=3cm, left=2.5cm, right=2.5cm, headheight=14pt]{geometry}
\usepackage{fontspec}

% Language setup for Korean
\usepackage[korean, provide=*]{babel}

% Fonts: Using Noto Serif for body (formal) and Noto Sans for headers
\babelfont{rm}{Noto Serif CJK KR}
\babelfont{sf}{Noto Sans CJK KR}

% CJK 폰트는 이탤릭이 없으므로 폰트 대체 규칙 설정 (경고 방지)
\DeclareFontShape{TU}{NotoSerifCJKKR(0)}{m}{it}{<->ssub*NotoSerifCJKKR(0)/m/n}{}
\DeclareFontShape{TU}{NotoSerifCJKKR(0)}{m}{sl}{<->ssub*NotoSerifCJKKR(0)/m/n}{}
\DeclareFontShape{TU}{NotoSerifCJKKR(0)}{bx}{it}{<->ssub*NotoSerifCJKKR(0)/bx/n}{}
\DeclareFontShape{TU}{NotoSansCJKKR(0)}{m}{it}{<->ssub*NotoSansCJKKR(0)/m/n}{}
\DeclareFontShape{TU}{NotoSansCJKKR(0)}{m}{sl}{<->ssub*NotoSansCJKKR(0)/m/n}{}
\DeclareFontShape{TU}{NotoSansCJKKR(0)}{bx}{it}{<->ssub*NotoSansCJKKR(0)/bx/n}{}

% Essential Packages
\usepackage{enumitem}
\usepackage{booktabs}   % Professional tables
\usepackage{tabularx}   % Auto-width tables
\usepackage{titlesec}   % Section formatting
\usepackage{xcolor}     % Colors for visual hierarchy
\usepackage{fancyhdr}   % Headers and footers
\usepackage{setspace}   % Line spacing
\usepackage{longtable}  % For tables that might span pages
\usepackage{xltabular}  % For flexible width long tables
\usepackage{float}      % For [H] option to fix table position
\usepackage{kotex}      % Additional Korean support insurance
\usepackage[hidelinks]{hyperref}

% --- STYLING COMMANDS ---

% Define a formal navy blue for headers
\definecolor{eduNavy}{RGB}{0, 43, 91}

% Section styling
\titleformat{\section}
  {\Large\sffamily\bfseries\color{eduNavy}}
  {\thesection}{1em}{}
  [\vspace{-0.5em}\hrule height 1pt]

\titleformat{\subsection}
  {\large\sffamily\bfseries\color{eduNavy}}
  {\thesubsection}{1em}{}

\titleformat{\subsubsection}
  {\normalsize\sffamily\bfseries\color{eduNavy}}
  {\thesubsubsection}{1em}{}

% List settings
\setlist[itemize]{label=\textbullet, leftmargin=1.5em, itemsep=0.2em}
\setlist[enumerate]{label=\arabic*., leftmargin=1.5em, itemsep=0.2em}

% Line spacing for readability
\setstretch{1.2}

% Header and Footer
\pagestyle{fancy}
\fancyhf{}
\fancyhead[L]{\small\sffamily \authors}
\fancyhead[R]{\small\sffamily \docversion}
\fancyfoot[C]{\thepage}
\renewcommand{\headrulewidth}{0.5pt}

% =============================================================================
% 사용법:
% 각 문서에서 다음과 같이 사용합니다:
%
% \documentclass[11pt, a4paper]{article}
% % =============================================================================
% GitHub Copilot 가이드 공통 프리앰블
% _preamble.tex - 모든 LaTeX 문서에서 % =============================================================================
% GitHub Copilot 가이드 공통 프리앰블
% _preamble.tex - 모든 LaTeX 문서에서 \input{_preamble} 으로 재사용
% =============================================================================

% --- DOCUMENT VERSION CONTROL ---
% PDF 빌드 시점의 날짜와 시간(KST, UTC+9)이 자동으로 반영됩니다
% 참고: Overleaf 서버는 CET(UTC+1) 시간대를 사용하므로 +8로 보정합니다
\usepackage{datetime2}
\ExplSyntaxOn
\int_new:N \g_kst_hour_int
\int_new:N \g_kst_day_int
\int_new:N \g_kst_month_int
\int_new:N \g_kst_year_int
\int_gset:Nn \g_kst_hour_int { \the\time / 60 + 9 }
\int_gset:Nn \g_kst_day_int { \the\day }
\int_gset:Nn \g_kst_month_int { \the\month }
\int_gset:Nn \g_kst_year_int { \the\year }
\int_compare:nNnT { \g_kst_hour_int } > { 23 }
  {
    \int_gadd:Nn \g_kst_hour_int { -24 }
    \int_gincr:N \g_kst_day_int
  }
\newcommand{\docversion}{최종~수정:~\int_use:N \g_kst_year_int.~\int_compare:nNnTF { \g_kst_month_int } < { 10 } {0}{} \int_use:N \g_kst_month_int.~\int_compare:nNnTF { \g_kst_day_int } < { 10 } {0}{} \int_use:N \g_kst_day_int.
  \c_space_tl \int_compare:nNnTF { \g_kst_hour_int } < { 10 } {0}{} \int_use:N \g_kst_hour_int
  : \int_compare:nNnTF { \int_mod:nn {\the\time} {60} } < { 10 } {0}{} \int_eval:n { \int_mod:nn {\the\time} {60} }}
\newcommand{\docdate}{\int_use:N \g_kst_year_int 년~\int_compare:nNnTF { \g_kst_month_int } < { 10 } {0}{} \int_use:N \g_kst_month_int 월~\int_compare:nNnTF { \g_kst_day_int } < { 10 } {0}{} \int_use:N \g_kst_day_int 일}
\ExplSyntaxOff

% --- 저자 정보 (각 문서에서 재정의 가능) ---
\providecommand{\authors}{동의대학교 서진석(jsseo@deu.ac.kr)}

% --- UNIVERSAL PREAMBLE BLOCK ---
\usepackage[a4paper, top=3cm, bottom=3cm, left=2.5cm, right=2.5cm, headheight=14pt]{geometry}
\usepackage{fontspec}

% Language setup for Korean
\usepackage[korean, provide=*]{babel}

% Fonts: Using Noto Serif for body (formal) and Noto Sans for headers
\babelfont{rm}{Noto Serif CJK KR}
\babelfont{sf}{Noto Sans CJK KR}

% CJK 폰트는 이탤릭이 없으므로 폰트 대체 규칙 설정 (경고 방지)
\DeclareFontShape{TU}{NotoSerifCJKKR(0)}{m}{it}{<->ssub*NotoSerifCJKKR(0)/m/n}{}
\DeclareFontShape{TU}{NotoSerifCJKKR(0)}{m}{sl}{<->ssub*NotoSerifCJKKR(0)/m/n}{}
\DeclareFontShape{TU}{NotoSerifCJKKR(0)}{bx}{it}{<->ssub*NotoSerifCJKKR(0)/bx/n}{}
\DeclareFontShape{TU}{NotoSansCJKKR(0)}{m}{it}{<->ssub*NotoSansCJKKR(0)/m/n}{}
\DeclareFontShape{TU}{NotoSansCJKKR(0)}{m}{sl}{<->ssub*NotoSansCJKKR(0)/m/n}{}
\DeclareFontShape{TU}{NotoSansCJKKR(0)}{bx}{it}{<->ssub*NotoSansCJKKR(0)/bx/n}{}

% Essential Packages
\usepackage{enumitem}
\usepackage{booktabs}   % Professional tables
\usepackage{tabularx}   % Auto-width tables
\usepackage{titlesec}   % Section formatting
\usepackage{xcolor}     % Colors for visual hierarchy
\usepackage{fancyhdr}   % Headers and footers
\usepackage{setspace}   % Line spacing
\usepackage{longtable}  % For tables that might span pages
\usepackage{xltabular}  % For flexible width long tables
\usepackage{float}      % For [H] option to fix table position
\usepackage{kotex}      % Additional Korean support insurance
\usepackage[hidelinks]{hyperref}

% --- STYLING COMMANDS ---

% Define a formal navy blue for headers
\definecolor{eduNavy}{RGB}{0, 43, 91}

% Section styling
\titleformat{\section}
  {\Large\sffamily\bfseries\color{eduNavy}}
  {\thesection}{1em}{}
  [\vspace{-0.5em}\hrule height 1pt]

\titleformat{\subsection}
  {\large\sffamily\bfseries\color{eduNavy}}
  {\thesubsection}{1em}{}

\titleformat{\subsubsection}
  {\normalsize\sffamily\bfseries\color{eduNavy}}
  {\thesubsubsection}{1em}{}

% List settings
\setlist[itemize]{label=\textbullet, leftmargin=1.5em, itemsep=0.2em}
\setlist[enumerate]{label=\arabic*., leftmargin=1.5em, itemsep=0.2em}

% Line spacing for readability
\setstretch{1.2}

% Header and Footer
\pagestyle{fancy}
\fancyhf{}
\fancyhead[L]{\small\sffamily \authors}
\fancyhead[R]{\small\sffamily \docversion}
\fancyfoot[C]{\thepage}
\renewcommand{\headrulewidth}{0.5pt}

% =============================================================================
% 사용법:
% 각 문서에서 다음과 같이 사용합니다:
%
% \documentclass[11pt, a4paper]{article}
% \input{_preamble}
%
% % 필요시 저자 재정의
% \renewcommand{\authors}{작성자명 (email@example.com)}
%
% % 제목 설정
% \title{...}
% \author{\authors}
% \date{\docversion}
%
% \begin{document}
% \maketitle
% ...
% \end{document}
% =============================================================================
 으로 재사용
% =============================================================================

% --- DOCUMENT VERSION CONTROL ---
% PDF 빌드 시점의 날짜와 시간(KST, UTC+9)이 자동으로 반영됩니다
% 참고: Overleaf 서버는 CET(UTC+1) 시간대를 사용하므로 +8로 보정합니다
\usepackage{datetime2}
\ExplSyntaxOn
\int_new:N \g_kst_hour_int
\int_new:N \g_kst_day_int
\int_new:N \g_kst_month_int
\int_new:N \g_kst_year_int
\int_gset:Nn \g_kst_hour_int { \the\time / 60 + 9 }
\int_gset:Nn \g_kst_day_int { \the\day }
\int_gset:Nn \g_kst_month_int { \the\month }
\int_gset:Nn \g_kst_year_int { \the\year }
\int_compare:nNnT { \g_kst_hour_int } > { 23 }
  {
    \int_gadd:Nn \g_kst_hour_int { -24 }
    \int_gincr:N \g_kst_day_int
  }
\newcommand{\docversion}{최종~수정:~\int_use:N \g_kst_year_int.~\int_compare:nNnTF { \g_kst_month_int } < { 10 } {0}{} \int_use:N \g_kst_month_int.~\int_compare:nNnTF { \g_kst_day_int } < { 10 } {0}{} \int_use:N \g_kst_day_int.
  \c_space_tl \int_compare:nNnTF { \g_kst_hour_int } < { 10 } {0}{} \int_use:N \g_kst_hour_int
  : \int_compare:nNnTF { \int_mod:nn {\the\time} {60} } < { 10 } {0}{} \int_eval:n { \int_mod:nn {\the\time} {60} }}
\newcommand{\docdate}{\int_use:N \g_kst_year_int 년~\int_compare:nNnTF { \g_kst_month_int } < { 10 } {0}{} \int_use:N \g_kst_month_int 월~\int_compare:nNnTF { \g_kst_day_int } < { 10 } {0}{} \int_use:N \g_kst_day_int 일}
\ExplSyntaxOff

% --- 저자 정보 (각 문서에서 재정의 가능) ---
\providecommand{\authors}{동의대학교 서진석(jsseo@deu.ac.kr)}

% --- UNIVERSAL PREAMBLE BLOCK ---
\usepackage[a4paper, top=3cm, bottom=3cm, left=2.5cm, right=2.5cm, headheight=14pt]{geometry}
\usepackage{fontspec}

% Language setup for Korean
\usepackage[korean, provide=*]{babel}

% Fonts: Using Noto Serif for body (formal) and Noto Sans for headers
\babelfont{rm}{Noto Serif CJK KR}
\babelfont{sf}{Noto Sans CJK KR}

% CJK 폰트는 이탤릭이 없으므로 폰트 대체 규칙 설정 (경고 방지)
\DeclareFontShape{TU}{NotoSerifCJKKR(0)}{m}{it}{<->ssub*NotoSerifCJKKR(0)/m/n}{}
\DeclareFontShape{TU}{NotoSerifCJKKR(0)}{m}{sl}{<->ssub*NotoSerifCJKKR(0)/m/n}{}
\DeclareFontShape{TU}{NotoSerifCJKKR(0)}{bx}{it}{<->ssub*NotoSerifCJKKR(0)/bx/n}{}
\DeclareFontShape{TU}{NotoSansCJKKR(0)}{m}{it}{<->ssub*NotoSansCJKKR(0)/m/n}{}
\DeclareFontShape{TU}{NotoSansCJKKR(0)}{m}{sl}{<->ssub*NotoSansCJKKR(0)/m/n}{}
\DeclareFontShape{TU}{NotoSansCJKKR(0)}{bx}{it}{<->ssub*NotoSansCJKKR(0)/bx/n}{}

% Essential Packages
\usepackage{enumitem}
\usepackage{booktabs}   % Professional tables
\usepackage{tabularx}   % Auto-width tables
\usepackage{titlesec}   % Section formatting
\usepackage{xcolor}     % Colors for visual hierarchy
\usepackage{fancyhdr}   % Headers and footers
\usepackage{setspace}   % Line spacing
\usepackage{longtable}  % For tables that might span pages
\usepackage{xltabular}  % For flexible width long tables
\usepackage{float}      % For [H] option to fix table position
\usepackage{kotex}      % Additional Korean support insurance
\usepackage[hidelinks]{hyperref}

% --- STYLING COMMANDS ---

% Define a formal navy blue for headers
\definecolor{eduNavy}{RGB}{0, 43, 91}

% Section styling
\titleformat{\section}
  {\Large\sffamily\bfseries\color{eduNavy}}
  {\thesection}{1em}{}
  [\vspace{-0.5em}\hrule height 1pt]

\titleformat{\subsection}
  {\large\sffamily\bfseries\color{eduNavy}}
  {\thesubsection}{1em}{}

\titleformat{\subsubsection}
  {\normalsize\sffamily\bfseries\color{eduNavy}}
  {\thesubsubsection}{1em}{}

% List settings
\setlist[itemize]{label=\textbullet, leftmargin=1.5em, itemsep=0.2em}
\setlist[enumerate]{label=\arabic*., leftmargin=1.5em, itemsep=0.2em}

% Line spacing for readability
\setstretch{1.2}

% Header and Footer
\pagestyle{fancy}
\fancyhf{}
\fancyhead[L]{\small\sffamily \authors}
\fancyhead[R]{\small\sffamily \docversion}
\fancyfoot[C]{\thepage}
\renewcommand{\headrulewidth}{0.5pt}

% =============================================================================
% 사용법:
% 각 문서에서 다음과 같이 사용합니다:
%
% \documentclass[11pt, a4paper]{article}
% % =============================================================================
% GitHub Copilot 가이드 공통 프리앰블
% _preamble.tex - 모든 LaTeX 문서에서 \input{_preamble} 으로 재사용
% =============================================================================

% --- DOCUMENT VERSION CONTROL ---
% PDF 빌드 시점의 날짜와 시간(KST, UTC+9)이 자동으로 반영됩니다
% 참고: Overleaf 서버는 CET(UTC+1) 시간대를 사용하므로 +8로 보정합니다
\usepackage{datetime2}
\ExplSyntaxOn
\int_new:N \g_kst_hour_int
\int_new:N \g_kst_day_int
\int_new:N \g_kst_month_int
\int_new:N \g_kst_year_int
\int_gset:Nn \g_kst_hour_int { \the\time / 60 + 9 }
\int_gset:Nn \g_kst_day_int { \the\day }
\int_gset:Nn \g_kst_month_int { \the\month }
\int_gset:Nn \g_kst_year_int { \the\year }
\int_compare:nNnT { \g_kst_hour_int } > { 23 }
  {
    \int_gadd:Nn \g_kst_hour_int { -24 }
    \int_gincr:N \g_kst_day_int
  }
\newcommand{\docversion}{최종~수정:~\int_use:N \g_kst_year_int.~\int_compare:nNnTF { \g_kst_month_int } < { 10 } {0}{} \int_use:N \g_kst_month_int.~\int_compare:nNnTF { \g_kst_day_int } < { 10 } {0}{} \int_use:N \g_kst_day_int.
  \c_space_tl \int_compare:nNnTF { \g_kst_hour_int } < { 10 } {0}{} \int_use:N \g_kst_hour_int
  : \int_compare:nNnTF { \int_mod:nn {\the\time} {60} } < { 10 } {0}{} \int_eval:n { \int_mod:nn {\the\time} {60} }}
\newcommand{\docdate}{\int_use:N \g_kst_year_int 년~\int_compare:nNnTF { \g_kst_month_int } < { 10 } {0}{} \int_use:N \g_kst_month_int 월~\int_compare:nNnTF { \g_kst_day_int } < { 10 } {0}{} \int_use:N \g_kst_day_int 일}
\ExplSyntaxOff

% --- 저자 정보 (각 문서에서 재정의 가능) ---
\providecommand{\authors}{동의대학교 서진석(jsseo@deu.ac.kr)}

% --- UNIVERSAL PREAMBLE BLOCK ---
\usepackage[a4paper, top=3cm, bottom=3cm, left=2.5cm, right=2.5cm, headheight=14pt]{geometry}
\usepackage{fontspec}

% Language setup for Korean
\usepackage[korean, provide=*]{babel}

% Fonts: Using Noto Serif for body (formal) and Noto Sans for headers
\babelfont{rm}{Noto Serif CJK KR}
\babelfont{sf}{Noto Sans CJK KR}

% CJK 폰트는 이탤릭이 없으므로 폰트 대체 규칙 설정 (경고 방지)
\DeclareFontShape{TU}{NotoSerifCJKKR(0)}{m}{it}{<->ssub*NotoSerifCJKKR(0)/m/n}{}
\DeclareFontShape{TU}{NotoSerifCJKKR(0)}{m}{sl}{<->ssub*NotoSerifCJKKR(0)/m/n}{}
\DeclareFontShape{TU}{NotoSerifCJKKR(0)}{bx}{it}{<->ssub*NotoSerifCJKKR(0)/bx/n}{}
\DeclareFontShape{TU}{NotoSansCJKKR(0)}{m}{it}{<->ssub*NotoSansCJKKR(0)/m/n}{}
\DeclareFontShape{TU}{NotoSansCJKKR(0)}{m}{sl}{<->ssub*NotoSansCJKKR(0)/m/n}{}
\DeclareFontShape{TU}{NotoSansCJKKR(0)}{bx}{it}{<->ssub*NotoSansCJKKR(0)/bx/n}{}

% Essential Packages
\usepackage{enumitem}
\usepackage{booktabs}   % Professional tables
\usepackage{tabularx}   % Auto-width tables
\usepackage{titlesec}   % Section formatting
\usepackage{xcolor}     % Colors for visual hierarchy
\usepackage{fancyhdr}   % Headers and footers
\usepackage{setspace}   % Line spacing
\usepackage{longtable}  % For tables that might span pages
\usepackage{xltabular}  % For flexible width long tables
\usepackage{float}      % For [H] option to fix table position
\usepackage{kotex}      % Additional Korean support insurance
\usepackage[hidelinks]{hyperref}

% --- STYLING COMMANDS ---

% Define a formal navy blue for headers
\definecolor{eduNavy}{RGB}{0, 43, 91}

% Section styling
\titleformat{\section}
  {\Large\sffamily\bfseries\color{eduNavy}}
  {\thesection}{1em}{}
  [\vspace{-0.5em}\hrule height 1pt]

\titleformat{\subsection}
  {\large\sffamily\bfseries\color{eduNavy}}
  {\thesubsection}{1em}{}

\titleformat{\subsubsection}
  {\normalsize\sffamily\bfseries\color{eduNavy}}
  {\thesubsubsection}{1em}{}

% List settings
\setlist[itemize]{label=\textbullet, leftmargin=1.5em, itemsep=0.2em}
\setlist[enumerate]{label=\arabic*., leftmargin=1.5em, itemsep=0.2em}

% Line spacing for readability
\setstretch{1.2}

% Header and Footer
\pagestyle{fancy}
\fancyhf{}
\fancyhead[L]{\small\sffamily \authors}
\fancyhead[R]{\small\sffamily \docversion}
\fancyfoot[C]{\thepage}
\renewcommand{\headrulewidth}{0.5pt}

% =============================================================================
% 사용법:
% 각 문서에서 다음과 같이 사용합니다:
%
% \documentclass[11pt, a4paper]{article}
% \input{_preamble}
%
% % 필요시 저자 재정의
% \renewcommand{\authors}{작성자명 (email@example.com)}
%
% % 제목 설정
% \title{...}
% \author{\authors}
% \date{\docversion}
%
% \begin{document}
% \maketitle
% ...
% \end{document}
% =============================================================================

%
% % 필요시 저자 재정의
% \renewcommand{\authors}{작성자명 (email@example.com)}
%
% % 제목 설정
% \title{...}
% \author{\authors}
% \date{\docversion}
%
% \begin{document}
% \maketitle
% ...
% \end{document}
% =============================================================================

%
% % 필요시 저자 재정의
% \renewcommand{\authors}{작성자명 (email@example.com)}
%
% % 제목 설정
% \title{...}
% \author{\authors}
% \date{\docversion}
%
% \begin{document}
% \maketitle
% ...
% \end{document}
% =============================================================================


% --- 문서별 메타데이터 ---
\renewcommand{\authors}{동의대학교 서진석(jsseo@deu.ac.kr)}

\title{\vspace{-2.5cm} \Huge\sffamily\bfseries GitHub Copilot 교수용 가이드 \\ \large\vspace{0.2em} \textmd{강의 및 연구를 위한 AI 활용법}\vspace{-0.5em}}
\author{\vspace{-0.5em}\authors}
\date{\vspace{-0.5em}\docversion}

\begin{document}

\maketitle
\thispagestyle{empty}
\vspace{-1.5cm}

\tableofcontents
\newpage

% ==============================================================================
\section{연구윤리 및 보안}
% ==============================================================================

\subsection{보안: 기업용 vs 무료 버전}
연구 데이터의 보안에 대해 걱정하시는 교수님들이 많습니다. \textbf{기업용(Enterprise) 버전}은 다릅니다.

\begin{table}[H]
    \centering
    \renewcommand{\arraystretch}{1.2}
    \begin{tabularx}{\textwidth}{@{}l X X@{}}
        \toprule
        \textbf{구분} & \textbf{무료 버전 (일반용)} & \textbf{기업용 버전 (Copilot)} \\
        \midrule
        데이터 학습 & 사용자 데이터를 학습에 사용 & \textbf{절대 학습하지 않음} \\
        데이터 처리 & 서버에 저장될 수 있음 & 암호화 처리 후 즉시 폐기 \\
        권장 & 민감 정보 입력 금지 & 연구 자료 입력 가능 \\
        \bottomrule
    \end{tabularx}
\end{table}

\subsection{AI 생성 콘텐츠와 학술적 정직성}

\subsubsection{절대 금지 사항}
\begin{itemize}
    \item \textbf{AI 생성 콘텐츠 그대로 제출 금지}: AI가 생성한 논문, 보고서, 연구 계획서를 본인 저작물로 제출하는 것은 \textbf{표절}입니다.
    \item \textbf{학생 과제 평가 시 주의}: AI로 작성 가능한 과제는 학생도 AI를 사용할 수 있습니다. 평가 방식을 재고하세요.
\end{itemize}

\subsubsection{권장 사용 방법}
\begin{itemize}
    \item \textbf{초안 작성 도우미}: AI로 아이디어를 브레인스토밍하고, 본인이 직접 검토·수정·보완
    \item \textbf{문법 및 표현 개선}: 본인이 작성한 텍스트의 문법 오류나 표현을 다듬는 용도로 활용
    \item \textbf{명시적 인용}: AI 도구를 사용했다면 각주나 방법론에 명시 \\
    (예: "본 문서는 GitHub Copilot을 보조 도구로 활용하여 작성되었습니다.")
\end{itemize}

\subsubsection{학생의 AI 사용 감지 FAQ}
\textbf{Q: "학생이 AI를 사용했는지 어떻게 알 수 있나요?"}

\begin{itemize}
    \item \textbf{AI 탐지 도구는 불완전}: GPTZero, Turnitin 등의 탐지 도구는 오탐(false positive)률이 높습니다.
    \item \textbf{대안: 과정 중심 평가}: 최종 결과물만이 아닌, 작성 과정(초안$\rightarrow$수정본)을 함께 제출받으세요.
    \item \textbf{구두 발표 추가}: 과제 내용에 대한 즉석 질문이나 발표를 통해 이해도를 확인하세요.
\end{itemize}

\subsection{학생 데이터 프라이버시 (개인정보 보호)}

\subsubsection{준수해야 할 법률}
\begin{itemize}
    \item \textbf{미국 FERPA}: 학생 교육 기록의 프라이버시 보호 (Family Educational Rights and Privacy Act)
    \item \textbf{한국 개인정보보호법}: 학생의 성명, 학번, 성적 등은 민감 정보로 분류
\end{itemize}

\subsubsection{안전한 사용법}
\begin{itemize}
    \item \textbf{익명화 필수}: 학생 이름 대신 "학생A", "학생B" 또는 "학번 마스킹(202X****)" 사용
    \item \textbf{성적 데이터 금지}: 개별 학생의 점수, 평가 내역은 AI에 입력하지 마세요
    \item \textbf{대안: 집합 데이터}: "30명 학생 중 15명이 80점 이상"과 같은 통계적 요약만 사용
\end{itemize}

% ==============================================================================
\section{기본 사용법}
% ==============================================================================

\subsection{설치 및 설정}
\begin{enumerate}
    \item \textbf{Visual Studio Code 설치}: \href{https://code.visualstudio.com}{code.visualstudio.com}에서 다운로드 및 설치
    \item \textbf{GitHub Copilot 확장 프로그램}: VS Code 왼쪽 `블록 아이콘(Extensions)' 클릭 $\rightarrow$ \texttt{GitHub Copilot} 검색 및 설치
    \item \textbf{로그인}: 좌측 하단 프로필 아이콘을 눌러 GitHub 계정으로 로그인
\end{enumerate}

\textit{※ 중요: 2025년 버전부터는 `Copilot Chat'이 통합되어 별도 설치가 필요 없습니다.}

\subsection{채팅 vs 자동완성}
\begin{table}[H]
    \centering
    \renewcommand{\arraystretch}{1.2}
    \begin{tabularx}{\textwidth}{@{}l l X@{}}
        \toprule
        \textbf{기능} & \textbf{단축키} & \textbf{용도} \\
        \midrule
        대화형 (Chat) & \texttt{Ctrl/Cmd + Alt + I} & 질문, 요약, 초안 생성, 복잡한 요청 \\
        자동완성 (Ghost Text) & \texttt{Tab} 키로 수락 & 다음 문장 예측, 표 자동 채우기, 반복 패턴 완성 \\
        \bottomrule
    \end{tabularx}
\end{table}

% ==============================================================================
\section{학문 분야별 활용}
% ==============================================================================

\subsection{인문학: 문학 작품 분석 루브릭}
\textbf{시나리오}: 세익스피어 희곡 분석 과제의 평가 기준표 작성

\textbf{프롬프트 예시}:
\begin{quote}
"저는 영문학과 교수입니다. 학부생 대상 `햄릿' 분석 과제에 대한 평가 루브릭을 작성해주세요.

평가 항목:
\begin{enumerate}
    \item 주제 분석 (30점) - 복수, 광기, 도덕적 딜레마
    \item 인물 분석 (25점) - 햄릿의 심리 변화
    \item 문학적 기법 (25점) - 독백, 상징, 비극적 결함
    \item 학술적 논증 (20점) - 논리성, 근거 제시
\end{enumerate}
각 항목별로 우수/보통/미흡 기준을 구체적으로 작성해주세요."
\end{quote}

\subsection{사회과학: 질적 연구 데이터 코딩}
\textbf{시나리오}: 인터뷰 녹취록에서 반복되는 주제 추출

\textbf{프롬프트 예시}:
\begin{quote}
"저는 사회학과 교수로 노인 고독에 관한 질적 연구를 진행 중입니다. 아래 3명의 인터뷰 녹취록에서 공통 주제(themes)를 추출하고, 각 주제별 대표 인용구(quotes)를 정리해주세요.

산출물 형식:
\begin{itemize}
    \item 주제 1: [제목]
    \item 설명: [2문장]
    \item 대표 인용: "..." (참가자 A)
\end{itemize}"
\end{quote}

\subsection{자연과학: 실험 보고서 구조화}
\textbf{시나리오}: 학생 제출 실험 보고서의 구조적 문제 진단

\textbf{프롬프트 예시}:
\begin{quote}
"저는 화학과 교수입니다. 학생이 제출한 적정 실험 보고서를 검토해주세요.

다음 관점에서 분석:
\begin{enumerate}
    \item IMRaD 구조 준수 여부 (Introduction-Methods-Results-Discussion)
    \item 실험 방법의 재현 가능성
    \item 데이터 표와 그래프의 적절성
    \item 오차 분석 및 한계점 언급
\end{enumerate}
각 항목에 대해 `잘함' / `개선 필요'로 평가하고, 개선이 필요한 부분은 구체적인 수정 제안을 해주세요."
\end{quote}

\subsection{공학: 설계 문서 검토}
\textbf{시나리오}: 캡스톤 프로젝트 설계 문서의 기술적 타당성 평가

\textbf{프롬프트 예시}:
\begin{quote}
"저는 전기공학과 교수입니다. 학생팀의 `스마트 에너지 관리 시스템' 설계서를 검토해주세요.

검토 항목:
\begin{enumerate}
    \item 시스템 아키텍처 (센서-마이크로컨트롤러-서버 연결)
    \item 전력 소모 계산의 정확성
    \item 보안 프로토콜 (데이터 암호화)
    \item 비용 대비 효과
    \item 제작 가능성 (부품 조달, 일정)
\end{enumerate}
각 항목별 점수(1-5점)와 개선 제안을 표 형식으로 정리해주세요."
\end{quote}

\subsection{예술: 작품 평가 기준 작성}
\textbf{시나리오}: 시각예술 전공 학생의 포트폴리오 평가 기준 마련

\textbf{평가 영역}:
\begin{enumerate}
    \item \textbf{독창성 (Originality)} - 30\%: 주제의 신선함, 표현 방식의 참신함
    \item \textbf{기술적 완성도 (Technical Proficiency)} - 25\%: 재료 활용, 표현 기법 숙련도
    \item \textbf{개념적 깊이 (Conceptual Depth)} - 25\%: 작가 노트와 작품의 연계성
    \item \textbf{전시 구성 (Exhibition Composition)} - 20\%: 작품 간 조화, 공간 활용
\end{enumerate}

% ==============================================================================
\section{교육학 프레임워크 적용}
% ==============================================================================

\subsection{Bloom's Taxonomy 기반 학습목표}
Bloom의 교육목표 분류법은 학습 목표를 6단계 인지적 영역으로 구분합니다.

\begin{table}[H]
    \centering
    \renewcommand{\arraystretch}{1.2}
    \small
    \begin{tabularx}{\textwidth}{@{}l l X@{}}
        \toprule
        \textbf{단계} & \textbf{영문} & \textbf{동사 예시} \\
        \midrule
        1. 기억하기 & Remember & 정의하다, 나열하다, 식별하다, 암기하다 \\
        2. 이해하기 & Understand & 설명하다, 요약하다, 분류하다, 비유하다 \\
        3. 적용하기 & Apply & 적용하다, 실행하다, 해결하다, 시연하다 \\
        4. 분석하기 & Analyze & 분석하다, 비교하다, 대조하다, 구분하다 \\
        5. 평가하기 & Evaluate & 평가하다, 비판하다, 판단하다, 정당화하다 \\
        6. 창조하기 & Create & 설계하다, 구성하다, 개발하다, 창조하다 \\
        \bottomrule
    \end{tabularx}
\end{table}

\textbf{Copilot 활용 프롬프트}:
\begin{quote}
"저는 교육심리학과 교수입니다. `학습 동기 이론' 3주차 강의의 학습 목표를 Bloom의 교육목표 분류법 기반으로 작성해주세요. 각 목표는 `학생은 \textasciitilde할 수 있다' 형식으로 시작하고, 측정 가능한(measurable) 동사를 사용해주세요."
\end{quote}

\subsection{Backward Design 강의 설계}
Backward Design은 최종 학습 성과에서 역으로 설계하는 3단계 접근법입니다.

\begin{enumerate}
    \item \textbf{Step 1 - 학습 성과 정의}: 학기말 학생이 무엇을 알아야 하나?
    \item \textbf{Step 2 - 평가 방법 설계}: 어떻게 성취를 확인할 것인가?
    \item \textbf{Step 3 - 학습 활동 계획}: 목표 달성을 위한 수업은?
\end{enumerate}

\subsection{Constructive Alignment 체크리스트}
Constructive Alignment는 학습 목표-교수 활동-평가 방법의 일치를 확인하는 원리입니다.

\textbf{검토 예시}:
\begin{quote}
"다음 강의 요소가 Constructive Alignment 원리에 맞게 정렬되었는지 검토하고, 개선점을 제안해주세요.
\begin{itemize}
    \item \textbf{학습 목표}: "학생은 SWOT 분석을 실제 기업에 적용할 수 있다."
    \item \textbf{교수 활동}: "SWOT 분석의 정의와 이론을 강의식으로 설명한다."
    \item \textbf{평가 방법}: "4지선다형 객관식 시험으로 SWOT의 정의를 묻는다."
\end{itemize}
분석 결과: 정렬 상태, 문제점, 개선안"
\end{quote}

\subsection{학술 인용 자동화 (APA/MLA)}
\textbf{APA 7판 프롬프트 예시}:
\begin{quote}
"아래 참고문헌을 APA 7판 형식으로 변환해주세요:
\begin{enumerate}
    \item 저자: Smith, J. / 출판년도: 2023 / 제목: The future of AI / 출판사: MIT Press
    \item 저자: Lee, K. \& Park, M. / 출판년도: 2022 / 논문 제목: Machine learning applications / 학술지: Journal of Education / 권: 45 / 호: 3 / 페이지: 120-135
\end{enumerate}"
\end{quote}

% ==============================================================================
\section{LMS 연계}
% ==============================================================================

\subsection{Canvas LMS 연계}
\textbf{SpeedGrader 워크플로우}:
\begin{enumerate}
    \item Canvas에서 과제 제출물 다운로드 (일괄 ZIP)
    \item VS Code에서 폴더 열기
    \item Copilot으로 각 파일 검토 및 피드백 초안 생성
    \item 생성된 피드백을 SpeedGrader에 붙여넣기
\end{enumerate}

\textbf{프롬프트 예시}:
\begin{quote}
"이 학생의 에세이에 대해 Canvas SpeedGrader용 피드백을 작성해주세요:
\begin{itemize}
    \item 강점 2가지
    \item 개선점 2가지
    \item 점수: /100점
\end{itemize}
어조는 격려하되 구체적으로 작성해주세요."
\end{quote}

\subsection{Blackboard 연계}
\textbf{루브릭 변환}:
\begin{quote}
"아래 평가 기준을 Blackboard 루브릭 임포트 형식(CSV)으로 변환해주세요:
\begin{itemize}
    \item 항목 1: 논증의 명확성 (30점)
    \item 항목 2: 근거 제시 (25점)
    \item 항목 3: 문법 및 구조 (20점)
    \item 항목 4: 참고문헌 인용 (25점)
\end{itemize}
각 항목별로 4단계 기준 포함"
\end{quote}

\subsection{Moodle 연계}
\textbf{XML 퀴즈 생성}:
\begin{quote}
"아래 내용을 Moodle XML 퀴즈 형식으로 변환해주세요:
\begin{itemize}
    \item 객관식 문제 5개 (4지선다)
    \item 진위형 문제 3개
    \item 단답형 문제 2개
\end{itemize}
주제: 교육심리학 - 동기 이론"
\end{quote}

% ==============================================================================
\section{부록}
% ==============================================================================

\subsection{학문 분야별 Instructions 템플릿}

\subsubsection{인문학 템플릿}
프로젝트 폴더에 \texttt{.github/copilot-instructions.md} 파일을 생성:
\begin{quote}
\small
\begin{verbatim}
# 영문학과 강의 AI 지침
## 1. 글쓰기 스타일
- 학술적 어조 유지
- MLA 9판 인용 형식 사용
- 비평 용어 정확히 사용

## 2. 분석 방법
- 텍스트 근거 기반 논증
- 역사적/문화적 맥락 고려
- 다양한 비평 관점 제시
\end{verbatim}
\end{quote}

\subsubsection{사회과학 템플릿}
\begin{quote}
\small
\begin{verbatim}
# 사회학과 강의 AI 지침
## 1. 연구 방법론
- APA 7판 인용 형식 사용
- 통계 용어 정확히 사용

## 2. 데이터 처리
- 학생 정보 익명화 필수
- 집합 데이터만 사용
\end{verbatim}
\end{quote}

\subsection{교육용 프롬프트 사전}
\begin{itemize}
    \item \textbf{강의계획서}: "이 강의의 16주차 일정표를 표 형식으로 작성해주세요. 각 주차별로 주제, 학습 목표, 과제를 포함해주세요."
    \item \textbf{평가 루브릭}: "이 과제에 대한 5단계 평가 루브릭을 작성해주세요. 각 항목별 점수 배점과 구체적인 기준을 포함해주세요."
    \item \textbf{피드백 작성}: "이 학생의 보고서에 대해 건설적인 피드백을 작성해주세요. 강점과 개선점을 균형 있게 포함해주세요."
    \item \textbf{퀴즈 생성}: "이 주제에 대한 객관식 문제 10개를 생성해주세요. 난이도는 중급으로 설정해주세요."
\end{itemize}

\subsection{단축키 요약}
\begin{table}[H]
    \centering
    \renewcommand{\arraystretch}{1.1}
    \begin{tabularx}{\textwidth}{@{}l X@{}}
        \toprule
        \textbf{단축키} & \textbf{기능} \\
        \midrule
        \texttt{Ctrl/Cmd + Alt + I} & Copilot 채팅창 열기/닫기 \\
        \texttt{Tab} & 자동완성 제안 수락 \\
        \texttt{Esc} & 자동완성 제안 무시 \\
        \texttt{\#} & 채팅창에서 파일 참조 \\
        \texttt{@} & 에이전트 호출 (@workspace, @terminal 등) \\
        \bottomrule
    \end{tabularx}
\end{table}

\vspace{1em}
\hrule
\vspace{0.5em}

\section*{유의 사항}
\small
본 문서는 작성 시점(\docdate) 기준이며, GitHub Copilot의 정책 및 기능은 변경될 수 있습니다. AI 도구 사용 시 연구윤리 및 개인정보보호법을 준수하십시오. 최신 정보는 공식 문서(\href{https://docs.github.com/en/copilot}{docs.github.com/copilot})를 참조하십시오.

\end{document}
