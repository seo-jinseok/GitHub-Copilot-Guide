\documentclass[11pt, a4paper]{article}
% =============================================================================
% GitHub Copilot 가이드 공통 프리앰블
% _preamble.tex - 모든 LaTeX 문서에서 % =============================================================================
% GitHub Copilot 가이드 공통 프리앰블
% _preamble.tex - 모든 LaTeX 문서에서 % =============================================================================
% GitHub Copilot 가이드 공통 프리앰블
% _preamble.tex - 모든 LaTeX 문서에서 \input{_preamble} 으로 재사용
% =============================================================================

% --- DOCUMENT VERSION CONTROL ---
% PDF 빌드 시점의 날짜와 시간(KST, UTC+9)이 자동으로 반영됩니다
% 참고: Overleaf 서버는 CET(UTC+1) 시간대를 사용하므로 +8로 보정합니다
\usepackage{datetime2}
\ExplSyntaxOn
\int_new:N \g_kst_hour_int
\int_new:N \g_kst_day_int
\int_new:N \g_kst_month_int
\int_new:N \g_kst_year_int
\int_gset:Nn \g_kst_hour_int { \the\time / 60 + 9 }
\int_gset:Nn \g_kst_day_int { \the\day }
\int_gset:Nn \g_kst_month_int { \the\month }
\int_gset:Nn \g_kst_year_int { \the\year }
\int_compare:nNnT { \g_kst_hour_int } > { 23 }
  {
    \int_gadd:Nn \g_kst_hour_int { -24 }
    \int_gincr:N \g_kst_day_int
  }
\newcommand{\docversion}{최종~수정:~\int_use:N \g_kst_year_int.~\int_compare:nNnTF { \g_kst_month_int } < { 10 } {0}{} \int_use:N \g_kst_month_int.~\int_compare:nNnTF { \g_kst_day_int } < { 10 } {0}{} \int_use:N \g_kst_day_int.
  \c_space_tl \int_compare:nNnTF { \g_kst_hour_int } < { 10 } {0}{} \int_use:N \g_kst_hour_int
  : \int_compare:nNnTF { \int_mod:nn {\the\time} {60} } < { 10 } {0}{} \int_eval:n { \int_mod:nn {\the\time} {60} }}
\newcommand{\docdate}{\int_use:N \g_kst_year_int 년~\int_compare:nNnTF { \g_kst_month_int } < { 10 } {0}{} \int_use:N \g_kst_month_int 월~\int_compare:nNnTF { \g_kst_day_int } < { 10 } {0}{} \int_use:N \g_kst_day_int 일}
\ExplSyntaxOff

% --- 저자 정보 (각 문서에서 재정의 가능) ---
\providecommand{\authors}{동의대학교 서진석(jsseo@deu.ac.kr)}

% --- UNIVERSAL PREAMBLE BLOCK ---
\usepackage[a4paper, top=3cm, bottom=3cm, left=2.5cm, right=2.5cm, headheight=14pt]{geometry}
\usepackage{fontspec}

% Language setup for Korean
\usepackage[korean, provide=*]{babel}

% Fonts: Using Noto Serif for body (formal) and Noto Sans for headers
\babelfont{rm}{Noto Serif CJK KR}
\babelfont{sf}{Noto Sans CJK KR}

% CJK 폰트는 이탤릭이 없으므로 폰트 대체 규칙 설정 (경고 방지)
\DeclareFontShape{TU}{NotoSerifCJKKR(0)}{m}{it}{<->ssub*NotoSerifCJKKR(0)/m/n}{}
\DeclareFontShape{TU}{NotoSerifCJKKR(0)}{m}{sl}{<->ssub*NotoSerifCJKKR(0)/m/n}{}
\DeclareFontShape{TU}{NotoSerifCJKKR(0)}{bx}{it}{<->ssub*NotoSerifCJKKR(0)/bx/n}{}
\DeclareFontShape{TU}{NotoSansCJKKR(0)}{m}{it}{<->ssub*NotoSansCJKKR(0)/m/n}{}
\DeclareFontShape{TU}{NotoSansCJKKR(0)}{m}{sl}{<->ssub*NotoSansCJKKR(0)/m/n}{}
\DeclareFontShape{TU}{NotoSansCJKKR(0)}{bx}{it}{<->ssub*NotoSansCJKKR(0)/bx/n}{}

% Essential Packages
\usepackage{enumitem}
\usepackage{booktabs}   % Professional tables
\usepackage{tabularx}   % Auto-width tables
\usepackage{titlesec}   % Section formatting
\usepackage{xcolor}     % Colors for visual hierarchy
\usepackage{fancyhdr}   % Headers and footers
\usepackage{setspace}   % Line spacing
\usepackage{longtable}  % For tables that might span pages
\usepackage{xltabular}  % For flexible width long tables
\usepackage{float}      % For [H] option to fix table position
\usepackage{kotex}      % Additional Korean support insurance
\usepackage[hidelinks]{hyperref}

% --- STYLING COMMANDS ---

% Define a formal navy blue for headers
\definecolor{eduNavy}{RGB}{0, 43, 91}

% Section styling
\titleformat{\section}
  {\Large\sffamily\bfseries\color{eduNavy}}
  {\thesection}{1em}{}
  [\vspace{-0.5em}\hrule height 1pt]

\titleformat{\subsection}
  {\large\sffamily\bfseries\color{eduNavy}}
  {\thesubsection}{1em}{}

\titleformat{\subsubsection}
  {\normalsize\sffamily\bfseries\color{eduNavy}}
  {\thesubsubsection}{1em}{}

% List settings
\setlist[itemize]{label=\textbullet, leftmargin=1.5em, itemsep=0.2em}
\setlist[enumerate]{label=\arabic*., leftmargin=1.5em, itemsep=0.2em}

% Line spacing for readability
\setstretch{1.2}

% Header and Footer
\pagestyle{fancy}
\fancyhf{}
\fancyhead[L]{\small\sffamily \authors}
\fancyhead[R]{\small\sffamily \docversion}
\fancyfoot[C]{\thepage}
\renewcommand{\headrulewidth}{0.5pt}

% =============================================================================
% 사용법:
% 각 문서에서 다음과 같이 사용합니다:
%
% \documentclass[11pt, a4paper]{article}
% \input{_preamble}
%
% % 필요시 저자 재정의
% \renewcommand{\authors}{작성자명 (email@example.com)}
%
% % 제목 설정
% \title{...}
% \author{\authors}
% \date{\docversion}
%
% \begin{document}
% \maketitle
% ...
% \end{document}
% =============================================================================
 으로 재사용
% =============================================================================

% --- DOCUMENT VERSION CONTROL ---
% PDF 빌드 시점의 날짜와 시간(KST, UTC+9)이 자동으로 반영됩니다
% 참고: Overleaf 서버는 CET(UTC+1) 시간대를 사용하므로 +8로 보정합니다
\usepackage{datetime2}
\ExplSyntaxOn
\int_new:N \g_kst_hour_int
\int_new:N \g_kst_day_int
\int_new:N \g_kst_month_int
\int_new:N \g_kst_year_int
\int_gset:Nn \g_kst_hour_int { \the\time / 60 + 9 }
\int_gset:Nn \g_kst_day_int { \the\day }
\int_gset:Nn \g_kst_month_int { \the\month }
\int_gset:Nn \g_kst_year_int { \the\year }
\int_compare:nNnT { \g_kst_hour_int } > { 23 }
  {
    \int_gadd:Nn \g_kst_hour_int { -24 }
    \int_gincr:N \g_kst_day_int
  }
\newcommand{\docversion}{최종~수정:~\int_use:N \g_kst_year_int.~\int_compare:nNnTF { \g_kst_month_int } < { 10 } {0}{} \int_use:N \g_kst_month_int.~\int_compare:nNnTF { \g_kst_day_int } < { 10 } {0}{} \int_use:N \g_kst_day_int.
  \c_space_tl \int_compare:nNnTF { \g_kst_hour_int } < { 10 } {0}{} \int_use:N \g_kst_hour_int
  : \int_compare:nNnTF { \int_mod:nn {\the\time} {60} } < { 10 } {0}{} \int_eval:n { \int_mod:nn {\the\time} {60} }}
\newcommand{\docdate}{\int_use:N \g_kst_year_int 년~\int_compare:nNnTF { \g_kst_month_int } < { 10 } {0}{} \int_use:N \g_kst_month_int 월~\int_compare:nNnTF { \g_kst_day_int } < { 10 } {0}{} \int_use:N \g_kst_day_int 일}
\ExplSyntaxOff

% --- 저자 정보 (각 문서에서 재정의 가능) ---
\providecommand{\authors}{동의대학교 서진석(jsseo@deu.ac.kr)}

% --- UNIVERSAL PREAMBLE BLOCK ---
\usepackage[a4paper, top=3cm, bottom=3cm, left=2.5cm, right=2.5cm, headheight=14pt]{geometry}
\usepackage{fontspec}

% Language setup for Korean
\usepackage[korean, provide=*]{babel}

% Fonts: Using Noto Serif for body (formal) and Noto Sans for headers
\babelfont{rm}{Noto Serif CJK KR}
\babelfont{sf}{Noto Sans CJK KR}

% CJK 폰트는 이탤릭이 없으므로 폰트 대체 규칙 설정 (경고 방지)
\DeclareFontShape{TU}{NotoSerifCJKKR(0)}{m}{it}{<->ssub*NotoSerifCJKKR(0)/m/n}{}
\DeclareFontShape{TU}{NotoSerifCJKKR(0)}{m}{sl}{<->ssub*NotoSerifCJKKR(0)/m/n}{}
\DeclareFontShape{TU}{NotoSerifCJKKR(0)}{bx}{it}{<->ssub*NotoSerifCJKKR(0)/bx/n}{}
\DeclareFontShape{TU}{NotoSansCJKKR(0)}{m}{it}{<->ssub*NotoSansCJKKR(0)/m/n}{}
\DeclareFontShape{TU}{NotoSansCJKKR(0)}{m}{sl}{<->ssub*NotoSansCJKKR(0)/m/n}{}
\DeclareFontShape{TU}{NotoSansCJKKR(0)}{bx}{it}{<->ssub*NotoSansCJKKR(0)/bx/n}{}

% Essential Packages
\usepackage{enumitem}
\usepackage{booktabs}   % Professional tables
\usepackage{tabularx}   % Auto-width tables
\usepackage{titlesec}   % Section formatting
\usepackage{xcolor}     % Colors for visual hierarchy
\usepackage{fancyhdr}   % Headers and footers
\usepackage{setspace}   % Line spacing
\usepackage{longtable}  % For tables that might span pages
\usepackage{xltabular}  % For flexible width long tables
\usepackage{float}      % For [H] option to fix table position
\usepackage{kotex}      % Additional Korean support insurance
\usepackage[hidelinks]{hyperref}

% --- STYLING COMMANDS ---

% Define a formal navy blue for headers
\definecolor{eduNavy}{RGB}{0, 43, 91}

% Section styling
\titleformat{\section}
  {\Large\sffamily\bfseries\color{eduNavy}}
  {\thesection}{1em}{}
  [\vspace{-0.5em}\hrule height 1pt]

\titleformat{\subsection}
  {\large\sffamily\bfseries\color{eduNavy}}
  {\thesubsection}{1em}{}

\titleformat{\subsubsection}
  {\normalsize\sffamily\bfseries\color{eduNavy}}
  {\thesubsubsection}{1em}{}

% List settings
\setlist[itemize]{label=\textbullet, leftmargin=1.5em, itemsep=0.2em}
\setlist[enumerate]{label=\arabic*., leftmargin=1.5em, itemsep=0.2em}

% Line spacing for readability
\setstretch{1.2}

% Header and Footer
\pagestyle{fancy}
\fancyhf{}
\fancyhead[L]{\small\sffamily \authors}
\fancyhead[R]{\small\sffamily \docversion}
\fancyfoot[C]{\thepage}
\renewcommand{\headrulewidth}{0.5pt}

% =============================================================================
% 사용법:
% 각 문서에서 다음과 같이 사용합니다:
%
% \documentclass[11pt, a4paper]{article}
% % =============================================================================
% GitHub Copilot 가이드 공통 프리앰블
% _preamble.tex - 모든 LaTeX 문서에서 \input{_preamble} 으로 재사용
% =============================================================================

% --- DOCUMENT VERSION CONTROL ---
% PDF 빌드 시점의 날짜와 시간(KST, UTC+9)이 자동으로 반영됩니다
% 참고: Overleaf 서버는 CET(UTC+1) 시간대를 사용하므로 +8로 보정합니다
\usepackage{datetime2}
\ExplSyntaxOn
\int_new:N \g_kst_hour_int
\int_new:N \g_kst_day_int
\int_new:N \g_kst_month_int
\int_new:N \g_kst_year_int
\int_gset:Nn \g_kst_hour_int { \the\time / 60 + 9 }
\int_gset:Nn \g_kst_day_int { \the\day }
\int_gset:Nn \g_kst_month_int { \the\month }
\int_gset:Nn \g_kst_year_int { \the\year }
\int_compare:nNnT { \g_kst_hour_int } > { 23 }
  {
    \int_gadd:Nn \g_kst_hour_int { -24 }
    \int_gincr:N \g_kst_day_int
  }
\newcommand{\docversion}{최종~수정:~\int_use:N \g_kst_year_int.~\int_compare:nNnTF { \g_kst_month_int } < { 10 } {0}{} \int_use:N \g_kst_month_int.~\int_compare:nNnTF { \g_kst_day_int } < { 10 } {0}{} \int_use:N \g_kst_day_int.
  \c_space_tl \int_compare:nNnTF { \g_kst_hour_int } < { 10 } {0}{} \int_use:N \g_kst_hour_int
  : \int_compare:nNnTF { \int_mod:nn {\the\time} {60} } < { 10 } {0}{} \int_eval:n { \int_mod:nn {\the\time} {60} }}
\newcommand{\docdate}{\int_use:N \g_kst_year_int 년~\int_compare:nNnTF { \g_kst_month_int } < { 10 } {0}{} \int_use:N \g_kst_month_int 월~\int_compare:nNnTF { \g_kst_day_int } < { 10 } {0}{} \int_use:N \g_kst_day_int 일}
\ExplSyntaxOff

% --- 저자 정보 (각 문서에서 재정의 가능) ---
\providecommand{\authors}{동의대학교 서진석(jsseo@deu.ac.kr)}

% --- UNIVERSAL PREAMBLE BLOCK ---
\usepackage[a4paper, top=3cm, bottom=3cm, left=2.5cm, right=2.5cm, headheight=14pt]{geometry}
\usepackage{fontspec}

% Language setup for Korean
\usepackage[korean, provide=*]{babel}

% Fonts: Using Noto Serif for body (formal) and Noto Sans for headers
\babelfont{rm}{Noto Serif CJK KR}
\babelfont{sf}{Noto Sans CJK KR}

% CJK 폰트는 이탤릭이 없으므로 폰트 대체 규칙 설정 (경고 방지)
\DeclareFontShape{TU}{NotoSerifCJKKR(0)}{m}{it}{<->ssub*NotoSerifCJKKR(0)/m/n}{}
\DeclareFontShape{TU}{NotoSerifCJKKR(0)}{m}{sl}{<->ssub*NotoSerifCJKKR(0)/m/n}{}
\DeclareFontShape{TU}{NotoSerifCJKKR(0)}{bx}{it}{<->ssub*NotoSerifCJKKR(0)/bx/n}{}
\DeclareFontShape{TU}{NotoSansCJKKR(0)}{m}{it}{<->ssub*NotoSansCJKKR(0)/m/n}{}
\DeclareFontShape{TU}{NotoSansCJKKR(0)}{m}{sl}{<->ssub*NotoSansCJKKR(0)/m/n}{}
\DeclareFontShape{TU}{NotoSansCJKKR(0)}{bx}{it}{<->ssub*NotoSansCJKKR(0)/bx/n}{}

% Essential Packages
\usepackage{enumitem}
\usepackage{booktabs}   % Professional tables
\usepackage{tabularx}   % Auto-width tables
\usepackage{titlesec}   % Section formatting
\usepackage{xcolor}     % Colors for visual hierarchy
\usepackage{fancyhdr}   % Headers and footers
\usepackage{setspace}   % Line spacing
\usepackage{longtable}  % For tables that might span pages
\usepackage{xltabular}  % For flexible width long tables
\usepackage{float}      % For [H] option to fix table position
\usepackage{kotex}      % Additional Korean support insurance
\usepackage[hidelinks]{hyperref}

% --- STYLING COMMANDS ---

% Define a formal navy blue for headers
\definecolor{eduNavy}{RGB}{0, 43, 91}

% Section styling
\titleformat{\section}
  {\Large\sffamily\bfseries\color{eduNavy}}
  {\thesection}{1em}{}
  [\vspace{-0.5em}\hrule height 1pt]

\titleformat{\subsection}
  {\large\sffamily\bfseries\color{eduNavy}}
  {\thesubsection}{1em}{}

\titleformat{\subsubsection}
  {\normalsize\sffamily\bfseries\color{eduNavy}}
  {\thesubsubsection}{1em}{}

% List settings
\setlist[itemize]{label=\textbullet, leftmargin=1.5em, itemsep=0.2em}
\setlist[enumerate]{label=\arabic*., leftmargin=1.5em, itemsep=0.2em}

% Line spacing for readability
\setstretch{1.2}

% Header and Footer
\pagestyle{fancy}
\fancyhf{}
\fancyhead[L]{\small\sffamily \authors}
\fancyhead[R]{\small\sffamily \docversion}
\fancyfoot[C]{\thepage}
\renewcommand{\headrulewidth}{0.5pt}

% =============================================================================
% 사용법:
% 각 문서에서 다음과 같이 사용합니다:
%
% \documentclass[11pt, a4paper]{article}
% \input{_preamble}
%
% % 필요시 저자 재정의
% \renewcommand{\authors}{작성자명 (email@example.com)}
%
% % 제목 설정
% \title{...}
% \author{\authors}
% \date{\docversion}
%
% \begin{document}
% \maketitle
% ...
% \end{document}
% =============================================================================

%
% % 필요시 저자 재정의
% \renewcommand{\authors}{작성자명 (email@example.com)}
%
% % 제목 설정
% \title{...}
% \author{\authors}
% \date{\docversion}
%
% \begin{document}
% \maketitle
% ...
% \end{document}
% =============================================================================
 으로 재사용
% =============================================================================

% --- DOCUMENT VERSION CONTROL ---
% PDF 빌드 시점의 날짜와 시간(KST, UTC+9)이 자동으로 반영됩니다
% 참고: Overleaf 서버는 CET(UTC+1) 시간대를 사용하므로 +8로 보정합니다
\usepackage{datetime2}
\ExplSyntaxOn
\int_new:N \g_kst_hour_int
\int_new:N \g_kst_day_int
\int_new:N \g_kst_month_int
\int_new:N \g_kst_year_int
\int_gset:Nn \g_kst_hour_int { \the\time / 60 + 9 }
\int_gset:Nn \g_kst_day_int { \the\day }
\int_gset:Nn \g_kst_month_int { \the\month }
\int_gset:Nn \g_kst_year_int { \the\year }
\int_compare:nNnT { \g_kst_hour_int } > { 23 }
  {
    \int_gadd:Nn \g_kst_hour_int { -24 }
    \int_gincr:N \g_kst_day_int
  }
\newcommand{\docversion}{최종~수정:~\int_use:N \g_kst_year_int.~\int_compare:nNnTF { \g_kst_month_int } < { 10 } {0}{} \int_use:N \g_kst_month_int.~\int_compare:nNnTF { \g_kst_day_int } < { 10 } {0}{} \int_use:N \g_kst_day_int.
  \c_space_tl \int_compare:nNnTF { \g_kst_hour_int } < { 10 } {0}{} \int_use:N \g_kst_hour_int
  : \int_compare:nNnTF { \int_mod:nn {\the\time} {60} } < { 10 } {0}{} \int_eval:n { \int_mod:nn {\the\time} {60} }}
\newcommand{\docdate}{\int_use:N \g_kst_year_int 년~\int_compare:nNnTF { \g_kst_month_int } < { 10 } {0}{} \int_use:N \g_kst_month_int 월~\int_compare:nNnTF { \g_kst_day_int } < { 10 } {0}{} \int_use:N \g_kst_day_int 일}
\ExplSyntaxOff

% --- 저자 정보 (각 문서에서 재정의 가능) ---
\providecommand{\authors}{동의대학교 서진석(jsseo@deu.ac.kr)}

% --- UNIVERSAL PREAMBLE BLOCK ---
\usepackage[a4paper, top=3cm, bottom=3cm, left=2.5cm, right=2.5cm, headheight=14pt]{geometry}
\usepackage{fontspec}

% Language setup for Korean
\usepackage[korean, provide=*]{babel}

% Fonts: Using Noto Serif for body (formal) and Noto Sans for headers
\babelfont{rm}{Noto Serif CJK KR}
\babelfont{sf}{Noto Sans CJK KR}

% CJK 폰트는 이탤릭이 없으므로 폰트 대체 규칙 설정 (경고 방지)
\DeclareFontShape{TU}{NotoSerifCJKKR(0)}{m}{it}{<->ssub*NotoSerifCJKKR(0)/m/n}{}
\DeclareFontShape{TU}{NotoSerifCJKKR(0)}{m}{sl}{<->ssub*NotoSerifCJKKR(0)/m/n}{}
\DeclareFontShape{TU}{NotoSerifCJKKR(0)}{bx}{it}{<->ssub*NotoSerifCJKKR(0)/bx/n}{}
\DeclareFontShape{TU}{NotoSansCJKKR(0)}{m}{it}{<->ssub*NotoSansCJKKR(0)/m/n}{}
\DeclareFontShape{TU}{NotoSansCJKKR(0)}{m}{sl}{<->ssub*NotoSansCJKKR(0)/m/n}{}
\DeclareFontShape{TU}{NotoSansCJKKR(0)}{bx}{it}{<->ssub*NotoSansCJKKR(0)/bx/n}{}

% Essential Packages
\usepackage{enumitem}
\usepackage{booktabs}   % Professional tables
\usepackage{tabularx}   % Auto-width tables
\usepackage{titlesec}   % Section formatting
\usepackage{xcolor}     % Colors for visual hierarchy
\usepackage{fancyhdr}   % Headers and footers
\usepackage{setspace}   % Line spacing
\usepackage{longtable}  % For tables that might span pages
\usepackage{xltabular}  % For flexible width long tables
\usepackage{float}      % For [H] option to fix table position
\usepackage{kotex}      % Additional Korean support insurance
\usepackage[hidelinks]{hyperref}

% --- STYLING COMMANDS ---

% Define a formal navy blue for headers
\definecolor{eduNavy}{RGB}{0, 43, 91}

% Section styling
\titleformat{\section}
  {\Large\sffamily\bfseries\color{eduNavy}}
  {\thesection}{1em}{}
  [\vspace{-0.5em}\hrule height 1pt]

\titleformat{\subsection}
  {\large\sffamily\bfseries\color{eduNavy}}
  {\thesubsection}{1em}{}

\titleformat{\subsubsection}
  {\normalsize\sffamily\bfseries\color{eduNavy}}
  {\thesubsubsection}{1em}{}

% List settings
\setlist[itemize]{label=\textbullet, leftmargin=1.5em, itemsep=0.2em}
\setlist[enumerate]{label=\arabic*., leftmargin=1.5em, itemsep=0.2em}

% Line spacing for readability
\setstretch{1.2}

% Header and Footer
\pagestyle{fancy}
\fancyhf{}
\fancyhead[L]{\small\sffamily \authors}
\fancyhead[R]{\small\sffamily \docversion}
\fancyfoot[C]{\thepage}
\renewcommand{\headrulewidth}{0.5pt}

% =============================================================================
% 사용법:
% 각 문서에서 다음과 같이 사용합니다:
%
% \documentclass[11pt, a4paper]{article}
% % =============================================================================
% GitHub Copilot 가이드 공통 프리앰블
% _preamble.tex - 모든 LaTeX 문서에서 % =============================================================================
% GitHub Copilot 가이드 공통 프리앰블
% _preamble.tex - 모든 LaTeX 문서에서 \input{_preamble} 으로 재사용
% =============================================================================

% --- DOCUMENT VERSION CONTROL ---
% PDF 빌드 시점의 날짜와 시간(KST, UTC+9)이 자동으로 반영됩니다
% 참고: Overleaf 서버는 CET(UTC+1) 시간대를 사용하므로 +8로 보정합니다
\usepackage{datetime2}
\ExplSyntaxOn
\int_new:N \g_kst_hour_int
\int_new:N \g_kst_day_int
\int_new:N \g_kst_month_int
\int_new:N \g_kst_year_int
\int_gset:Nn \g_kst_hour_int { \the\time / 60 + 9 }
\int_gset:Nn \g_kst_day_int { \the\day }
\int_gset:Nn \g_kst_month_int { \the\month }
\int_gset:Nn \g_kst_year_int { \the\year }
\int_compare:nNnT { \g_kst_hour_int } > { 23 }
  {
    \int_gadd:Nn \g_kst_hour_int { -24 }
    \int_gincr:N \g_kst_day_int
  }
\newcommand{\docversion}{최종~수정:~\int_use:N \g_kst_year_int.~\int_compare:nNnTF { \g_kst_month_int } < { 10 } {0}{} \int_use:N \g_kst_month_int.~\int_compare:nNnTF { \g_kst_day_int } < { 10 } {0}{} \int_use:N \g_kst_day_int.
  \c_space_tl \int_compare:nNnTF { \g_kst_hour_int } < { 10 } {0}{} \int_use:N \g_kst_hour_int
  : \int_compare:nNnTF { \int_mod:nn {\the\time} {60} } < { 10 } {0}{} \int_eval:n { \int_mod:nn {\the\time} {60} }}
\newcommand{\docdate}{\int_use:N \g_kst_year_int 년~\int_compare:nNnTF { \g_kst_month_int } < { 10 } {0}{} \int_use:N \g_kst_month_int 월~\int_compare:nNnTF { \g_kst_day_int } < { 10 } {0}{} \int_use:N \g_kst_day_int 일}
\ExplSyntaxOff

% --- 저자 정보 (각 문서에서 재정의 가능) ---
\providecommand{\authors}{동의대학교 서진석(jsseo@deu.ac.kr)}

% --- UNIVERSAL PREAMBLE BLOCK ---
\usepackage[a4paper, top=3cm, bottom=3cm, left=2.5cm, right=2.5cm, headheight=14pt]{geometry}
\usepackage{fontspec}

% Language setup for Korean
\usepackage[korean, provide=*]{babel}

% Fonts: Using Noto Serif for body (formal) and Noto Sans for headers
\babelfont{rm}{Noto Serif CJK KR}
\babelfont{sf}{Noto Sans CJK KR}

% CJK 폰트는 이탤릭이 없으므로 폰트 대체 규칙 설정 (경고 방지)
\DeclareFontShape{TU}{NotoSerifCJKKR(0)}{m}{it}{<->ssub*NotoSerifCJKKR(0)/m/n}{}
\DeclareFontShape{TU}{NotoSerifCJKKR(0)}{m}{sl}{<->ssub*NotoSerifCJKKR(0)/m/n}{}
\DeclareFontShape{TU}{NotoSerifCJKKR(0)}{bx}{it}{<->ssub*NotoSerifCJKKR(0)/bx/n}{}
\DeclareFontShape{TU}{NotoSansCJKKR(0)}{m}{it}{<->ssub*NotoSansCJKKR(0)/m/n}{}
\DeclareFontShape{TU}{NotoSansCJKKR(0)}{m}{sl}{<->ssub*NotoSansCJKKR(0)/m/n}{}
\DeclareFontShape{TU}{NotoSansCJKKR(0)}{bx}{it}{<->ssub*NotoSansCJKKR(0)/bx/n}{}

% Essential Packages
\usepackage{enumitem}
\usepackage{booktabs}   % Professional tables
\usepackage{tabularx}   % Auto-width tables
\usepackage{titlesec}   % Section formatting
\usepackage{xcolor}     % Colors for visual hierarchy
\usepackage{fancyhdr}   % Headers and footers
\usepackage{setspace}   % Line spacing
\usepackage{longtable}  % For tables that might span pages
\usepackage{xltabular}  % For flexible width long tables
\usepackage{float}      % For [H] option to fix table position
\usepackage{kotex}      % Additional Korean support insurance
\usepackage[hidelinks]{hyperref}

% --- STYLING COMMANDS ---

% Define a formal navy blue for headers
\definecolor{eduNavy}{RGB}{0, 43, 91}

% Section styling
\titleformat{\section}
  {\Large\sffamily\bfseries\color{eduNavy}}
  {\thesection}{1em}{}
  [\vspace{-0.5em}\hrule height 1pt]

\titleformat{\subsection}
  {\large\sffamily\bfseries\color{eduNavy}}
  {\thesubsection}{1em}{}

\titleformat{\subsubsection}
  {\normalsize\sffamily\bfseries\color{eduNavy}}
  {\thesubsubsection}{1em}{}

% List settings
\setlist[itemize]{label=\textbullet, leftmargin=1.5em, itemsep=0.2em}
\setlist[enumerate]{label=\arabic*., leftmargin=1.5em, itemsep=0.2em}

% Line spacing for readability
\setstretch{1.2}

% Header and Footer
\pagestyle{fancy}
\fancyhf{}
\fancyhead[L]{\small\sffamily \authors}
\fancyhead[R]{\small\sffamily \docversion}
\fancyfoot[C]{\thepage}
\renewcommand{\headrulewidth}{0.5pt}

% =============================================================================
% 사용법:
% 각 문서에서 다음과 같이 사용합니다:
%
% \documentclass[11pt, a4paper]{article}
% \input{_preamble}
%
% % 필요시 저자 재정의
% \renewcommand{\authors}{작성자명 (email@example.com)}
%
% % 제목 설정
% \title{...}
% \author{\authors}
% \date{\docversion}
%
% \begin{document}
% \maketitle
% ...
% \end{document}
% =============================================================================
 으로 재사용
% =============================================================================

% --- DOCUMENT VERSION CONTROL ---
% PDF 빌드 시점의 날짜와 시간(KST, UTC+9)이 자동으로 반영됩니다
% 참고: Overleaf 서버는 CET(UTC+1) 시간대를 사용하므로 +8로 보정합니다
\usepackage{datetime2}
\ExplSyntaxOn
\int_new:N \g_kst_hour_int
\int_new:N \g_kst_day_int
\int_new:N \g_kst_month_int
\int_new:N \g_kst_year_int
\int_gset:Nn \g_kst_hour_int { \the\time / 60 + 9 }
\int_gset:Nn \g_kst_day_int { \the\day }
\int_gset:Nn \g_kst_month_int { \the\month }
\int_gset:Nn \g_kst_year_int { \the\year }
\int_compare:nNnT { \g_kst_hour_int } > { 23 }
  {
    \int_gadd:Nn \g_kst_hour_int { -24 }
    \int_gincr:N \g_kst_day_int
  }
\newcommand{\docversion}{최종~수정:~\int_use:N \g_kst_year_int.~\int_compare:nNnTF { \g_kst_month_int } < { 10 } {0}{} \int_use:N \g_kst_month_int.~\int_compare:nNnTF { \g_kst_day_int } < { 10 } {0}{} \int_use:N \g_kst_day_int.
  \c_space_tl \int_compare:nNnTF { \g_kst_hour_int } < { 10 } {0}{} \int_use:N \g_kst_hour_int
  : \int_compare:nNnTF { \int_mod:nn {\the\time} {60} } < { 10 } {0}{} \int_eval:n { \int_mod:nn {\the\time} {60} }}
\newcommand{\docdate}{\int_use:N \g_kst_year_int 년~\int_compare:nNnTF { \g_kst_month_int } < { 10 } {0}{} \int_use:N \g_kst_month_int 월~\int_compare:nNnTF { \g_kst_day_int } < { 10 } {0}{} \int_use:N \g_kst_day_int 일}
\ExplSyntaxOff

% --- 저자 정보 (각 문서에서 재정의 가능) ---
\providecommand{\authors}{동의대학교 서진석(jsseo@deu.ac.kr)}

% --- UNIVERSAL PREAMBLE BLOCK ---
\usepackage[a4paper, top=3cm, bottom=3cm, left=2.5cm, right=2.5cm, headheight=14pt]{geometry}
\usepackage{fontspec}

% Language setup for Korean
\usepackage[korean, provide=*]{babel}

% Fonts: Using Noto Serif for body (formal) and Noto Sans for headers
\babelfont{rm}{Noto Serif CJK KR}
\babelfont{sf}{Noto Sans CJK KR}

% CJK 폰트는 이탤릭이 없으므로 폰트 대체 규칙 설정 (경고 방지)
\DeclareFontShape{TU}{NotoSerifCJKKR(0)}{m}{it}{<->ssub*NotoSerifCJKKR(0)/m/n}{}
\DeclareFontShape{TU}{NotoSerifCJKKR(0)}{m}{sl}{<->ssub*NotoSerifCJKKR(0)/m/n}{}
\DeclareFontShape{TU}{NotoSerifCJKKR(0)}{bx}{it}{<->ssub*NotoSerifCJKKR(0)/bx/n}{}
\DeclareFontShape{TU}{NotoSansCJKKR(0)}{m}{it}{<->ssub*NotoSansCJKKR(0)/m/n}{}
\DeclareFontShape{TU}{NotoSansCJKKR(0)}{m}{sl}{<->ssub*NotoSansCJKKR(0)/m/n}{}
\DeclareFontShape{TU}{NotoSansCJKKR(0)}{bx}{it}{<->ssub*NotoSansCJKKR(0)/bx/n}{}

% Essential Packages
\usepackage{enumitem}
\usepackage{booktabs}   % Professional tables
\usepackage{tabularx}   % Auto-width tables
\usepackage{titlesec}   % Section formatting
\usepackage{xcolor}     % Colors for visual hierarchy
\usepackage{fancyhdr}   % Headers and footers
\usepackage{setspace}   % Line spacing
\usepackage{longtable}  % For tables that might span pages
\usepackage{xltabular}  % For flexible width long tables
\usepackage{float}      % For [H] option to fix table position
\usepackage{kotex}      % Additional Korean support insurance
\usepackage[hidelinks]{hyperref}

% --- STYLING COMMANDS ---

% Define a formal navy blue for headers
\definecolor{eduNavy}{RGB}{0, 43, 91}

% Section styling
\titleformat{\section}
  {\Large\sffamily\bfseries\color{eduNavy}}
  {\thesection}{1em}{}
  [\vspace{-0.5em}\hrule height 1pt]

\titleformat{\subsection}
  {\large\sffamily\bfseries\color{eduNavy}}
  {\thesubsection}{1em}{}

\titleformat{\subsubsection}
  {\normalsize\sffamily\bfseries\color{eduNavy}}
  {\thesubsubsection}{1em}{}

% List settings
\setlist[itemize]{label=\textbullet, leftmargin=1.5em, itemsep=0.2em}
\setlist[enumerate]{label=\arabic*., leftmargin=1.5em, itemsep=0.2em}

% Line spacing for readability
\setstretch{1.2}

% Header and Footer
\pagestyle{fancy}
\fancyhf{}
\fancyhead[L]{\small\sffamily \authors}
\fancyhead[R]{\small\sffamily \docversion}
\fancyfoot[C]{\thepage}
\renewcommand{\headrulewidth}{0.5pt}

% =============================================================================
% 사용법:
% 각 문서에서 다음과 같이 사용합니다:
%
% \documentclass[11pt, a4paper]{article}
% % =============================================================================
% GitHub Copilot 가이드 공통 프리앰블
% _preamble.tex - 모든 LaTeX 문서에서 \input{_preamble} 으로 재사용
% =============================================================================

% --- DOCUMENT VERSION CONTROL ---
% PDF 빌드 시점의 날짜와 시간(KST, UTC+9)이 자동으로 반영됩니다
% 참고: Overleaf 서버는 CET(UTC+1) 시간대를 사용하므로 +8로 보정합니다
\usepackage{datetime2}
\ExplSyntaxOn
\int_new:N \g_kst_hour_int
\int_new:N \g_kst_day_int
\int_new:N \g_kst_month_int
\int_new:N \g_kst_year_int
\int_gset:Nn \g_kst_hour_int { \the\time / 60 + 9 }
\int_gset:Nn \g_kst_day_int { \the\day }
\int_gset:Nn \g_kst_month_int { \the\month }
\int_gset:Nn \g_kst_year_int { \the\year }
\int_compare:nNnT { \g_kst_hour_int } > { 23 }
  {
    \int_gadd:Nn \g_kst_hour_int { -24 }
    \int_gincr:N \g_kst_day_int
  }
\newcommand{\docversion}{최종~수정:~\int_use:N \g_kst_year_int.~\int_compare:nNnTF { \g_kst_month_int } < { 10 } {0}{} \int_use:N \g_kst_month_int.~\int_compare:nNnTF { \g_kst_day_int } < { 10 } {0}{} \int_use:N \g_kst_day_int.
  \c_space_tl \int_compare:nNnTF { \g_kst_hour_int } < { 10 } {0}{} \int_use:N \g_kst_hour_int
  : \int_compare:nNnTF { \int_mod:nn {\the\time} {60} } < { 10 } {0}{} \int_eval:n { \int_mod:nn {\the\time} {60} }}
\newcommand{\docdate}{\int_use:N \g_kst_year_int 년~\int_compare:nNnTF { \g_kst_month_int } < { 10 } {0}{} \int_use:N \g_kst_month_int 월~\int_compare:nNnTF { \g_kst_day_int } < { 10 } {0}{} \int_use:N \g_kst_day_int 일}
\ExplSyntaxOff

% --- 저자 정보 (각 문서에서 재정의 가능) ---
\providecommand{\authors}{동의대학교 서진석(jsseo@deu.ac.kr)}

% --- UNIVERSAL PREAMBLE BLOCK ---
\usepackage[a4paper, top=3cm, bottom=3cm, left=2.5cm, right=2.5cm, headheight=14pt]{geometry}
\usepackage{fontspec}

% Language setup for Korean
\usepackage[korean, provide=*]{babel}

% Fonts: Using Noto Serif for body (formal) and Noto Sans for headers
\babelfont{rm}{Noto Serif CJK KR}
\babelfont{sf}{Noto Sans CJK KR}

% CJK 폰트는 이탤릭이 없으므로 폰트 대체 규칙 설정 (경고 방지)
\DeclareFontShape{TU}{NotoSerifCJKKR(0)}{m}{it}{<->ssub*NotoSerifCJKKR(0)/m/n}{}
\DeclareFontShape{TU}{NotoSerifCJKKR(0)}{m}{sl}{<->ssub*NotoSerifCJKKR(0)/m/n}{}
\DeclareFontShape{TU}{NotoSerifCJKKR(0)}{bx}{it}{<->ssub*NotoSerifCJKKR(0)/bx/n}{}
\DeclareFontShape{TU}{NotoSansCJKKR(0)}{m}{it}{<->ssub*NotoSansCJKKR(0)/m/n}{}
\DeclareFontShape{TU}{NotoSansCJKKR(0)}{m}{sl}{<->ssub*NotoSansCJKKR(0)/m/n}{}
\DeclareFontShape{TU}{NotoSansCJKKR(0)}{bx}{it}{<->ssub*NotoSansCJKKR(0)/bx/n}{}

% Essential Packages
\usepackage{enumitem}
\usepackage{booktabs}   % Professional tables
\usepackage{tabularx}   % Auto-width tables
\usepackage{titlesec}   % Section formatting
\usepackage{xcolor}     % Colors for visual hierarchy
\usepackage{fancyhdr}   % Headers and footers
\usepackage{setspace}   % Line spacing
\usepackage{longtable}  % For tables that might span pages
\usepackage{xltabular}  % For flexible width long tables
\usepackage{float}      % For [H] option to fix table position
\usepackage{kotex}      % Additional Korean support insurance
\usepackage[hidelinks]{hyperref}

% --- STYLING COMMANDS ---

% Define a formal navy blue for headers
\definecolor{eduNavy}{RGB}{0, 43, 91}

% Section styling
\titleformat{\section}
  {\Large\sffamily\bfseries\color{eduNavy}}
  {\thesection}{1em}{}
  [\vspace{-0.5em}\hrule height 1pt]

\titleformat{\subsection}
  {\large\sffamily\bfseries\color{eduNavy}}
  {\thesubsection}{1em}{}

\titleformat{\subsubsection}
  {\normalsize\sffamily\bfseries\color{eduNavy}}
  {\thesubsubsection}{1em}{}

% List settings
\setlist[itemize]{label=\textbullet, leftmargin=1.5em, itemsep=0.2em}
\setlist[enumerate]{label=\arabic*., leftmargin=1.5em, itemsep=0.2em}

% Line spacing for readability
\setstretch{1.2}

% Header and Footer
\pagestyle{fancy}
\fancyhf{}
\fancyhead[L]{\small\sffamily \authors}
\fancyhead[R]{\small\sffamily \docversion}
\fancyfoot[C]{\thepage}
\renewcommand{\headrulewidth}{0.5pt}

% =============================================================================
% 사용법:
% 각 문서에서 다음과 같이 사용합니다:
%
% \documentclass[11pt, a4paper]{article}
% \input{_preamble}
%
% % 필요시 저자 재정의
% \renewcommand{\authors}{작성자명 (email@example.com)}
%
% % 제목 설정
% \title{...}
% \author{\authors}
% \date{\docversion}
%
% \begin{document}
% \maketitle
% ...
% \end{document}
% =============================================================================

%
% % 필요시 저자 재정의
% \renewcommand{\authors}{작성자명 (email@example.com)}
%
% % 제목 설정
% \title{...}
% \author{\authors}
% \date{\docversion}
%
% \begin{document}
% \maketitle
% ...
% \end{document}
% =============================================================================

%
% % 필요시 저자 재정의
% \renewcommand{\authors}{작성자명 (email@example.com)}
%
% % 제목 설정
% \title{...}
% \author{\authors}
% \date{\docversion}
%
% \begin{document}
% \maketitle
% ...
% \end{document}
% =============================================================================


% --- 문서별 메타데이터 ---
\renewcommand{\authors}{동의대학교 서진석(jsseo@deu.ac.kr)}

\title{\vspace{-2.5cm} \Huge\sffamily\bfseries GitHub Copilot 연구원용 가이드 \\ \large\vspace{0.2em} \textmd{연구 및 논문 작성을 위한 AI 활용법}\vspace{-0.5em}}
\author{\vspace{-0.5em}\authors}
\date{\vspace{-0.5em}\docversion}

\begin{document}

\maketitle
\thispagestyle{empty}
\vspace{-1.5cm}

\tableofcontents
\newpage

% ==============================================================================
\section{연구 환경 설정}
% ==============================================================================

\subsection{GitHub Education 자격 요건}
GitHub Copilot을 \textbf{무료로 사용}하려면 다음 조건 중 하나를 충족해야 합니다:

\begin{table}[H]
    \centering
    \renewcommand{\arraystretch}{1.3}
    \begin{tabularx}{\textwidth}{@{}l X l@{}}
        \toprule
        \textbf{신분} & \textbf{조건} & \textbf{무료 여부} \\
        \midrule
        대학원생 & 재학 중인 석/박사 과정 학생 & ✅ 무료 \\
        교수/연구교수 & 대학 소속 교원 & ✅ 무료 \\
        박사후연구원 & 대학 소속 계약 연구원 & ⚠️ 확인 필요 \\
        순수 연구원 & 대학 소속이 아닌 연구기관 & ❌ 유료 \\
        \bottomrule
    \end{tabularx}
\end{table}

\textit{※ 대학 소속이 아닌 연구원은 GitHub Copilot 유료 구독(\$10/월) 또는 기관 라이선스가 필요합니다.}

\subsubsection{신청 절차 (대학 소속인 경우)}
\begin{enumerate}
    \item \href{https://education.github.com/pack}{education.github.com/pack} 접속
    \item \textbf{`Get your pack'} 버튼 클릭
    \item GitHub 계정으로 로그인
    \item \textbf{학교 이메일 인증}: 대학 메일 주소 입력 후 인증
    \item \textbf{신분 증명 제출}:
    \begin{itemize}
        \item 대학원생: 재학증명서 또는 학생증
        \item 연구교수/박사후연구원: 재직증명서 또는 임용계약서
    \end{itemize}
    \item 승인 대기 (1-5일 소요)
\end{enumerate}

\subsection{VS Code 설치 및 설정}
\begin{enumerate}
    \item \href{https://code.visualstudio.com}{code.visualstudio.com}에서 VS Code 다운로드 및 설치
    \item VS Code 실행 후 왼쪽 \textbf{Extensions 아이콘} 클릭
    \item 검색창에 \texttt{GitHub Copilot} 입력 후 설치
    \item 추가 권장 확장 프로그램:
    \begin{itemize}
        \item \textbf{Python}: 데이터 분석을 위한 Python 지원
        \item \textbf{Jupyter}: 노트북 환경 지원
        \item \textbf{Mermaid Preview}: 다이어그램 미리보기
        \item \textbf{LaTeX Workshop}: 논문 작성용 LaTeX 지원
    \end{itemize}
    \item 좌측 하단 프로필 아이콘 클릭 $\rightarrow$ GitHub 계정으로 로그인
\end{enumerate}

% ==============================================================================
\section{연구 단계별 활용}
% ==============================================================================

\subsection{문헌 검토 보조}

\subsubsection{논문 요약}
\textbf{프롬프트 예시}:
\begin{quote}
"다음 논문 초록을 분석하고, 아래 형식으로 요약해주세요:
\begin{itemize}
    \item 연구 목적 (1문장)
    \item 연구 방법 (1-2문장)
    \item 주요 결과 (2-3문장)
    \item 한계점 (1문장)
    \item 후속 연구 방향 (1문장)
\end{itemize}

[초록 내용 붙여넣기]"
\end{quote}

\subsubsection{복수 논문 비교}
\textbf{프롬프트 예시}:
\begin{quote}
"다음 3개 논문의 연구 방법론을 비교 분석해주세요:
\begin{itemize}
    \item 논문 A: [제목 및 초록]
    \item 논문 B: [제목 및 초록]
    \item 논문 C: [제목 및 초록]
\end{itemize}
비교 항목: 연구 설계, 표본 크기, 분석 방법, 주요 변수

결과를 표 형식으로 정리해주세요."
\end{quote}

\textbf{주의}: AI 요약만 의존하지 말고, \textbf{반드시 원문을 직접 읽으세요}. AI는 핵심 뉘앙스를 놓칠 수 있습니다.

\subsection{연구 계획서 구조화}
\textbf{프롬프트 예시}:
\begin{quote}
"다음 연구 아이디어를 연구 계획서 개요로 구조화해주세요:

연구 주제: [주제 설명]

포함 섹션:
\begin{enumerate}
    \item 연구 배경 및 필요성
    \item 연구 목적 및 가설
    \item 이론적 배경
    \item 연구 방법론
    \item 예상 결과 및 기대 효과
    \item 연구 일정 (간트 차트 형식)
    \item 예산 계획
\end{enumerate}"
\end{quote}

\subsection{실험 프로토콜 정리}
\textbf{프롬프트 예시}:
\begin{quote}
"다음 실험 과정을 재현 가능한 프로토콜 형식으로 정리해주세요:

[실험 과정 메모 붙여넣기]

요구 형식:
\begin{itemize}
    \item 필요 재료 및 장비 목록
    \item 단계별 절차 (번호 매김)
    \item 각 단계의 소요 시간
    \item 주의사항 및 안전 수칙
    \item 예상 결과 및 품질 관리 기준
\end{itemize}"
\end{quote}

% ==============================================================================
\section{데이터 분석}
% ==============================================================================

\subsection{코드 생성 (Python, R)}

\subsubsection{Python 데이터 분석}
\textbf{프롬프트 예시}:
\begin{quote}
"다음 CSV 파일의 구조를 분석하고, 기술 통계량을 계산하는 Python 코드를 작성해주세요:
\begin{itemize}
    \item 파일: \texttt{survey\_data.csv}
    \item 컬럼: id, age, gender, score1, score2, score3
    \item 분석 항목: 평균, 표준편차, 결측치 수, 성별별 평균 비교
\end{itemize}
pandas와 scipy 라이브러리를 사용해주세요."
\end{quote}

\subsubsection{R 통계 분석}
\textbf{프롬프트 예시}:
\begin{quote}
"다음 연구 설계에 맞는 R 분석 코드를 작성해주세요:
\begin{itemize}
    \item 연구 설계: 2(집단) × 3(시점) 반복측정 ANOVA
    \item 종속 변수: performance\_score
    \item 독립 변수: group (실험/통제), time (사전/중간/사후)
    \item 피험자 변수: participant\_id
\end{itemize}
가정 검정(정규성, 구형성)과 사후 검정까지 포함해주세요."
\end{quote}

\subsection{통계 분석 지원}

\subsubsection{분석 방법 선택}
\textbf{프롬프트 예시}:
\begin{quote}
"다음 연구 설계에 적합한 통계 분석 방법을 추천해주세요:
\begin{itemize}
    \item 연구 질문: 교수법(3종류)에 따른 학업 성취도 차이가 있는가?
    \item 표본: 90명 (각 집단 30명)
    \item 종속 변수: 시험 점수 (연속형, 0-100)
    \item 공변인: 사전 학업 능력
\end{itemize}
분석 방법 이름, 가정, R/Python 함수명을 알려주세요."
\end{quote}

\subsubsection{결과 해석 도움}
\textbf{프롬프트 예시}:
\begin{quote}
"다음 ANOVA 결과를 학술 논문에 보고할 형식으로 해석해주세요:
\begin{verbatim}
F(2, 87) = 4.52, p = .013, η² = .094
사후 검정: Group A > Group C (p = .009)
\end{verbatim}
APA 7판 형식으로 결과 문장과 해석을 작성해주세요."
\end{quote}

\subsection{결과 시각화}

\subsubsection{Mermaid 다이어그램}
연구 모델이나 프로세스를 시각화할 때 유용합니다.

\textbf{프롬프트 예시}:
\begin{quote}
"다음 연구 가설을 Mermaid 다이어그램 코드로 시각화해주세요:
\begin{itemize}
    \item 독립 변수: 리더십 스타일 (변혁적/거래적)
    \item 매개 변수: 조직 몰입
    \item 종속 변수: 업무 성과
    \item 조절 변수: 근속 연수
\end{itemize}"
\end{quote}

\subsubsection{matplotlib/seaborn 시각화}
\textbf{프롬프트 예시}:
\begin{quote}
"다음 데이터를 시각화하는 Python 코드를 작성해주세요:
\begin{itemize}
    \item 데이터: 집단별(3개) 시점별(4개) 평균 점수
    \item 그래프 유형: 선 그래프 (오차 막대 포함)
    \item 스타일: 학술 논문 출판 품질
    \item 파일 저장: 300 DPI, PDF 형식
\end{itemize}"
\end{quote}

% ==============================================================================
\section{논문 작성}
% ==============================================================================

\subsection{초록/서론 초안}
\textbf{주의}: AI가 생성한 초안은 \textbf{시작점}일 뿐입니다. 반드시 본인이 직접 수정하고, AI 사용을 논문에 명시해야 합니다.

\textbf{프롬프트 예시}:
\begin{quote}
"다음 연구 내용을 바탕으로 250단어 분량의 영문 초록 초안을 작성해주세요:
\begin{itemize}
    \item 연구 배경: [1-2문장]
    \item 연구 방법: [2-3문장]
    \item 주요 결과: [2-3문장]
    \item 결론 및 시사점: [1-2문장]
\end{itemize}
학술 논문에 적합한 객관적이고 간결한 문체를 사용해주세요."
\end{quote}

\subsection{IMRaD 구조 점검}
\textbf{프롬프트 예시}:
\begin{quote}
"다음 논문 초고가 IMRaD 구조를 적절히 따르고 있는지 검토해주세요:

[논문 초고 붙여넣기]

검토 항목:
\begin{enumerate}
    \item Introduction: 연구 배경, 문헌 검토, 연구 목적/가설 명시 여부
    \item Methods: 연구 설계, 참가자, 절차, 분석 방법 명확성
    \item Results: 결과 제시의 논리적 순서, 표/그림 적절성
    \item Discussion: 결과 해석, 한계점, 후속 연구 제안 포함 여부
\end{enumerate}
각 섹션별 개선점을 구체적으로 제안해주세요."
\end{quote}

\subsection{참고문헌 형식화 (APA, MLA)}

\subsubsection{APA 7판 변환}
\textbf{프롬프트 예시}:
\begin{quote}
"다음 참고문헌 정보를 APA 7판 형식으로 변환해주세요:
\begin{enumerate}
    \item 저자: Kim, S. H., Park, J. / 연도: 2023 / 제목: AI in education / 저널: Educational Technology / 권: 15 / 호: 2 / 페이지: 45-67 / DOI: 10.1234/et.2023.15.2.45
    \item 저자: Lee, M. / 연도: 2022 / 책 제목: Machine Learning Fundamentals / 출판사: Academic Press / 출판지: Seoul
\end{enumerate}"
\end{quote}

\subsubsection{EndNote/Zotero 내보내기}
\textbf{프롬프트 예시}:
\begin{quote}
"다음 참고문헌을 EndNote XML 또는 RIS 형식으로 변환해주세요:

[참고문헌 목록 붙여넣기]"
\end{quote}

% ==============================================================================
\section{연구윤리}
% ==============================================================================

\subsection{데이터 보안 주의사항}
\begin{table}[H]
    \centering
    \renewcommand{\arraystretch}{1.3}
    \begin{tabularx}{\textwidth}{@{}l X@{}}
        \toprule
        \textbf{데이터 유형} & \textbf{AI 입력 가능 여부} \\
        \midrule
        공개된 학술 데이터 & ✅ 가능 \\
        익명화된 집합 데이터 & ✅ 가능 \\
        연구 참여자 원시 데이터 & ⚠️ 익명화 후 가능 \\
        개인 식별 정보 포함 데이터 & ❌ 금지 \\
        IRB 승인 전 민감 데이터 & ❌ 금지 \\
        기업/기관 기밀 데이터 & ❌ 금지 \\
        \bottomrule
    \end{tabularx}
\end{table}

\textbf{안전한 사용법}:
\begin{itemize}
    \item 참여자 이름 $\rightarrow$ "P01", "P02" 등 코드로 대체
    \item 구체적 연령 $\rightarrow$ 연령대(20대, 30대)로 일반화
    \item 기관명 $\rightarrow$ "A 대학", "B 병원"으로 익명화
\end{itemize}

\subsection{AI 사용 공개 가이드라인}

\subsubsection{학술지 정책 확인}
제출 전 해당 학술지의 AI 사용 정책을 반드시 확인하세요:
\begin{itemize}
    \item \textbf{Nature}: AI 사용 시 방법론에 명시 필수
    \item \textbf{Science}: AI 생성 텍스트 저자 자격 불인정
    \item \textbf{국내 학술지}: 학회별 정책 상이, 개별 확인 필요
\end{itemize}

\subsubsection{권장 명시 방법}
\textbf{방법론 섹션에 기재}:
\begin{quote}
"본 연구의 원고 작성 과정에서 GitHub Copilot(Microsoft, 2024)을 문법 검토 및 표현 개선 목적으로 활용하였습니다. 모든 연구 내용, 분석, 해석은 저자가 직접 수행하였으며, AI 생성 콘텐츠는 저자의 검토를 거쳐 수정되었습니다."
\end{quote}

\textbf{감사의 글에 기재}:
\begin{quote}
"저자는 원고 편집 과정에서 GitHub Copilot의 보조를 받았습니다."
\end{quote}

\subsection{절대 금지 사항}
\begin{itemize}
    \item AI 생성 논문/보고서를 본인 저작물로 제출
    \item AI가 만들어낸 \textbf{가짜 참고문헌} 인용
    \item 연구 데이터를 AI로 생성하거나 조작
    \item 피어 리뷰 과정에서 AI 도구 무단 사용
\end{itemize}

% ==============================================================================
\section{부록}
% ==============================================================================

\subsection{연구자용 프롬프트 사전}
\begin{itemize}
    \item \textbf{문헌 검토}: "이 분야의 최근 5년간 연구 동향을 요약해주세요."
    \item \textbf{가설 검토}: "이 연구 가설의 논리적 타당성을 검토해주세요."
    \item \textbf{통계 선택}: "이 연구 설계에 적합한 통계 방법을 추천해주세요."
    \item \textbf{코드 디버깅}: "이 R/Python 코드의 오류를 찾고 수정해주세요."
    \item \textbf{결과 해석}: "이 통계 결과를 APA 형식으로 보고해주세요."
    \item \textbf{영어 교정}: "이 문단을 학술 영어로 다듬어주세요."
\end{itemize}

\subsection{단축키 요약}
\begin{table}[H]
    \centering
    \renewcommand{\arraystretch}{1.1}
    \begin{tabularx}{\textwidth}{@{}l X@{}}
        \toprule
        \textbf{단축키} & \textbf{기능} \\
        \midrule
        \texttt{Ctrl/Cmd + Alt + I} & Copilot 채팅창 열기/닫기 \\
        \texttt{Tab} & 자동완성 제안 수락 \\
        \texttt{Esc} & 자동완성 제안 무시 \\
        \texttt{\#} & 채팅창에서 파일 참조 \\
        \texttt{@workspace} & 프로젝트 전체 파일 참조 \\
        \bottomrule
    \end{tabularx}
\end{table}

\vspace{1em}
\hrule
\vspace{0.5em}

\section*{유의 사항}
\small
본 문서는 작성 시점(\docdate) 기준이며, GitHub Copilot의 정책 및 기능은 변경될 수 있습니다. AI 도구 사용 시 연구윤리, IRB 규정, 학술지 정책을 반드시 준수하십시오. 최신 정보는 공식 문서(\href{https://docs.github.com/en/copilot}{docs.github.com/copilot})를 참조하십시오.

\end{document}
