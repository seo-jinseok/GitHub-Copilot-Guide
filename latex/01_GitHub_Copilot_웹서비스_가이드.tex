\documentclass[11pt, a4paper]{article}

% --- DOCUMENT VERSION CONTROL ---
% PDF 빌드 시점의 날짜와 시간(KST, UTC+9)이 자동으로 반영됩니다
% 참고: Overleaf 서버는 CET(UTC+1) 시간대를 사용하므로 +8로 보정합니다
\usepackage{datetime2}
\ExplSyntaxOn
\int_new:N \g_kst_hour_int
\int_new:N \g_kst_day_int
\int_new:N \g_kst_month_int
\int_new:N \g_kst_year_int
\int_gset:Nn \g_kst_hour_int { \the\time / 60 + 9 }
\int_gset:Nn \g_kst_day_int { \the\day }
\int_gset:Nn \g_kst_month_int { \the\month }
\int_gset:Nn \g_kst_year_int { \the\year }
\int_compare:nNnT { \g_kst_hour_int } > { 23 }
  {
    \int_gadd:Nn \g_kst_hour_int { -24 }
    \int_gincr:N \g_kst_day_int
  }
\newcommand{\docversion}{최종~수정:~\int_use:N \g_kst_year_int.~\int_compare:nNnTF { \g_kst_month_int } < { 10 } {0}{} \int_use:N \g_kst_month_int.~\int_compare:nNnTF { \g_kst_day_int } < { 10 } {0}{} \int_use:N \g_kst_day_int.
  \c_space_tl \int_compare:nNnTF { \g_kst_hour_int } < { 10 } {0}{} \int_use:N \g_kst_hour_int
  : \int_compare:nNnTF { \int_mod:nn {\the\time} {60} } < { 10 } {0}{} \int_eval:n { \int_mod:nn {\the\time} {60} }}
\newcommand{\docdate}{\int_use:N \g_kst_year_int 년~\int_compare:nNnTF { \g_kst_month_int } < { 10 } {0}{} \int_use:N \g_kst_month_int 월~\int_compare:nNnTF { \g_kst_day_int } < { 10 } {0}{} \int_use:N \g_kst_day_int 일}
\ExplSyntaxOff
\newcommand{\authors}{동의대학교 서진석(jsseo@deu.ac.kr)}

% --- UNIVERSAL PREAMBLE BLOCK ---
\usepackage[a4paper, top=3cm, bottom=3cm, left=2.5cm, right=2.5cm, headheight=14pt]{geometry}
\usepackage{fontspec}

% Language setup for Korean
\usepackage[korean, provide=*]{babel}

% Fonts: Using Noto Serif for body (formal) and Noto Sans for headers
\babelfont{rm}{Noto Serif CJK KR}
\babelfont{sf}{Noto Sans CJK KR}

% CJK 폰트는 이탤릭이 없으므로 폰트 대체 규칙 설정 (경고 방지)
\DeclareFontShape{TU}{NotoSerifCJKKR(0)}{m}{it}{<->ssub*NotoSerifCJKKR(0)/m/n}{}
\DeclareFontShape{TU}{NotoSerifCJKKR(0)}{m}{sl}{<->ssub*NotoSerifCJKKR(0)/m/n}{}
\DeclareFontShape{TU}{NotoSerifCJKKR(0)}{bx}{it}{<->ssub*NotoSerifCJKKR(0)/bx/n}{}
\DeclareFontShape{TU}{NotoSansCJKKR(0)}{m}{it}{<->ssub*NotoSansCJKKR(0)/m/n}{}
\DeclareFontShape{TU}{NotoSansCJKKR(0)}{m}{sl}{<->ssub*NotoSansCJKKR(0)/m/n}{}
\DeclareFontShape{TU}{NotoSansCJKKR(0)}{bx}{it}{<->ssub*NotoSansCJKKR(0)/bx/n}{}

% Essential Packages
\usepackage{enumitem}
\usepackage{booktabs}   % Professional tables
\usepackage{tabularx}   % Auto-width tables
\usepackage{titlesec}   % Section formatting
\usepackage{xcolor}     % Colors for visual hierarchy
\usepackage{fancyhdr}   % Headers and footers
\usepackage{setspace}   % Line spacing
\usepackage{longtable}  % For tables that might span pages
\usepackage{xltabular}  % For flexible width long tables
\usepackage{float}      % For [H] option to fix table position
\usepackage{kotex}      % Additional Korean support insurance
\usepackage[hidelinks]{hyperref}

% --- STYLING COMMANDS ---

% Define a formal navy blue for headers
\definecolor{eduNavy}{RGB}{0, 43, 91}

% Section styling
\titleformat{\section}
  {\Large\sffamily\bfseries\color{eduNavy}}
  {\thesection}{1em}{}
  [\vspace{-0.5em}\hrule height 1pt]

\titleformat{\subsection}
  {\large\sffamily\bfseries\color{eduNavy}}
  {\thesubsection}{1em}{}

% List settings
\setlist[itemize]{label=\textbullet, leftmargin=1.5em, itemsep=0.2em}
\setlist[enumerate]{label=\arabic*., leftmargin=1.5em, itemsep=0.2em}

% Line spacing for readability
\setstretch{1.2}

% Header and Footer
\pagestyle{fancy}
\fancyhf{}
\fancyhead[L]{\small\sffamily \authors}
\fancyhead[R]{\small\sffamily \docversion}
\fancyfoot[C]{\thepage}
\renewcommand{\headrulewidth}{0.5pt}

% Metadata
\title{\vspace{-2.5cm} \Huge\sffamily\bfseries GitHub Copilot 웹서비스 활용 가이드 \\ \large\vspace{0.2em} \textmd{대학 구성원을 위한 교육용 라이선스 등록 및 모델 활용 매뉴얼}\vspace{-0.5em}}
\author{\vspace{-0.5em}\authors}
\date{\vspace{-0.5em}\docversion}

\begin{document}

\maketitle
\thispagestyle{empty}
\vspace{-1.5cm}

\section{개요}
\subsection{GitHub Copilot이란?}
GitHub Copilot(깃허브 코파일럿)은 웹 브라우저에서 OpenAI, Google, Anthropic 등 다양한 최신 AI 모델을 선택하여 사용할 수 있는 범용 AI 채팅 서비스입니다.

\subsection{주요 특징}
\begin{itemize}
    \item \textbf{무료 라이선스}: 대학 구성원(교수, 학생)에게 \textbf{GitHub Copilot Pro} 무료 제공
    \item \textbf{다양한 AI 모델}: GPT-5, Claude, Gemini 등 최신 모델을 한 곳에서 선택하여 사용 가능
\end{itemize}

\subsection{지원 자격}
\begin{table}[H]
    \centering
    \renewcommand{\arraystretch}{1.1}
    \begin{tabularx}{\textwidth}{@{}l l X@{}}
        \toprule
        \textbf{구분} & \textbf{신청 가능 여부} & \textbf{비고} \\
        \midrule
        학생(학부/대학원) & \textbf{가능} & 영문 재학증명서 제출 필요 \\
        교수 & \textbf{가능} & 영문 재직증명서 제출 필요 \\
        연구원 & 직접 확인 필요 & 승인 사례가 일관되지 않음 \\
        행정 직원 & 불가 & 별도의 유료 라이선스 구매 필요 \\
        \bottomrule
    \end{tabularx}
\end{table}

\subsection{GitHub Copilot Pro 라이선스 혜택}
\begin{table}[H]
    \centering
    \renewcommand{\arraystretch}{1.2}
    \small
    \begin{tabularx}{\textwidth}{@{}l X X@{}}
        \toprule
        \textbf{구분} & \textbf{무료 사용자 (Basic)} & \textbf{Pro 사용자 (대학 구성원)} \\
        \midrule
        \textbf{대화 한도} & 기본 모델 월 50회 제한 & \textbf{기본 모델 무제한}, 프리미엄 월 300회 \\
        \textbf{사용 모델} & 기본 모델만 (GPT-5 mini 등) & \textbf{전체 모델} (Claude Opus, Gemini Pro 등) \\
        \textbf{데이터 보안} & 학습에 사용될 수 있음 & \textbf{학습 데이터로 사용 안 됨} \\
        \bottomrule
    \end{tabularx}
\end{table}

\section{무료 사용 신청 가이드}
신청 후 승인까지는 통상 영업일 기준 \textbf{1\textasciitilde3일}이 소요됩니다.

\subsection{사전 준비 사항}
\begin{itemize}
    \item \textbf{학교 공식 이메일}: \texttt{@deu.ac.kr}, \texttt{@g.deu.ac.kr} 등 공식 도메인 사용
    \item \textbf{영문 증명 서류}: \textbf{영문(English)} 재학증명서(학생) 또는 재직증명서(교수)
    \begin{itemize}
        \item \textbf{종이 원본}과 \textbf{스캔 파일} 모두 준비 (PC 환경에 따라 카메라 촬영만 가능한 경우도 있음)
        \item \textbf{스캔 시 중요}: 반드시 \textbf{컬러 모드}로 스캔하고 \textbf{자동 문서 보정 기능} 비활성화. 학교 로고(워터마크)가 사라지면 위조 문서로 의심받아 승인이 거절됨
    \end{itemize}
    \item \textbf{인증 앱}: 스마트폰에 \textit{Microsoft Authenticator} 설치 (회원가입 불필요, 설치만)
\end{itemize}

\subsection{등록 절차}
\begin{enumerate}
    \item \textbf{GitHub 회원가입}: \href{https://github.com}{github.com}에서 학교 이메일로 가입 후 이메일 인증
    \item \textbf{프로필 설정}: GitHub 로그인 $\rightarrow$ 우측 상단 프로필 아이콘 클릭 $\rightarrow$ Settings $\rightarrow$ Public profile에서 \textbf{Name}을 \textbf{증명서와 동일한 영문 이름}으로 입력
    \item \textbf{결제 정보 입력}: Settings $\rightarrow$ Billing and licensing $\rightarrow$ Payment information에서 \textbf{증명서와 동일한 영문 이름} 입력 (결제 수단 등록 불필요)
    \item \textbf{2단계 인증(2FA) 설정}:
    \begin{itemize}
        \item 프로필 $\rightarrow$ \textbf{Settings} $\rightarrow$ \textbf{Password and authentication} $\rightarrow$ \textbf{Two-factor authentication} 활성화
        \item Authenticator 앱에서 QR 코드 스캔 후 6자리 숫자 입력
        \item \textit{주의: 복구 코드는 반드시 별도 저장 (USB, 클라우드, 인쇄물)}
    \end{itemize}
    \item \textbf{GitHub Education 신청}: \href{https://github.com/education}{github.com/education} 접속 $\rightarrow$ \textbf{Join GitHub Education}
    \item \textbf{인증}: 신분(Student/Teacher) 선택, 브라우저 위치 공유 허용 (거부 시 신청 거절)
    \item \textbf{서류 제출}: 영문 증명서 업로드 또는 카메라 촬영 후 \textbf{Submit}
\end{enumerate}

\section{설정 및 활용 방법}
승인 완료 후 \href{https://github.com/copilot}{copilot.github.com}에서 AI 기능을 사용할 수 있습니다.

\subsection{초기 설정}
\begin{enumerate}
    \item GitHub 로그인 후 프로필 메뉴의 \textbf{Copilot Settings}로 이동합니다.
    \item \textbf{Features} 항목에서 비활성화(Disabled)된 모델들을 \textbf{Enabled}로 변경합니다.
    \item \textbf{Copilot can search the web}을 활성화하면 실시간 웹 검색 기반 답변이 가능해집니다.
\end{enumerate}

\subsection{대화 시작 및 파일 첨부}
\begin{itemize}
    \item \href{https://github.com/copilot}{copilot.github.com} 접속 후 입력창에 질문을 입력하고 \textbf{Enter}를 누르면 AI가 응답합니다.
    \item \textbf{파일 첨부}: 입력창 좌측 \textbf{+ 아이콘} $\rightarrow$ \textbf{Upload from computer}
    \item \textbf{지원 형식}: 이미지(JPG, PNG, GIF, WEBP), 텍스트(TXT, MD, CSV)
    \item \textbf{미지원}: PDF, Word, Excel, PPT, 동영상, 음성 \textrightarrow{} 텍스트 복사 또는 화면 캡처로 대체
    \item \textbf{Excel 팁}: CSV 형식으로 변환 후 첨부 (파일 $\rightarrow$ 다른 이름으로 저장 $\rightarrow$ CSV UTF-8)
    \item \textbf{모델 변경}: 입력창 우측 하단의 모델명 클릭
\end{itemize}

\section{AI 모델 선택 가이드}
\textit{※ 일반적 권장사항이며, 직접 사용 후 본인에게 적합한 모델을 찾으시길 권장합니다.}

\subsection{모델별 상세 사양}
\begin{xltabular}{\textwidth}{@{}l l X@{}}
\toprule
\textbf{등급} & \textbf{모델명} & \textbf{특징 및 주요 강점} \\
\midrule
\endhead

\textbf{일반} & GPT-5 mini & \textbf{빠른 속도 + 이미지 분석}. 일상적 문의, 문서 작성에 최적. \\
\textbf{일반} & GPT-4.1 & \textbf{안정적 범용 + 이미지 분석}. 다양한 업무에 균형 잡힌 품질. \\
\textbf{일반} & Claude Haiku 4.5 & \textbf{대량 작업 처리}. 많은 문서를 빠르게 처리해야 할 때 적합. \\
\midrule
\textbf{프리미엄} & Grok Code Fast 1 & \textbf{프로그래밍 전문}. 코드 작성, 오류 수정에 빠른 응답. \\
\textbf{프리미엄} & GPT-5 & \textbf{고급 문제 해결}. 복잡한 문제를 단계별로 분석. \\
\textbf{프리미엄} & GPT-5.1 & \textbf{장기 프로젝트}. 여러 단계의 복잡한 작업 수행에 특화. \\
\textbf{프리미엄} & Claude Sonnet 4 & \textbf{이미지 분석 + 심층 추론}. 코딩 작업 흐름에 최적화. \\
\textbf{프리미엄} & Claude Sonnet 4.5 & \textbf{한국어 문서 작성 최적}. 보고서, 기획서, 공문서에 탁월. \\
\textbf{프리미엄} & Claude Opus 4.1 & \textbf{학술 연구 및 심층 분석}. 논문 분석, 정책 검토에 적합. \\
\textbf{프리미엄} & Claude Opus 4.5 & \textbf{최신 최고 성능}. 극도로 복잡한 분석이나 창의적 업무에 적합. \\
\textbf{프리미엄} & Gemini 2.5 Pro & \textbf{대용량 문서 분석}. 매우 긴 문서 처리 + 이미지 분석 가능. \\
\textbf{프리미엄} & Gemini 3 Pro & \textbf{최신 멀티모달}. 뛰어난 추론 + 이미지 분석 (영상/음성 미지원). \\
\bottomrule
\end{xltabular}

\subsection{업무별 추천}
\begin{itemize}
    \item \textbf{일상 간단한 질문}: GPT-5 mini, GPT-4.1
    \item \textbf{한국어 문서}: Claude Sonnet 4.5 \quad \textbf{연구/심층 분석}: Claude Opus 4.1/4.5
    \item \textbf{대량 처리}: Claude Haiku 4.5 \quad \textbf{초대용량}: Gemini 2.5/3 Pro
\end{itemize}

\vspace{1em}
\hrule
\vspace{0.5em}

\section*{유의 사항}
\small
본 문서는 작성 시점(\docdate) 기준이며, GitHub Copilot의 정책 및 기능은 변경될 수 있습니다. 최신 정보는 공식 페이지(\href{https://github.com/features/copilot}{github.com/features/copilot}) 및 공식 문서(\href{https://docs.github.com/en/copilot}{docs.github.com/copilot})를 참조하십시오.
\end{document}