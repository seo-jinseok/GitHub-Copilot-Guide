\documentclass[11pt, a4paper]{article}

% --- DOCUMENT VERSION CONTROL ---
% PDF 빌드 시점의 날짜가 자동으로 반영됩니다
\newcommand{\docversion}{최종 수정: \today}

% --- UNIVERSAL PREAMBLE BLOCK ---
\usepackage[a4paper, top=3cm, bottom=3cm, left=2.5cm, right=2.5cm]{geometry}
\usepackage{fontspec}

% Language setup for Korean
\usepackage[korean, provide=*]{babel}

% Fonts: Using Noto Serif for body (formal) and Noto Sans for headers
\babelfont{rm}{Noto Serif CJK KR}
\babelfont{sf}{Noto Sans CJK KR}

% Essential Packages
\usepackage{enumitem}
\usepackage{booktabs}   % Professional tables
\usepackage{tabularx}   % Auto-width tables
\usepackage{titlesec}   % Section formatting
\usepackage{xcolor}     % Colors for visual hierarchy
\usepackage{fancyhdr}   % Headers and footers
\usepackage{setspace}   % Line spacing
\usepackage{longtable}  % For tables that might span pages
\usepackage{xltabular}  % For flexible width long tables
\usepackage{kotex}      % Additional Korean support insurance
\usepackage[hidelinks]{hyperref}

% --- STYLING COMMANDS ---

% Define a formal navy blue for headers
\definecolor{eduNavy}{RGB}{0, 43, 91}

% Section styling
\titleformat{\section}
  {\Large\sffamily\bfseries\color{eduNavy}}
  {\thesection}{1em}{}
  [\vspace{-0.5em}\hrule height 1pt]

\titleformat{\subsection}
  {\large\sffamily\bfseries\color{eduNavy}}
  {\thesubsection}{1em}{}

% List settings
\setlist[itemize]{label=\textbullet, leftmargin=1.5em, itemsep=0.2em}
\setlist[enumerate]{label=\arabic*., leftmargin=1.5em, itemsep=0.2em}

% Line spacing for readability
\setstretch{1.3}

% Header and Footer
\pagestyle{fancy}
\fancyhf{}
\fancyhead[L]{\small\sffamily 동의대학교 서진석}
\fancyhead[R]{\small\sffamily \docversion}
\fancyfoot[C]{\thepage}
\renewcommand{\headrulewidth}{0.5pt}

% Metadata
\title{\vspace{-2cm} \Huge\sffamily\bfseries GitHub Copilot 웹서비스 활용 가이드 \\ \large\vspace{0.5em} \textmd{대학 구성원을 위한 교육용 라이선스 등록 및 모델 활용 매뉴얼}}
\author{동의대학교 서진석(jsseo@deu.ac.kr), Sonnet 4.5, Gemini 3 Pro}
\date{\docversion}

\begin{document}

\maketitle
\thispagestyle{empty}
\vspace{-2cm}

\section{개요}
\subsection{GitHub Copilot이란?}
GitHub Copilot(깃허브 코파일럿)은 본래 프로그래머를 위한 코드 작성 보조 도구(VS Code 확장 프로그램)로 출시되었으나, 현재는 범용 AI 채팅 서비스를 통합하여 제공하고 있습니다. 사용자는 별도의 소프트웨어 설치 없이 웹 브라우저 상에서 OpenAI, Google, Anthropic 등의 최신 AI 모델을 통합적으로 활용할 수 있습니다.

\subsection{주요 특징}
\begin{itemize}
    \item \textbf{무료 라이선스 제공}: 대학 구성원(교수, 학생)에게는 월 \$10 상당의 \textbf{GitHub Copilot Pro} 기능이 무료로 제공됩니다.
    \item \textbf{모델 통합 플랫폼}: GPT-5, Claude, Gemini 등 다양한 최신 LLM(Large Language Model)을 하나의 플랫폼에서 선택하여 사용할 수 있습니다.
    \item \textbf{웹 기반 접근성}: 별도의 설치 과정 없이 웹사이트(\href{https://github.com/copilot}{copilot.github.com})를 통해 즉시 이용 가능합니다.
\end{itemize}

\subsection{지원 자격}
\begin{table}[h]
    \centering
    \renewcommand{\arraystretch}{1.2}
    \begin{tabularx}{\textwidth}{@{}l l X@{}}
        \toprule
        \textbf{구분} & \textbf{신청 가능 여부} & \textbf{비고} \\
        \midrule
        학생(학부/대학원) & \textbf{가능} & 영문 재학증명서 제출 필요 \\
        교수 & \textbf{가능} & 영문 재직증명서 제출 필요 \\
        연구원 & 불가 & 단, 학생 또는 교원 신분 겸직 시 신청 가능 \\
        행정 직원 & 불가 & 별도의 유료 라이선스 구매 필요 \\
        \bottomrule
    \end{tabularx}
\end{table}

\section{GitHub Copilot Pro 라이선스 혜택}
대학 구성원 인증 시 일반 사용자 대비 확장된 기능을 제공받을 수 있습니다.

\begin{table}[h]
    \centering
    \renewcommand{\arraystretch}{1.2}
    \small
    \begin{tabularx}{\textwidth}{@{}l X X@{}}
        \toprule
        \textbf{구분} & \textbf{무료 사용자 (Basic)} & \textbf{Pro 사용자 (대학 구성원)} \\
        \midrule
        \textbf{비용} & 무료 & \textbf{무료} (연 20만원 상당 지원) \\
        \addlinespace
        \textbf{대화 한도} & 기본 모델 월 50회 제한 & \textbf{기본 모델 무제한} \newline 프리미엄 모델 월 300회 (매월 1일 초기화) \\
        \addlinespace
        \textbf{사용 모델} & 기본 모델 제한적 사용 \newline (GPT-5 mini, GPT-4o 등) & \textbf{전체 모델 사용 가능} \newline (Grok, Claude Opus, Gemini Pro 등 포함) \\
        \addlinespace
        \textbf{데이터 보안} & 대화 내용이 학습에 사용될 수 있음 & \textbf{학습 데이터로 사용되지 않음} (철저한 보안) \\
        \bottomrule
    \end{tabularx}
\end{table}

\newpage

\section{무료 사용 신청 가이드}
신청 후 승인까지는 통상 영업일 기준 \textbf{1\textasciitilde3일}이 소요됩니다. 원활한 승인을 위해 아래 절차를 준수하여 주십시오.

\subsection{사전 준비 사항 (필수)}
\begin{itemize}
    \item \textbf{학교 공식 이메일}: \texttt{@deu.ac.kr}, \texttt{@g.deu.ac.kr}, \texttt{@office.deu.ac.kr} 등 공식 도메인 사용을 권장합니다.
    \item \textbf{영문 증명 서류}: 반드시 \textbf{영문(English)}으로 발급된 재학증명서(학생) 또는 재직증명서(교수)를 준비하십시오.
    \begin{itemize}
        \item 발급 받은 서류의 출력본(종이)과 서류 전체가 식별 가능한 스캔 파일(JPG/PNG)을 준비해야 합니다. 
        \item PC 환경에 따라 카메라 촬영을 요구할 수도 있고, 파일을 업로드 해야 할 수도 있습니다.
    \end{itemize}
    \item \textbf{인증 앱(OTP)}: 스마트폰에 \textit{Microsoft Authenticator} 앱을 사전에 설치하십시오.
\end{itemize}

\subsection{단계별 등록 절차}
\begin{enumerate}
    \item \textbf{GitHub 회원가입}: \href{https://github.com}{github.com}에 접속하여 학교 공식 이메일로 가입을 진행합니다. 이메일로 발송된 8자리 인증 코드를 입력하여 완료합니다.
    \item \textbf{2단계 인증(2FA) 설정}: 보안 정책상 필수 단계입니다.
    \begin{itemize}
        \item GitHub에 로그인 한 후, 우측 상단 프로필 $\rightarrow$ \textbf{Settings} $\rightarrow$ \textbf{Password and authentication}
        \item \textbf{Two-factor authentication}을 활성화하고 OTP 앱으로 QR 코드를 스캔하여 등록합니다.
        \item \textit{주의: 생성되는 복구 코드(Recovery codes)는 반드시 별도로 저장하십시오.}
    \end{itemize}
    \item \textbf{GitHub Education 신청}: \href{https://github.com/education}{github.com/education}으로 이동하여 \textbf{Join GitHub Education}을 선택, 신청서를 작성합니다.
    \item \textbf{신분 및 학교 인증}:
    \begin{itemize}
        \item 신분(Student/Teacher)을 선택하면 우리 대학이 자동으로 선택됩니다(회원 가입에 사용한 이메일 주소의 도메인 기준).
        \item 브라우저의 위치 공유 권한을 허용하여 현재 위치를 인증합니다.
    \end{itemize}
    \item \textbf{서류 제출}: 준비한 영문 증명서를 업로드하거나 카메라로 촬영하여 제출합니다. \textbf{Submit} 버튼을 누르면 신청이 완료됩니다.
\end{enumerate}

\section{설정 및 활용 방법}
승인이 완료되면 웹 브라우저(\href{https://github.com/copilot}{copilot.github.com})에서 즉시 AI 기능을 사용할 수 있습니다.

\subsection{모델 활성화 및 기능 설정}
다양한 AI 모델을 활용하기 위해 다음 설정을 확인하시기 바랍니다.
\begin{enumerate}
    \item GitHub 로그인 후 프로필 메뉴의 \textbf{Copilot Settings}로 이동합니다.
    \item \textbf{Features} 항목에서 비활성화(Disabled)된 모델들을 \textbf{Enabled}로 변경합니다.
    \item \textbf{Dashboard entry point}를 활성화하면 GitHub 메인 화면에서 즉시 Copilot에 접근할 수 있습니다.
    \item \textbf{Copilot can search the web} 기능을 활성화하면 실시간 웹 검색(Bing) 기반의 답변이 가능해집니다.
\end{enumerate}

\section{AI 모델 선택 가이드 (본 문서 작성일 기준)}
업무의 성격에 따라 최적화된 모델을 선택하여 효율성을 높일 수 있습니다.

\vspace{0.5em}
\noindent
\textit{※ 본 가이드는 일반적인 권장사항이며, 실제 모델 선택은 사용자의 업무 스타일, 선호하는 응답 형식, 업무의 난이도 등 개인별 상황에 따라 달라질 수 있습니다. 다양한 모델을 직접 사용해 보시고, 본인의 업무 환경에 가장 적합한 모델을 찾으시길 권장합니다.}
\vspace{0.5em}

\subsection{모델별 상세 사양}
\begin{xltabular}{\textwidth}{@{}l l X@{}}
\toprule
\textbf{등급} & \textbf{모델명} & \textbf{특징 및 주요 강점} \\
\midrule
\endhead

\textbf{일반} & GPT-5 mini & \textbf{빠른 속도와 이미지 분석}. 일상적인 문의, 문서 작성, 간단한 분석 등 일반적인 업무에 최적화된 모델입니다. 빠른 응답 속도를 제공하며, \textbf{이미지(사진, 도표, 차트) 분석 기능}을 지원합니다. \\
\addlinespace
\textbf{일반} & GPT-4.1 & \textbf{안정적인 범용 작업 + 이미지 분석}. 다양한 업무에 두루 활용할 수 있는 균형잡힌 모델입니다. 일관된 품질의 답변을 제공하며, \textbf{이미지 분석(비전) 기능}도 지원합니다. \\
\addlinespace
\textbf{일반} & Claude Haiku 4.5 & \textbf{대량 작업 처리}. 많은 양의 문서를 빠르게 처리해야 하거나, 반복적인 작업을 수행할 때 적합합니다. 고급 기능을 갖추었으면서도 신속한 응답이 필요한 상황에서 활용하십시오. \\
\midrule
\textbf{프리미엄} & Grok Code Fast 1 & \textbf{프로그래밍 전문}. 소프트웨어 개발 업무에 특화된 모델입니다. 코드 작성, 오류 수정, 프로그램 개선 작업에서 매우 빠른 응답 속도를 제공합니다. (프로그래밍 비전문가에게는 권장하지 않습니다) \\
\addlinespace
\textbf{프리미엄} & GPT-5 & \textbf{고급 문제 해결}. OpenAI의 최신 고성능 모델로, 복잡한 문제를 단계별로 분석하고 해결합니다. 질문의 난이도에 따라 자동으로 적절한 수준의 분석 깊이를 조절하여 답변합니다. \\
\addlinespace
\textbf{프리미엄} & GPT-5.1 & \textbf{장기 프로젝트 수행}. 여러 단계로 구성된 복잡한 작업을 체계적으로 수행하는 데 특화되었습니다. 외부 정보 검색, 구조화된 데이터 생성 등 고급 기능을 활용할 수 있으며, 장기간 진행되는 업무에 적합합니다. \\
\addlinespace
\textbf{프리미엄} & Claude Sonnet 4 & \textbf{이미지 분석 + 심층 추론}. 성능과 실용성의 균형이 뛰어난 모델입니다. \textbf{이미지 분석(비전) 기능}을 지원하며, 코딩 작업 흐름에 최적화되어 있습니다. \\
\addlinespace
\textbf{프리미엄} & Claude Sonnet 4.5 & \textbf{한국어 문서 작성 최적}. 가장 자연스러운 한국어 문장을 생성하는 모델입니다. 보고서, 기획서, 공문서 작성에 탁월하며, 복잡한 문제 해결 능력도 뛰어납니다. 에이전트 작업에 특화되어 있습니다. \\
\addlinespace
\textbf{프리미엄} & Claude Opus 4.1 & \textbf{학술 연구 및 심층 분석}. 가장 높은 수준의 논리적 사고가 필요한 작업에 특화되었습니다. 학술 논문 분석, 연구 계획 수립, 복잡한 정책 검토 등 깊이 있는 사고가 필요한 업무에 활용하십시오. \\
\addlinespace
\textbf{프리미엄} & Claude Opus 4.5 & \textbf{최신 최고 성능}. Anthropic의 최신 플래그십 모델로, Claude 시리즈 중 가장 강력한 추론 능력을 제공합니다. 극도로 복잡한 분석이나 창의적 업무에 적합합니다. \\
\addlinespace
\textbf{프리미엄} & Gemini 2.5 Pro & \textbf{대용량 문서 분석}. Google의 고성능 모델로, 매우 긴 문서(일반 책 약 10권 분량)를 한 번에 처리할 수 있습니다. 이미지 분석이 가능하지만, GitHub Copilot 웹 환경에서는 영상·음성 파일 업로드가 지원되지 않습니다. \\
\addlinespace
\textbf{프리미엄} & Gemini 3 Pro & \textbf{최신 멀티모달 (이미지만 지원)}. Google의 최신 모델로, 뛰어난 추론 능력과 이미지 분석 기능을 제공합니다. 단, \textbf{GitHub Copilot 웹 환경에서는 영상·음성 파일 업로드가 지원되지 않으며}, 이미지 분석만 가능합니다. \\
\bottomrule
\end{xltabular}

\subsection{업무 상황별 추천 모델}
\begin{itemize}
    \item \textbf{일상 업무 (이메일 작성, 간단한 질문, 요약 등)}: GPT-5 mini 또는 GPT-4.1 사용을 권장합니다. 빠른 응답이 필요하고 횟수 제한이 없어 부담 없이 활용하실 수 있습니다.
    \item \textbf{사진·도표 분석 (차트 해석, 문서 스캔, 이미지 속 텍스트 추출)}: GPT-5 mini, GPT-4.1, 또는 Claude Sonnet 4를 사용하십시오. 이 세 모델이 이미지(비전) 분석 기능을 지원합니다.
    \item \textbf{공식 문서 작성 (보고서, 기획서, 공문)}: Claude Sonnet 4.5를 권장합니다. 가장 자연스럽고 격식 있는 한국어 문장을 생성하며, 수정이 거의 필요 없습니다.
    \item \textbf{연구 및 심층 분석 (논문 검토, 정책 분석, 복잡한 의사결정)}: Claude Opus 4.1, Claude Opus 4.5, 또는 GPT-5.1을 활용하십시오. 깊이 있는 논리적 분석이 가능합니다.
    \item \textbf{프로그래밍 개발 (코드 작성, 오류 수정)}: Grok Code Fast 1, Claude Sonnet 4.5, GPT-5.1 중 선택하십시오. (※ 프로그래밍 전문가 대상)
    \item \textbf{동영상·음성 자료 분석}: \textbf{주의: GitHub Copilot 웹 환경(copilot.github.com)에서는 영상·음성 파일 업로드를 지원하지 않습니다.} 영상·음성 분석이 필요한 경우, Google의 Gemini 네이티브 서비스(gemini.google.com)를 별도로 이용해야 합니다.
    \item \textbf{대량 문서 처리 (다수의 파일 일괄 분석, 반복 작업)}: Claude Haiku 4.5를 권장합니다. 빠른 속도로 많은 양의 작업을 처리할 수 있습니다.
    \item \textbf{초대용량 문서 분석 (긴 보고서, 대량 자료)}: Gemini 2.5 Pro 또는 Gemini 3 Pro를 사용하십시오. 일반 책 약 10권 분량의 텍스트를 한 번에 처리할 수 있습니다.
\end{itemize}

\vspace{2em}
\hrule
\vspace{1em}

\section*{유의 사항}
\small
본 문서는 작성 시점(\docversion) 기준의 정보를 담고 있으며, GitHub Copilot의 정책, 모델 사양, 기능 등은 지속적으로 변경될 수 있습니다. 일부 정보에 오류가 있거나 최신 내용과 상이할 가능성이 있으므로, 정확하고 최신의 정보는 반드시 GitHub Copilot 공식 페이지(\href{https://github.com/features/copilot}{github.com/features/copilot}) 및 공식 문서(\href{https://docs.github.com/en/copilot}{docs.github.com/copilot})를 통해 확인하시기 바랍니다.
\end{document}